\section{Justificación}

El sistema de la informática  tiene su desafío  a partir de la decana de los ochentas, donde la sociedad con principios industrializados empieza a realizar cambios estructurales de organización y  comunicación, asumiendo nuevos roles de expansión de  sus productos y servicios.

El desarrollo de las tecnologías  de la informática y de las comunicaciones, impulsados por los avances de la electrónica, han cambiado al mundo moderno. Han incrementado sustancialmente el potencial humano y han establecido vínculos que acortaron las distancias e hicieron realidad un mundo globalizado.

La computación y la tecnología de la información rompieron las barreras organizacionales, introdujeron nuevos modelos de Marketing y dieron lugar a los negocios englobados bajo la denominación de e-Business, permitieron una nueva manera de modernizar los procesos Industriales y los Servicios y nuevas formas de gestión, basado en la disponibilidad de la información en tiempos adecuados, no solo para tomar oportunas decisiones, sino también para descubrir y aprovechar las oportunidades y anticiparse al futuro.

La población del Ecuador en el año 2001 fue de 12'156,608, de acuerdo a las proyecciones del 2005, será de 13'798,000 y para el 2010 será de 14'899,000 habitantes (CEPAR Estimaciones y proyecciones de población)

El Ecuador tiene repartido se población: en el año 2001, la zona rural tenia en 4'725,253 y en la urbana 7'431,355 habitantes. De acuerdo a las proyecciones de 2005, la zona rural es de 4'705,000 y en la rural 9'093,000 habitantes: Para el 2010 será de 4'649,000 y 10'250,000 habitantes y, para el 2015 será de 4'908,000 en la zona rural y 11'028,000 en la zona urbana de habitantes.

En la provincia de Manabí, su número de habitantes fue de  1'186.025, repartidos en 596,502 hombres y 589,523 mujeres. Portoviejo su capital  tenía una población de 171,847 habitantes (INEC. VI Censo Nacional  de Población y V de Vivienda)

Es importante reconocer que la población se encuentra concentrada en un 56\% en las Provincias de Guayas, Pichincha y Manabí y una dispersión en el resto de Provincias.

Al identificar la población por edad y por sexo, es necesario reconocer que existe un descenso en las tazas de fecundidad, la población  ecuatoriana se caracteriza por ser esencialmente joven. El 33.2 \% de la población es menor de 15 años y el 43.4\% menor a 20 años. La población de edad activa (de 15 a 65 años)  llega al 57\% y de los 65 años y más representa apenas el 4.3\%

Se considera que el sectores que están en capacidad de trabajar (15 a 65 años) crecerá, lo cual exigirá un mayor número de fuentes de trabajo (Censo 2001)

En relación a la Población Económicamente Activa (PEA) está comprendida  entre 18 y 65 años de edad, que están cumpliendo o que pueden cumplir actividades productivas.

La PEA en el Ecuador según el Censo del 2001 es de 4'585.575, de esta el 62\% corresponde a la PEA urbana y el 38\% a la rural.

Para mayo 2004 la tasa de desempleo es de 11.53\% y la de subempleo de 44\% Es importante reconocer que  la PEA por área de actividad, presenta una diferencia en cuanto al proceso técnico y demandas sociales; por ejemplo, en industria y manufactura el empleo es de 472.805, en transporte, almacenamiento y comunicaciones es de 227.789, en enseñanza 211. 318, en administración pública y defensa 169.271, actividad mobiliaria empresarial 131,790, intermediación financiera 31,429 entre otras (fuente. INEC VI censo nacional de población  y V de vivienda 2001).

El problema del desempleo y subempleo es un asunto estructural. El desempleo ha bajado, pero no es precisamente por que ha mejorado la situación, sino por la migración masiva, que provoca también la fuga de mano de obra calificada, hay regiones que se están quedando sin trabajadores  como la zona sur del país.

El desempleo y el subempleo impacta más en las mujeres y en los jóvenes y, las ramas de la economía más afectadas son la construcción, la manufactura y el servicio doméstico. 

Las personas que tienen trabajo también están en el centro de la acción de las políticas neoliberales. El gobierno enmarcándose en las políticas de flexibilización laboral,  estaría examinando la posibilidad el 15\% del reparto de las utilidades a los trabajadores. El sector público también ha sido afectado por la flexibilización laboral con la aprobación de la ``Ley Orgánica de Servicio Civil y Carrera Administrativa y la Unificación y Homologación de las remuneraciones del Sector Público'' que tiene como objetivo, el congelamiento incrementar en porcentajes no relacionados con la canasta familiar y en otros casos el congelamiento de los salarios de los servidores públicos.

Otro aspecto importante que justificó la dolarización fue precisamente la necesidad de controlar la inflación, aspecto que no se ha logrado totalmente. Durante 1999 la inflación mantuvo un crecimiento permanente que llegó a su máximo nivel en el 2000, ya en dolarización, pasando de 78,1\% en enero a 107,9\% en el mismo año. Desde octubre del 2000, la tendencia cambia, produciéndose una desaceleración y cerrando el año 2005 con una inflación del 2,3\%

A pesar que la inflación llega a un dígito, es superior a la de economías vecinas sin dolarización  de otros países de América del sur y centro, que tienen política cambiaria propia. La inflación ha provocado permanentes alzas de los precios, los salarios no guardan correspondencia con los costos de una canasta básica, la diferencia entre una y otra se denomina restricción 

La proyecciones a diciembre del 2005, la canasta familiar se ubicaría en 425,38 dólares, mientras que el salario vital está en 174,9. Es decir esa familia puede cubrir solamente el 41\% de la canasta básica considerada básica para subsistir. (Fuente: Banco Central del Ecuador; Análisis de Coyuntura 2003)

El deterioro de las condiciones de vida en el Ecuador, determina  que si  la pobreza ha bajado, aún  se mantiene el 85\%  de la población rural  y el 52\% en la urbana. sE estima que en Ecuador más del 70\% de los menores de 18 años son pobres. Más de medio millón de niños y jóvenes no tienen acceso al sistema escolar. Hay un déficit de 1.400.000 unidades habitacionales, cada año este déficit aumenta 200 mil unidades más.(Gestión agosto 2005 No 134 p 16)

En relación al sistema empresarial, datos obtenidos de la Superintendencia de Compañías, para el final del 2002 se encuentran registradas 28.745 empresas en el Ecuador, las mismas geográficamente muestran una gran concentración en dos provincias; Guayas y Quito. En Manabí  existen  702 empresas (diciembre 2002).

La Pequeña y Mediana Industria (PYMIS), se ha considerado que la  Industria no es una de las actividades económicas más importante del país, representa apenas el 2,6\%, del PIB nacional. El Ecuador se caracteriza por un alto números de medianas y pequeñas empresas que representan un alto porcentaje en el total y ocupa abundante mano de obra, pero su incidencia en las cuentas nacionales es mínima, registra que el 95\% de los establecimientos industriales tienen: Participación el en PIB industrial el 24\%; participación en la exportaciones el 5\% ; exportaciones a los Estados Unidos el 2\% (180 millones de dólares); generación de empleo directo e indirecto 1.000.000 de personas (fuente COPEI).

Con referentes a las Instituciones Financieras que operan en el Ecuador. Para el año 2005 las compañías que se encuentran bajo el control de la Superintendencia de Banco son\footnote{Fuente: Catastro de Entidades 2005. Superintendencia de Banco}: 

\begin{table}[h!]
\centering
\begin{tabularx}{9cm}{|X|c|} \hline
Instituciones Financieras Públicas	&  7 \\ \hline
Bancos Privados Nacionales 		& 19 \\ \hline
Bancos Privados Extranjeros 		&  6 \\ \hline
Bancos Off shore			& 22 \\ \hline
Sociedad Financiera 			& 29 \\ \hline
Tarjetas de Créditos  			&  5 \\ \hline
Administradoras de Fondos		& 10 \\ \hline
Casa de Cambios				& 23 \\ \hline
Cooperativas de Ahorro y Créditos	& 29 \\ \hline
Mutualistas				& 11 \\ \hline
Almacenes Generales de Depósitos	& 11 \\ \hline
Casas de Valores			& 10 \\ \hline
Seguros					& 38 \\ \hline
Aseguradoras Nacionales			&  6 \\ \hline
Inmobiliarias 				&  5 \\ \hline
Aseguradoras Extranjeras		&  6 \\ \hline
Reaseguradotas Nacionales 		&  2 \\ \hline
Cia de Titularización 			&  1 \\ \hline
Servicios de Computación 		&  7 \\ \hline
Entidades Operativas Extranjeras	&  3 \\ \hline
TOTAL DEL SISTEMA 			&152 \\ \hline
\end{tabularx}
\caption{Instituciones financieras que se encuentran bajo el control de la Superintendencia de Banco}
\end{table}

En base a la ley, la mayoría de los grupos financieros del Ecuador por lo general tienen entre sus actividades económicas  a las bancarias (matriz, \textit{off-shore}, operativos), almaceneras, tarjeta de crédito, inmobiliarias, casas de valores, administradoras de fondos y compañías de seguros. La participación de las subsidiarias off-shore  muestra el proceso de transnacionalización de los llamados grupos financieros.

\textit{Off-shore} en el extranjero\footnote{Fuente Superintendencia de Banco, datos actualizados a junio del 2003}:

\begin{itemize}
\item Produbank
\item Pichincha Ltd. Bahamas 
\item Guayaquil Bank Trust
\item Pacifico Panamá
\item Continental Overseas
\end{itemize}


Existen 10 grupos financieros con sus activos, patrimonios, utilidades en mayor auge en el Ecuador.  Pichincha, Pacífico, Produbanco, Guayaquil, Bolivarano, Internacional, Austro, Austro, Diners Club, Citabank, Enlace. (fuente: Revista Gestión, N 132, junio 2005. p.20)

Al momento, en la tercera revolución científica técnica, sobre todo en el campo de la informática, sobre todo en el campo de la informática, la cibernética, las ciencias de la vida y de los materiales, coloca a la humanidad ante la posibilidad de guiar su destino, la vinculación de estos resultados al control de los intereses particulares bloquea esta proyección y coloca a la humanidad ante riesgos de supervivencia. Con la  globalización los temas de la información, la comunicación y el conocimiento se proyectan como cuestiones culturales y civilizatorias.

La revolución  científico-técnica, considera a la informática y la cibernética, como el poder del desarrollo viviente, productivo y de comunicación,  han abierto poderes ilimitados  para el ser humano  que ejercita   y administra  la comunicación con el propósito de  reducir el tiempo de consenso y control de  de la vida. 

En la actualidad se reconoce  la importancia de la educación para promover el bienestar y reducir las desigualdades  sociales. En el caso del Ecuador desde 1960 se han hecho intentos por expandir la cobertura educativa, especialmente en el área  de la educación pública 

La educación es importante por que  impacta a todos los ámbitos de la vida, en la productividad laboral, en la participación social y en general en el mejoramiento de la calida de vida; sin embargo, su acceso no ha sido igual para todos, depende en muchos casos de la capacidad económica  y de las oportunidades individuales. Los sectores con mayor oportunidad de  educarse  han sido  los  urbanos. La educación no es solo un derecho humano y una responsabilidad social sino una condición básica para cualquier proceso de desarrollo.

La calidad  de la educación  tiene que ver con  las condiciones de vida, de trabajo, de seguridad, de pobreza, que afecta a la mayoría de los hogares ecuatorianos. Uno de los problemas  que afecta a la crisis es la deserción. En  ciclo primario, en el área urbana es el 11\% y en el área rural el 53\%. 

Otro elemento, es el índices de repitencia en el nivel primario con un 7.7 años para concluir la escuela, lo que le representa al Estado un gasto  extra del 28\%. Es grave también la limitada capacidad de permanencia en el sistema escolar, solo el 56 de cada 100 estudiantes que se matriculan en primer grado llegan al colegio y de estos solamente 14 terminan el mismo.

Actualmente el 69\% de la matrícula esta atendida por la educación pública, el 26,5\% por la educación particular ya sea laica o religiosa, el 4\% por la educación fiscomicional y el 0.5\% por la municipal

Según SINEC, existen en el Ecuador alrededor de 3.900.000 niños/as  y jóvenes en edad escolar, de estos, solamente 3,184.266 se encuentran en establecimiento primarios y medios en un total de 27.835 planteles  en los que laboran 187.997 profesores/as.

El  presupuesto asignado a la educación incide en su calidad. En el 2005 se asignó  apenas el 13.60\% y la anterior y actual constitución dispone el 30\%. Es conocido que la tendencia a nivel mundial es elevar la inversión en la educación, existe demanda de la mano de obra calificada. El Estado Ecuatoriano apenas destino el 2.3\% del PIB para este sector, se espera que para el año 2010 se aumente al 8.4\%.

El sistema educativo a nivel segundario  muestra similares problemas, el acceso a este nivel tiene una serie de limitaciones, acentuándose más en las áreas rurales. Cuatro de cada diez personas mayores de edad habían concluido sus estudios segundarios  y en la zona rural  uno de cada diez personas habían terminado el nivel de estudio (Información SIISE convenio BID-Ministerio del Frente Social 2001)

La crisis económica social también  ha afectado a las universidades, debido a variuas razones, entre ellas: el estancamiento a la producción de conocimiento; el retroceso en las metodologías y la pedagogía, la falta de investigación, la ruptura con la sociedad, sobrecarga de estudiantes en determinadas carreras; y, el ahogamiento económico por la falta de presupuesto.

La población de 24 años o más que alguna vez ha ingresado a estudios en la universidad fue del 14\%  en el 2001, pero solamente 8\% tiene certificado de egresado y un 6\% título universitario.

En el Ecuador existen  actualmente 72 Universidades  y Escuela Politécnicas, de la cuales 26 son Estatales y 46 Particulares  o Privadas, existen 352 Institutos Superiores registrados en el CONESUP. El porcentaje de estudiantes en carreras como educación, humanidades, ciencias sociales y administración es del 61 \%; en ciencias de la salud, ciencias naturales, ciencias exactas y ciencias agropecuarias 22\%, y en ingeniería y tecnología 15\%. (fuente: Suplemento Institucional N 12, 26 de julio 2005, p, 1)

En la Provincia de Manabí existen seis universidades, con diferentes  y similares ofertas académicas, que consagran sus esfuerzos institucionales a la construcción de ciencia, al desarrollo de la técnica y la tecnología. Sin embrago, la creciente demanda de  profesionales que hacen conocer las empresas y las instituciones de desarrollo productivo, financiero, cultural y ambiental y, los avances que de denuncia la práctica profesional en cada disciplina, permite que cada universidad realice permanentes investigaciones  con el propósito  de responder a estas necesidades. 

Manabí es una de la provincia de desarrollo  en áreas de servicios, producción y financiero, comparte sus actividades y beneficios  con su entorno local, nacional y extranjero. Adquiere permanentes responsabilidades de carácter técnico, por lo que su estado es creciente, demandando permanentes necesidades en lo tecnológico y con variadas oportunidades trabajo profesional.

El rol de la ingeniería en el campo de la informática, tiene los nuevos escenarios económicos, sociales y tecnológicos y se basan en la gestión de la tecnología, del conocimiento y de la información que deben atravesar las diferentes fases: innovación, desarrollo, transferencia, adopción, vigilancia, marketing entre otros, de las tecnologías de la información comunicación y conocimiento.

Nuestra principio filosófico de basa en la expansión de servicios  profesionales de calidad y que aporten al   crecimiento y bienestar  social, nos esmeramos al cumplimiento de nuestra misión y realizamos toda clase de gestión para llegar a la visión; de tal manera, que nuestro encargo es formar personas pensantes, líderes  en su  propio crecimiento intelectual, libres en la construcción de nuevos conocimientos, sustentados en valores y respeto a la sociedad. Un profesional que reconozca su  entorno como interés de cambio, de innovación técnica y tecnológica, que facilite la participación y creación  de nuevos enfoques de trabajo y permita la el desarrollo armónico de la sociedad en su conjunto.

Estudios realizados en el campo empresarial, financiero y de servicios generales en nuestra provincia, han dado como resultado, que  en la  última década,  se ha incrementado cuantitativamente el sistema  de control, registro, comunicación y gestión en estás empresas; con el propósito de monitoriar, replantear o proyectar nuevas estrategias de producción y productividad;  por lo tanto, su capacidad estructural - administrativa  necesita ser fortalecida con nuevos programas informáticos, que permita un sistema de alta disponibilidad y rendimiento, para hacer factible cada intervención de su propuesta funcional e investigativa.

También es importante reconocer en el presente acercamiento empresarial,  que cada instancia de gestión y administración, presenta carencias en los procesos del sistema informático  y comunicación, que  la capacidad de innovación  y de investigación se encuentran retrazadas o detenidas en algunos casos, por la deficiente  aplicación de nuevos programas informáticos, que están  siendo conducidos por sistemas de software, telemática y gestión de ingeniería tecnológica.

Estos referentes, va a permitir que la ingeniería en sistemas debe enfrentar cálculos complejo de diseño en el hardware, software, firmware y comunicaciones optimizando los diferentes tiempos de respuestas, tomando en consideración, las operaciones que se desarrollarán con equipos que cubran el área geográfica  de producción o servicios.

Nuestra propuesta técnica -académica  se apoya en las necesidades y requerimiento de los sectores productivos, financieros y de servicios generales, sin descuidar el desarrollo intelectual y cultural de los estudiantes, con niveles de formación dinámicos y progresivos, que consolide  en proceso cada espacio de creación teórica-práctica y el  permanente avance en los técnicas de investigación científica aplicada a la informática 


