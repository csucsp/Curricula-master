\section{Perfil de admisión}
Para ser admitidos a la Carrera Profesional de Ciencias en la Computación, los estudiantes deben 
presentar el título de bachillerato en Ciencias o Técnicos en Informática, Comercio, Administración, 
Secretariado y demás obligaciones documentadas de identificación personal que dispone el Reglamento 
Académico de la Universidad.

\section{Modalidad de estudios}
El proceso educativo teórico-práctico será presencial, apoyado por actividades de vinculación e investigación
operativa.

\section{Sistema de estudio}
La actividades académicas se desarrollarán por niveles (10), el componente educativo será por eventos, 
los mismos están estructurados en el Pensum de Estudios con el número de crédito correspondiente 
para el proceso de interacción docente-estudiantes.

\section{Duración de la carrera}
La carrera Profesional en su formación académica, tendrá una duración de diez niveles, a esto se 
le incorpora un semestre de diseño, gestión e informe de investigación previo a su titulación.

\section{Modelo educativo}
Se desarrolla en un marco de procesos Analítico-crítico-reflexivo-contextual, que permite el reconocimiento 
de los avances de la ciencia, la técnica y la tecnología, dando oportunidades de elevar el 
nivel de propuestas y proyecciones, que favorecen al desarrollo sistemático de los sectores de servicio 
o producción tecnológicas.

\section{Perfil del docente}
Los profesionales que se desempeñaran como docentes en la Carrera de Ciencia de la Computación 
deben tener título profesional universitario afín a la disciplina del conocimiento, a demás deben poseer 
el siguiente perfil:

\begin{itemize}
\item Conocimientos en el área de su desempeño,
\item Capacidad analítica y reflexiva,
\item Disponibilidad para integrarse a grupos académicos,
\item Conocimientos pedagógicos-didácticos,
\item Conocimientos en investigación formativa,
\item Dominio de los sistemas y equipos de comunicación tecnológica.
\end{itemize}