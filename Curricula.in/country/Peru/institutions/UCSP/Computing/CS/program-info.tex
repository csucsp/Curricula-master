\newcommand{\DocumentVersion}{2016}
\newcommand{\fecha}{\today}
\newcommand{\city}{Arequipa\xspace}
\newcommand{\country}{Peru\xspace}
\newcommand{\dictionary}{Español\xspace}
\newcommand{\SyllabusLangs}{Español,English}
\newcommand{\GraphVersion}{2\xspace}

\newcommand{\CurriculaVersion}{2016\xspace} % CS#-dependencies.tex: Malla 2006: 1, Malla 2010: 2, 
\newcommand{\YYYY}{2018\xspace}             % Plan 2006, 2018: Plan2018, 2021: Plan 2021
\newcommand{\Range}{1-10}                   % Plan 2010 1-8, Plan 2006 7-10   %   

           % Plan 2016 1-8, Plan 2006 7-10
\newcommand{\equivalences}{} %  {2006,2010}
\OutcomesVersion{V2}
\OutcomesList{V1}{a,b,c,d,e,f,g,h,i,j,k,l,m,n,o,p}
\OutcomesList{V2}{1,2,3,4,5,6,7}

\newcommand{\FacultadNameES}{Facultad de Ingeniería y Computación}
\newcommand{\FacultadNameEN}{School of Engineering and Computing}
\newcommand{\FacultadName}{}

\newcommand{\DepartmentNameES}{Ciencia de la Computación\xspace}
\newcommand{\DepartmentNameEN}{Department of Computer Science\xspace}
\newcommand{\DepartmentName}{\DepartmentNameES}

\newcommand{\SchoolShortNameES}{Ciencia de la Computación\xspace}
\newcommand{\SchoolShortNameEN}{Computer Science\xspace}
\newcommand{\SchoolShortName}{\SchoolShortNameES}

\newcommand{\SchoolFullNameES}{Escuela Profesional de Ciencia de la Computación\xspace}
\newcommand{\SchoolFullNameEN}{Undergraduate Program in Computer Science\xspace}
\newcommand{\SchoolFullName}{\SchoolFullNameES}

\newcommand{\SchoolFullNameBreakES}{Escuela Profesional de\\Ciencia de la Computación\xspace}
\newcommand{\SchoolFullNameBreakEN}{Undergraduate Program in\\Computer Science\xspace}
\newcommand{\SchoolFullNameBreak}{\SchoolFullNameBreakES}

\newcommand{\PosterTitle}{}

\newcommand{\SchoolAcro}{PPCS\xspace}
\newcommand{\SchoolURL}{\href{http://cs.ucsp.edu.pe}{http://cs.ucsp.edu.pe}\xspace}
\newcommand{\underlogotext}{}

\newcommand{\AcademicDegreeIssuedES}{Bachiller en Ciencia de la Computación\xspace}
\newcommand{\AcademicDegreeIssuedEN}{Bachelor in Computer Science\xspace}
\newcommand{\AcademicDegreeIssued}{\AcademicDegreeIssuedES}

\newcommand{\TitleIssuedES}{Licenciado en Ciencia de la Computación\xspace}
\newcommand{\TitleIssuedEN}{Professional in Computer Science\xspace}
\newcommand{\TitleIssued}{\TitleIssuedES}

\newcommand{\AcademicDegreeAndTitleES}%
{\begin{description}%
\item [Grado Académico: ] \AcademicDegreeIssued\xspace y% 
\item [Titulo Profesional: ] \TitleIssued%
\end{description}%
}

\newcommand{\AcademicDegreeAndTitleEN}%
{\begin{description}%
\item [Academic degree: ] \AcademicDegreeIssuedEN\xspace y% 
%\item [Titulo Profesional: ] \TitleIssued%
\end{description}%
}
\newcommand{\AcademicDegreeAndTitle}{\AcademicDegreeAndTitleES}

\newcommand{\doctitle}{Plan Curricular \YYYY\xspace del \SchoolFullName\\ \SchoolURL}

\newcommand{\AbstractIntro}{Este documento representa el informe final de la 
malla curricular \YYYY\xspace de la \SchoolFullName~de
la \University~(\textit{\InstitutionURL}) en la ciudad de \city-\country.}

\newcommand{\OtherKeyStones}%
{Un pilar que merece especial consideración en el caso de la \University~es el aspecto de 
valores humanos, básicos y cristianos debido a que forman parte fundamental 
de los lineamientos básicos de la existencia de la institución.\xspace}

\newcommand{\profileES}{%
El perfil profesional puede ser mejor entendido a partir de
\OnlyMainDoc{la Fig.~\ref{fig.cs} (Pág.~\pageref{fig.cs})}\OnlyPoster{las figuras del lado derecho}. 
Este profesional tiene como objetivo principal ser el impulsor del desarrollo de nuevas 
tecnologías computacionales con calidad internacional que puedan ser útiles a nivel local, nacional e internacional.
Nuestro perfil profesional también está orientado a ser generador de puestos de empleo a través de la innovación permanente. 
Nuestra formación profesional tiene 3 pilares fundamentales: 
un contenido computacional de acuerdo a normas internacionales (CS2013), una orientación marcada a la innovación ambos enriquecidos por una sólida
Formación Humana.
}
\newcommand{\profileEN}{%
The professional profile can be better understood from
\OnlyMainDoc{the Fig.~\ref{fig.cs} (Page~\pageref{fig.cs})}\OnlyPoster{the figures on the right side}.
This professional's main objective is to be the promoter of the development of new computational technologies with international quality that can be useful at a local, national and international level.
Our professional profile is also geared towards generating jobs through permanent innovation.
Our professional training has 3 fundamental pillars: a computational content according to international standards (CS2013), a marked orientation to innovation, 
both enriched by a solid Human Education.
}
\newcommand{\profile}{\profileES}

\newcommand{\missionES}{La Universidad Católica San Pablo es una comunidad académica animada por las orientaciones y vida de la Iglesia Católica que, 
a la luz de la fe y con el esfuerzo de la razón, busca la verdad y promueve la formación integral de la persona mediante actividades 
como la investigación, la enseñanza y la extensión, para contribuir con la 
configuración de la cultura conforme a la identidad y despliegue propios del ser humano.\xspace}

\newcommand{\missionEN}{Universidad Católica San Pablo is an academic community animated by the orientations and life of the Catholic Church that, 
in the light of faith and with the effort of reason, 
seeks the truth and promotes the integral formation of the person through activities 
such as research, teaching and extension, to contribute to the 
configuration of culture according to the identity and deployment of the human being.\xspace}
\newcommand{\mission}{\missionES}

\newcommand{\visionEN}{\missionEN}
\newcommand{\visionES}{\missionES}
\newcommand{\vision}{\visionES}
