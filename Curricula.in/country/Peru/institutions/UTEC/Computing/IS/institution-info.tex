\newcommand{\DocumentVersion}{2010}
\newcommand{\fecha}{\today}
\newcommand{\city}{Lima\xspace}
\newcommand{\country}{Perú\xspace}
\newcommand{\dictionary}{Español\xspace}
\newcommand{\SyllabusLangs}{Español} %,English
\newcommand{\GraphVersion}{2\xspace}

%Para UTEC 2018
\newcommand{\CurriculaVersion}{2020\xspace} % Malla 2006: 1, Malla 2010: 2
\newcommand{\YYYY}{2020\xspace}             % Plan 2006
\newcommand{\Range}{1-10}                   % Plan 2010 1-8, Plan 2006 7-10
\newcommand{\Semester}{2020-I\xspace}      

% newcommand{\OutcomesList}{a,b,c,d,e,f,g,h,i,j,k,l,m,HU,FH,TASDSH}
\OutcomesVersion{V2}
\OutcomesList{V1}{a,b,c,d,e,f,g,h,i,j,k,l,m,n,o,p}
\OutcomesList{V2}{1,2,3,4,5,6,7,8,9,10}

\newcommand{\logowidth}{6.3cm}
\newcommand{\InstitutionURL}{\htmladdnormallink{http://www.utec.edu.pe}{http://www.utec.edu.pe}\xspace}

\newcommand{\UniversityES}{Universidad de Ingenieríay Tecnología (UTEC)\xspace}
\newcommand{\UniversityEN}{Universidad de Ingenieríay Tecnología (UTEC)\xspace}
%newcommand{\University}{\UniversityES}
\newcommand{\University}{Universidad de Ingenieríay Tecnología (UTEC)\xspace}

\newcommand{\FacultadNameES}{Facultad de Computación\\ \xspace}
\newcommand{\FacultadNameEN}{}
\newcommand{\FacultadName}{\FacultadNameES}

\newcommand{\DepartmentNameES}{Departamento de Computación\xspace}
\newcommand{\DepartmentNameEN}{Department of Computing\xspace}
\newcommand{\DepartmentName}{\DepartmentNameES}

\newcommand{\SchoolShortNameES}{Sistemas de Información\xspace}
\newcommand{\SchoolShortNameEN}{Information Systems\xspace}
\newcommand{\SchoolShortName}{\SchoolShortNameES}

\newcommand{\SchoolFullNameES}{Escuela Profesional de \SchoolShortNameES}
\newcommand{\SchoolFullNameEN}{School of \SchoolShortNameEN}
\newcommand{\SchoolFullName}{\SchoolFullNameES}

\newcommand{\SchoolFullNameBreakES}{Escuela Profesional de \\ \SchoolShortNameES\xspace}
\newcommand{\SchoolFullNameBreakEN}{School of \SchoolShortNameEN\xspace}
\newcommand{\SchoolFullNameBreak}{\SchoolFullNameBreakES}

\newcommand{\PosterTitle}{}

\newcommand{\SchoolAcro}{EPCC\xspace}
\newcommand{\SchoolURL}{\htmladdnormallink{http://cs.utec.edu.pe}{http://cs.utec.edu.pe}\xspace}
\newcommand{\underlogotext}{}

\newcommand{\AcademicDegreeIssued}{Bachiller en Ciencia de la Computación\xspace}
\newcommand{\TitleIssued}{Licenciado en Ciencia de la Computación\xspace}
\newcommand{\AcademicDegreeAndTitle}%
{\begin{description}%
\item [Grado Académico: ] \AcademicDegreeIssued\xspace y% 
\item [Titulo Profesional: ] \TitleIssued%
\end{description}%
}

\newcommand{\doctitle}{Plan Curricular \YYYY\xspace de la \SchoolFullName\\ \SchoolURL}

\newcommand{\AbstractIntro}{Este documento representa el informe final de la nueva 
malla curricular \YYYY del \SchoolFullName de la \University (\textit{\InstitutionURL}) 
en la ciudad de \city-\country.}

\newcommand{\OtherKeyStones}{}


\newcommand{\profileES}{%
El perfil profesional de este programa profesional puede ser mejor entendido a partir de
\OnlyMainDoc{la Fig. \ref{fig.cs} (Pág. \pageref{fig.cs})}\OnlyPoster{las figuras del lado derecho}. 
Este profesional tiene como centro de sus estudios a la computación. Es decir, tiene a la computación 
como fin y no como medio. De acuerdo a la definición de esta área, este profesional está llamado 
directamente a ser un impulsor del desarrollo de nuevas técnicas computacionales que 
puedan ser útiles a nivel local, nacional e internacional.

Nuestro perfil profesional está orientado a ser generador de puestos de empleo a través de la innovación permanente. 
Nuestra formación profesional tiene 3 pilares fundamentales: 
un contenido de acuerdo a ACM/IEEE-CS Computing Curricula CS2013 y CC2020
un contenido de acuerdo a normas internacionales, una orientación marcada a la innovación y formación humana.
}

\newcommand{\profileEN}{The professional profile of this professional program can be better understood from 
\OnlyMainDoc{la Fig. \ref{fig.cs} (P. \pageref{fig.cs})} \OnlyPoster{figures on the right side}. 
This professional has Computing as the center of his studies. That is, it has computating as an end and not as a means. 
According to the definition of this area, this professional is called directly to be a promoter of 
the development of new computational techniques that can be useful at local, national and international level.

Our professional profile is aimed at generating jobs through permanent innovation. Our professional training has three fundamental pillars: 
a content according to ACM/IEEE-CS Computing Curricula CS2013 and CC2020, a marked orientation to innovation and human/soft skills.
}

\newcommand{\missionES}{Contribuir al desarrollo científico, tecnológico y técnico del país  
formando profesionales competentes, orientados a la creación de nueva 
ciencia y tecnología computacional, como motor que impulse y consolide la industria 
del software en base a la investigación científica y tecnológica en 
áreas innovadoras formando, EN NUESTROS profesionales, un conjunto de habilidades y 
destrezas para la solución de problemas computacionales con un compromiso social.\xspace}

\newcommand{\missionEN}{To contribute to the scientific, technological and technical development of the country 
forming competent professionals oriented to the creation of new science and computational technology, 
as engine that impels and consolidates the software industry based on 
scientific research and technological in innovative areas, forming, IN OUR professionals, 
a set of skills for solving computational problems with a social commitment.\xspace}
