\chapter{Evaluación de Recursos}\label{chap:cs-resources}

Los recursos considerados para la creación de esta nueva carrera están relacionados con la plana docente, infraestructura, laboratorios y financiamiento.

% \OnlyPeruUNSA{\section{Plana Docente}\label{sec:staff}
Los docentes con los cuales ya se cuenta por adhesión voluntaria para la nueva \SchoolFullName son los siguientes:

\begin{enumerate}
\item Lic. Wilber Ramos Lovón (docente principal del DAISI)
\item Mag. Jesús Silva Fernández (docente asociado del DAISI)
\item Lic. Percy Huertas Niquén (docente asociado del DAISI)
\item Dr.  Ernesto Cuadros-Vargas (docente auxiliar del DAISI)
\item Dr.  Luis Alfaro Casas (docente principal del DAISI)
\item Dr. César Beltrán Castañón (Dr. en Ciencia de la Computación, Universidad de S\~ao Paulo, Brasil 2007).
\item MSc. Juan Manuel Gutiérrez Cárdenas (docente contratado del DAISI)
\item Ing. Elizabeth Vidal Duarte (docente auxiliar del DAISI)
\item Ing. Eveling Castro Gutiérrez (docente auxiliar del DAISI)
\item Ing. Alfredo Paz Valderrama (docente contratado del DAISI)
\item Ing. Eliana Adriazola (docente contratado del DAISI)
\item MSc. Juan Carlos Gutiérrez Cáceres (docente contratado del DAISI)
\end{enumerate}

Los docentes que pertenecen al DAISI ya han manifestado abierta y voluntariamente su deseo de formar parte de la nueva carrera. Ademas hemos se cuenta con el interés manifiesto de:

\begin{enumerate}
\end{enumerate}

Además, es necesario recalcar que también se ha hecho contacto con los egresados de la EPIS que se encuentras en diversos países y que están dispuestos a ayudar en este proyecto. Este personal docente nos permitiría cubrir la necesidad inicial.

Sin embargo también hemos contactado a varios egresados de diversos países que se han mostrado dispuestos a venir y aportar a su \textit{alma matter}. Este grupo de profesionales estaría conformado por:.

\begin{enumerate}
\item Dr. Eduardo Tejada Gamero (Ex-alumno de la EPIS UNSA y alumno de doctorado en la University of Stuttgart-Alemania
\item Dra Nelly Condori (Ex-alumna de la EPIS UNSA, obtuvo su doctorado en la Universidad Politécnica de Valencia en España en Enero de 2007)
\item MSc. Jesús Mena Chalco (Ex-alumno de la EPIS UNSA, ganador del premio a la mejor tesis de maestría en el Centro Latinoamericano de Informática (CLEI), actualmente está realizando sus estudios de doctorado en la Universidad de Sao Paulo Brasil)
\item MSc. Dennis Barrios Aranibar (Ex-alumno de la EPIS UNSA y actualmente estudiante de doctorado de la Universidad Federal de Rio Grande do Norte Brasil).
\item MSc. Raquel Patiño (Ex-alumna de la EPIS UNSA y actualmente estudiante de doctorado de la Universidad Federal de Rio Grande do Norte Brasil).
\item MSc. Wilfredo Puma (Ex-alumno de la EPIS UNSA y actualmente estudiante de doctorado de la Universidad de Campinas-Brasil).
\item MSc. Alvaro Cuno Parari (Ex-alumno de la EPIS UNSA y actualmente estudiante de doctorado de la Universidad Federal de Rio de Janeiro-Brasil).
\item MSc. Manuel Loayza (Ex-alumno de la EPIS UNSA y actualmente estudiante de doctorado de la Pontificia Universidad de Rio de Janeiro-Brasil).
\item MSc. Michell Quintana Truyenque (Ex-alumno de la EPIS UNSA y actualmente estudiante de doctorado de la Pontificia Universidad de Rio de Janeiro-Brasil).
\item MSc. Karl Apaza Aguero (Ex-alumno de la EPIS UNSA y actualmente estudiante de doctorado de la Universidad Federal de Rio de Janeiro-Brasil).
\item MSc. Oscar Pacheco Calsin,(Ex-alumno de la EPIS UNSA y actualmente estudiante de doctorado de la Universidad Federal de Rio Grande do Sul-Brasil).
\item MSc. Carlos Rossel (Ex-alumno de la EPIS UNSA y actualmente estudiante de doctorado de la Universidad Nacional Autónoma de México).
\item MSc. Edison Aurelio Rios Pacheco (Ex-alumno de la EPIS UNSA y actualmente estudiante de posgrado en L`UniversitÉ d` OrlPÉans-Francia).
\item Ing. Christian López del Álamo (ex-docente contratado del DAISI).
\end{enumerate}
}
\OnlyPeruSPC{%
\section{Plana Docente}
En este punto debemos especificar la plana docente con la que se cuenta para esta carrera. 
}

\section{Infraestructura}\label{sec:cs-infraestructura}
Esta nueva carrera podría inicialmente compartir la infraestructura de la Facultad de Producción y Servicios de la UNSA. Sin embargo, en un mediano o largo plazo es necesario poder contar con un espacio propio ya que una de las principales tareas de esta nueva propuesta es la de servir como incubadora de empresas de base tecnológica. 

Desde este punto de vista, el tema de infraestructura no debe representar un problema, al menos en la etapa inicial.
Por otro lado, se cuenta con el apoyo tanto de la EPIS como del DAISI para resolver este punto en aprticular.

\section{Recursos para dictado de clases}
Un profesional innovador debe estar al tanto de los últimos avances de su área siempre. Los últimos avances de esta área no son presentados en los libros necesariamente. Debemos utilizar publicaciones de revistas indexadas de circulación mundial. Por esa razón tomamos como base la suscripción institucional a la ACM y a la IEEE-CS. Es recomendado que el docente use este material para discutir en clase las tendencias en todas las áreas.

\section{Laboratorios}\label{sec:cs-labs}
Al igual que el item anterior, los laboratorios podrían ser inicialmente compartidos con la Facultad de Ingeniería de Producción y Servicios. Para años posteriores es necesario considerar la implementacion de 3 laboratorios de 25 máquinas cada uno. El recurso físico en cuanto a laboratorios para cada curso está detallado en el Capítulo \ref{chap:laboratories}.

\section{Presupuesto}\label{sec:cs-budget}
\subsection{Justificación Económica}
La propia naturaleza de la carrera de Ciencia de la Computación orientada a la innovación y creación de tecnología abre un gran abanico de posibilidades para generar recursos para la universidad y el país.

Es a partir de este tipo de profesionales que podemos generar tecnología que resulte atractiva para a la inversión extranjera y consecuentemente ayude al desarrollo de la industria de software que aún es precaria en nuestro país.

Sólamente como referencia podemos mencionar que una única empresa de este rubro como es Google tiene ingresos anuales que bordean los 9 mil millones de dólares mientras que el presupuesto nacional de nuestro país esta al borde de los 63 mil millones.

Vale la pena resaltar en este aspecto la colaboración abierta de profesionales de Google hacia este proyecto considerando que es la única forma de generar industria de software de calidad internacional.

\subsection{Consideraciones de orden social}
En relación a las consecuencias sociales de este proyecto debemos resaltar que, generar tecnología de nivel competitivo internacionalmente, provocaría mayores inversiones extranjeras y consecuentemente una mayor cantidad de puestos de trabajo.

La proyección internacional de este perfil internacional también permite aumentar nuestras redes de colaboración con el extranjero. Esto ayuda a que podamos seguir enviando alumnos a estudiar a otros países y también facilita el intercambio y capacitación de docentes. Este tipo de movimiento internacional de nuestro recurso humano hará posible contar con un capital profesional altamente capacitado que permita desarrollar proyectos de gran envergadura a nivel nacional e internacional.

Este perfil profesional permitiría también resolver problemas complejos que nuestra ciudad o región presentan como el modelamiento del tráfico vehicular, detección de desplazamiento de la población para planear rutas de transporte, planeamiento de crecimiento urbano mediante simulaciones, entre otros.

% \OnlyPeruUNSA{\section{Primer año de funcionamiento}

En cuanto a la infraestructura inicial, se cuenta con el apoyo de la Dirección de la Escuela Profesional de Ingenieríade Sistemas (EPIS) para su funcionamiento inicial. Especificamente, se ha coordinado para que se utilice el aula 302 a partir de las 1:00pm hasta las 8:45pm.

Sin embargo, para un total funcionamiento de la carrera se necesita contar con un pabellón propio con un mínimo de 5 aulas de 40 espacios, 4 laboratorios de 24 máquinas cada uno, una biblioteca y ambientes para que la plana docente desarrolle y dirija proyectos de investigación, producción y servicio social.

% \OnlyPeruUNSA{\input{base-tex/\currentarea-UNSA-Plan-de-funcionamiento-por-areas}}}

