\begin{syllabus}

\course{CS4002. Capstone Project I}{Obligatorio}{CS4002}
% Source file: ../Curricula.in/lang/English/cycle/2020-I/Syllabi/Computing/CS/CS402.tex

\begin{justification}
This course aims to allow the student to carry out a study of the state of the art of a topic chosen by the student for his thesis.
\end{justification}

\begin{goals}
\item That the student carries out an initial investigation in a specific subject realizing the study of the state of the art of the chosen subject.
\item That the student shows mastery in the subject of the line of investigation chosen
\item That the student choose a teacher who dominates the research chosen as an advisor. 
\item The deliverables of this course are:
	\begin{description}
		\item [Avance parcial:] Solid bibliography and progress of a Technical Reporto.
		\item [Final:] Technical Report with preliminary comparative experiments that demonstrate that the student already knows the existing techniques in the area of his project and choose a teacher who dominates the area of his project as an adviser of his project.
	\end{description}
\end{goals}



\begin{unit}{Lifting the state of the art}{}{ieee,acm,citeseer}{60}{e,h,i,l}
  \begin{topics}
      \item Perform an in-depth study of the state of the art in a certain topic in the area of Computation.
      \item Writing technical articles in computing.
  \end{topics}
  \begin{learningoutcomes}
      \item Make a bibliographical survey of the state of the art of the chosen subject (this probably means 1 or 2 chapters of theoretical framework in addition to the introduction that is chapter I of the thesis) [\Usage]
      \item Writing a latex document in paper format with higher quality than Project I (master tables, figures, equations, indices, bibtex, cross references, citations, pstricks) [\Usage]
      \item Try to make presentations using prosper [\Usage]
      \item Show basic experiments [\Usage]
      \item Choose an advisor who dominates the research area [\Usage]
   \end{learningoutcomes}
\end{unit}

\begin{coursebibliography}
\bibfile{Computing/CS/CS401}
\end{coursebibliography}

\end{syllabus}
