\begin{syllabus}

\course{CS3101. Competitive Programming }{Obligatorio}{CS3101}
% Source file: ../Curricula.in/lang/English/cycle/2020-I/Syllabi/Computing/CS/CS311.tex

\begin{justification}
    Competitive Programming combines problem-solving challenges with the fun of competing with others. It teaches participants to think faster and develop problem-solving skills that are in high demand in the industry.
    This course will teach you to solve algorithmic problems quickly by combining theory of algorithms and data structures with practice solving problems.
  \end{justification}
  
  \begin{goals}
    \item That the student uses techniques of data structures and complex algorithms..
    \item That the student apply the concepts learned for the application on a real problem.
    \item That the student investigate the possibility of creating a new algorithm and / or new technique to solve a real problem.
  \end{goals}
  
  
  
  \begin{unit}{Introduction}{}{Cormen2009,Steven09,Kulikov09,SkienaRevilla:PC:2003,Laaksonen17,aziz2012elements}{20}{a,b,h}
    \begin{topics}
      \item Introduction to Competetive Programming
      \item Computacional model
      \item Runtime and space complexity
      \item Recurrence and recursion
      \item Divide and conquer
    \end{topics}
    \begin{learningoutcomes}
          \item Identify and learn how to use the resources in the Random Access Machine (RAM) computacional model. [\Usage] 
          \item Compute the runtime and space complexity for written algorithms. [\Usage]
          \item Compute the recurrence relations for recursive algorithms. [\Usage]
          \item Solve problems related to searching and sorting. [\Usage]
          \item Learning to select the right algorithms for divide-and-conquer problems. [\Usage] 
        \item Design new algorithms for real-world problem solving.[\Usage]
    \end{learningoutcomes}
  \end{unit}
  
  \begin{unit}{Data structure}{}{Cormen2009,Steven09,Kulikov09,SkienaRevilla:PC:2003,Laaksonen17,aziz2012elements}{20}{a,b,h}
    \begin{topics}
      \item Arrays and strings problems
      \item Linked lists problems
      \item Stacks and queues problems
      \item Trees problems
      \item Hash tables problems
      \item Heaps problems
    \end{topics}
    \begin{learningoutcomes}
        \item Recognize different data structures, their complexities, uses and restrictions.[\Usage]
        \item Identify the type of data structure appropriate to the resolution of the problem. [\Usage]
        \item Recognize types of problems associated with operations on data structures such as searching, inserting, deleting and updating.[\Usage]
    \end{learningoutcomes}
  \end{unit}
  
  \begin{unit}{Algorithmic Design Paradigms}{}{Cormen2009,Steven09,Kulikov09,SkienaRevilla:PC:2003,Laaksonen17,aziz2012elements}{20}{a,b,h}
    \begin{topics}
      \item Brute force
      \item Divide and conquer
      \item Backtracking
      \item Greedy
      \item Dynamic Programming
    \end{topics}
    \begin{learningoutcomes}
        \item Learning the different algorithhmic design paradigms.[\Usage]   
        \item Learning to select the right algorithms for different problems applying different algorithhmic design paradigms.[\Usage]
    \end{learningoutcomes}
  \end{unit}
  
  \begin{unit}{Graphs}{}{Cormen2009,Steven09,Kulikov09,SkienaRevilla:PC:2003,Laaksonen17,aziz2012elements}{20}{a,b,h}
    \begin{topics}
      \item Graphs transversal 
      \item Graphs aplications 
      \item Shortest path
      \item Networks and flows
    \end{topics}
    \begin{learningoutcomes}
        \item Identify problems classified as graph problems. [\Usage]
        \item Learn how to select the right algorithms for network problems (transversal, MST, shortest-path, network and flows). [\Usage]
    \end{learningoutcomes}
  \end{unit}
  
  \begin{unit}{Advanced topics}{}{Cormen2009,Steven09,Kulikov09,SkienaRevilla:PC:2003,Laaksonen17,aziz2012elements}{20}{a,b,h}
    \begin{topics}
      \item Number theory 
      \item Probabilities and combinations 
      \item String algorithms (tries, string hashing, z-algorithm)
      \item Geometric algorithms
    \end{topics}
    \begin{learningoutcomes}
        \item Learning to select the right algorithms for problems in number theory and mathematics as they are important in competitive programming. [\Usage]
        \item Learning to select the right algorithms for problems about probabilities and combinations, strings and computational geometry. [\Usage]
    \end{learningoutcomes}
  \end{unit}
  
  \begin{unit}{Domain specific problems}{}{Cormen2009,Steven09,Kulikov09,SkienaRevilla:PC:2003,Laaksonen17,aziz2012elements}{20}{a,b,h}
    \begin{topics}
      \item Latency and throughput
      \item Parallelism
      \item Networks
      \item Storage
      \item High availability
      \item Caching
      \item Proxies
      \item Load balancers
      \item Key-value stores
      \item Replicating and sharing
      \item Leader election
      \item Rate limiting
      \item Logging and monitoring
    \end{topics}
    \begin{learningoutcomes}
        \item Learning to design systems for different domain-specific problems by applying knowledge about networks, distributed computing, high availability, storage and system architecture.[\Usage]
    \end{learningoutcomes}
  \end{unit}
  
  \begin{coursebibliography}
  \bibfile{Computing/CS/CS311}
  \end{coursebibliography}
  
  \end{syllabus}
  
