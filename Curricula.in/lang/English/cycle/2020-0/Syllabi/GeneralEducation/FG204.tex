\begin{syllabus}

\course{FG204. Teología}{Obligatorio}{FG204}

\begin{justification}
La Universidad Católica San Pablo busca ofrecer una visión de la persona humana y del mundo iluminada por el Evangelio y, consiguientemente, por la fe en Cristo-Logos, como centro de la creación y de la historia. El estudio de la teología es fundamental para dicha comprensión de Dios, del hombre y del cosmos.
La Teología permite al creyente en Cristo conocer y comprender mejor su fe. Al no creyente, la comprensión de la cosmovisión que ha forjado la cultura occidental en la cual ha nacido, vive y desarrollará su propia vida, así como abrirse al conocimiento de Dios desde Jesucristo y su Iglesia.
\end{justification}

\begin{goals}
\item Conocer y comprender el Cristianismo en cuanto religión revelada desde las razones en las que se apoya, mostrando su credibilidad, a fin de ofrecer al creyente razones que motivan su opción de fe y presentar a quien no lo es razones para creer. [\Familiarity]
\end{goals}

\begin{outcomes}
    \item \ShowOutcome{n}{2}
    \item \ShowOutcome{o}{2}
    \item \ShowOutcome{p}{2}
\end{outcomes}

\begin{competences}
    \item \ShowCompetence{C17}{n,ñ}
    \item \ShowCompetence{C20}{n,ñ,o}
    \item \ShowCompetence{C21}{n,ñ}
    \item \ShowCompetence{C22}{n,ñ}
    \item \ShowCompetence{C24}{ñ,ñ}
\end{competences}

\begin{unit}{}{Primera Unidad: Homo Capa Dei, El hombres es capaz de Dios}{Homo,arrayas2006,Ratzinger2007,swinburne2011,de2011}{3}{C17,C24}
\begin{topics}
	\item El hombre: un ser inquieto en búsqueda.
	\item La vía ascendente del hombre a Dios.
	      \begin{subtopics}
		\item La razón y el conocimiento de Dios.
		\item La experiencia existencial.
		\item La búsqueda religiosa.
	      \end{subtopics}
	\item Expresiones del espíritu religioso.
	\item La negación de Dios.
\end{topics}
\begin{learningoutcomes}
	\item Mostrar la ``hipótesis Dios'' como algo connatural al espíritu humano y las consecuencias que de ello se derivan. [\Familiarity]
\end{learningoutcomes}
\end{unit}

\begin{unit}{}{Segunda Unidad: Dios sale al encuentro del hombre}{ConcilioII,Catecismo,latourelle1967}{9}{C22}
\begin{topics}
	\item Dios habla al hombre.
	\item Jesucristo: Plenitud de la Revelación.
	\item Las Sagradas Escrituras.
	\item La Tradición.
	\item La sucesión apostólica.
\end{topics}
\begin{learningoutcomes}
	\item Reconocer la doctrina católica de la Revelación, entendida como el camino de auto comunicación de Dios hacia el hombre y las implicaciones que de dicha doctrina se derivan. [\Familiarity]
\end{learningoutcomes}
\end{unit}

\begin{unit}{}{Tercera Unidad: Credo ut Intelligam. La fe y la razón}{Catecismo,benedicto2006discurso,pablo1998carta,Catecismo,XVI2011,anselmo1970}{6}{C20}
\begin{topics}
	\item La fe natural y el acto de creer como acto razonable.
	\item La Fe sobrenatural (Fides qua creditur).
	\item El contenido de la Fe (Fides quae creditur).
	\item Fe y razón.
\end{topics}
\begin{learningoutcomes}
	\item Estimar el acto de Fe como la respuesta del hombre a Dios, que se revela, y su relación con la razón. [\Familiarity]
\end{learningoutcomes}
\end{unit}

\begin{unit}{}{Cuarta Unidad: Jesús de Nazaret}{guardini2006,BenedictoXVIJesus,pablo1998creo,Catecismo,giacomo2001,adam1972}{15}{C21}
\begin{topics}
	\item ?`Quién es Jesús?
	    \begin{subtopics}
		  \item Historicidad de Jesús de Nazaret.
	    \end{subtopics}
	\item ?`Qué dice Jesús de sí mismo?
	    \begin{subtopics}
		\item Jesús el Mesías.
		\item Jesús el Hijo del Hombre.
		\item Jesús el Hijo de Dios.
	    \end{subtopics}
	\item ?`Qué hizo Jesús?
	    \begin{subtopics}
		\item Testigo de la Verdad: El mensaje de Jesús.
		\item Pasó haciendo el bien: Los milagros de Jesús.
		\item La Resurrección.
	    \end{subtopics}
	\item La Fe de la Iglesia en Cristo.
	    \begin{subtopics}
		\item Verdadero Dios: Logos.
		\item Verdadero Hombre: La Encarnación.
		\item Dios y hombre Verdadero: La unión Hipostática.
		\item El Reconciliador, el Señor.
	    \end{subtopics}

\end{topics}
\begin{learningoutcomes}
	\item Distinguir a Jesús de Nazaret como el Cristo, Plenitud de la revelación de Dios a los hombres. [\Familiarity]
\end{learningoutcomes}
\end{unit}

\begin{unit}{}{Quinta Unidad: La Iglesia de Cristo}{pablo1998creo,Catecismo,de1988,concilio1973,xvi1992}{6}{C20,C22,C24}
\begin{topics}
	\item Objeciones contra la Iglesia.
	\item La Iglesia de Cristo.
	    \begin{subtopics}
		\item Cristo funda la Iglesia.
		\item La Iglesia Cuerpo de Cristo.
		\item La Iglesia prolonga en la historia la presencia de Cristo.
		\item Sacramento Universal de Salvación.
	    \end{subtopics}

	\item Las notas de la Iglesia
		\begin{subtopics}
		\item Una
		\item Santa
		\item Católica
		\item Apostólica
	    \end{subtopics}
\end{topics}
\begin{learningoutcomes}
	\item Conocer y valorar la naturaleza y misión de la Iglesia y su inseparable relación con Jesucristo. [\Familiarity]
\end{learningoutcomes}
\end{unit}



\begin{coursebibliography}
\bibfile{GeneralEducation/FG204}
\end{coursebibliography}

\end{syllabus}
