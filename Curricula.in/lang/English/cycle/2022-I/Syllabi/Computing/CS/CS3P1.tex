\begin{syllabus}

\course{CS3P01. Parallel and Distributed Computing }{Obligatorio}{CS3P01}
% Source file: ../Curricula.in/lang/English/cycle/2020-I/Syllabi/Computing/CS/CS3P1.tex

\begin{justification}
The last decade has brought explosive growth in computing with multiprocessors, including
Multi-core processors and distributed data centers. As a result, computing parallel and distributed has become a widely elective subject to be one of the main components
in the mesh studies in computer science undergraduate. Both parallel and distributed computing the simultaneous execution of multiple processes, whose operations have the potential to
intercalar in a complex way. Parallel and distributed computing builds on foundations in many areas, including understanding the fundamental concepts of systems, such as: concurrency
and parallel execution, consistency in state / memory manipulation, and latency. The communication and coordination between processes has its foundations in the passage of messages and models of
shared memory of computing and algorithmic concepts like atomicity, consensus and conditional waiting.Achieving acceleration in practice requires an understanding of parallel algorithms, strategies for
decomposition problem, systems architecture, implementation strategies and analysis of performance. Distributed systems highlight the problems of security and tolerance to
Failures, emphasize the maintenance of the replicated state and introduce additional problems in the field of computer networks.
\end{justification}

\begin{goals}
\item That the student is able to create parallel applications of medium complexity by efficiently leveraging machines with multiple cores.
\item That the student is able to compare sequential and parallel applications.
\item That the student is able to convert, when the situation warrants, sequential applications to parallel efficiently
\end{goals}


\begin{unit}{\PDParallelismFundamentals}{}{peterpacheco,matloff,quinnz,introPerforComp}{18}{a}
\begin{topics}%
    \item \PDParallelismFundamentalsTopicMultiple
    \item \PDParallelismFundamentalsTopicGoals
    \item \PDParallelismFundamentalsTopicParallelism
    \item \PDParallelismFundamentalsTopicProgramming
\end{topics}    
\begin{learningoutcomes}%
    \item \PDParallelismFundamentalsLODistinguishUsing~[\Familiarity] %
    \item \PDParallelismFundamentalsLODistinguishMultiple~[\Familiarity] %
    \item \PDParallelismFundamentalsLODistinguishData~[\Familiarity] %
\end{learningoutcomes}%
\end{unit}

\begin{unit}{\PDParallelArchitecture}{}{peterpacheco,wenmei,sanders,introPerforComp}{12}{a,b}
\begin{topics}%
    \item \PDParallelArchitectureTopicMulticore
    \item \PDParallelArchitectureTopicShared
    \item \PDParallelArchitectureTopicSymmetric
    \item \PDParallelArchitectureTopicSimd
    \item \PDParallelArchitectureTopicGpu
    \item \PDParallelArchitectureTopicFlynns
    \item \PDParallelArchitectureTopicInstruction
    \item \PDParallelArchitectureTopicMemory
    \item \PDParallelArchitectureTopicTopologies
\end{topics}
\begin{learningoutcomes}%
    \item \PDParallelArchitectureLOExplainTheShared~[\Assessment] %
    \item \PDParallelArchitectureLODescribeTheAndKey~[\Assessment] %
    \item \PDParallelArchitectureLOCharacterizeTheTasks~[\Usage] %
    \item \PDParallelArchitectureLODescribeTheLimitationsVs~[\Usage] %
    \item \PDParallelArchitectureLOExplainTheEach~[\Usage] %
    \item \PDParallelArchitectureLODescribeTheMaintaining~[\Familiarity] %
    \item \PDParallelArchitectureLODescribeTheChallenges~[\Familiarity] %
\end{learningoutcomes}%
\end{unit}

\begin{unit}{\PDParallelDecomposition}{}{peterpacheco,matloff,quinn,introPerforComp}{18}{a,b}
\begin{topics}%
    \item \PDParallelDecompositionTopicNeed
    \item \PDParallelDecompositionTopicIndependence
    \item \PDParallelDecompositionTopicBasic
    \item \PDParallelDecompositionTopicTask
    \item \PDParallelDecompositionTopicData
    \item \PDParallelDecompositionTopicActors
\end{topics}
\begin{learningoutcomes}%
    \item \PDParallelDecompositionLOExplainWhyNecessary~[\Usage] %
    \item \PDParallelDecompositionLOIdentifyOpportunities~[\Familiarity] %
    \item \PDParallelDecompositionLOWriteAScalable~[\Usage] %
    \item \PDParallelDecompositionLOParallelize~[\Usage] %
    \item \PDParallelDecompositionLOParallelizeAn~[\Usage] %
    \item \PDParallelDecompositionLOWriteAActors~[\Usage] %
\end{learningoutcomes}%
\end{unit}

\begin{unit}{\PDCommunicationandCoordination}{}{peterpacheco,matloff,quinn,introPerforComp}{18}{a,b}
\begin{topics}%
    \item \PDCommunicationandCoordinationTopicShared
    \item \PDCommunicationandCoordinationTopicConsistency
    \item \PDCommunicationandCoordinationTopicMessage
    \item \PDCommunicationandCoordinationTopicAtomicity
    \item \PDCommunicationandCoordinationTopicConsensus
    \item \PDCommunicationandCoordinationTopicConditional
\end{topics}
\begin{learningoutcomes}%
    \item \PDCommunicationandCoordinationLOUseMutual~[\Usage] %
    \item \PDCommunicationandCoordinationLOGiveAn~[\Familiarity] %
    \item \PDCommunicationandCoordinationLOGiveAnA~[\Usage] %
    \item \PDCommunicationandCoordinationLOExplainWhenMulticast~[\Familiarity] %
    \item \PDCommunicationandCoordinationLOWriteACorrectly~[\Usage] %
    \item \PDCommunicationandCoordinationLOGiveAnAWhich~[\Familiarity] %
    \item \PDCommunicationandCoordinationLOUseSemaphores~[\Usage] %
\end{learningoutcomes}%
\end{unit}

\begin{unit}{\PDParallelAlgorithmsAnalysisandProgramming}{}{matloff,quinn,introPerforComp}{18}{a,b}
\begin{topics}%
    \item \PDParallelAlgorithmsAnalysisandProgrammingTopicCritical
    \item \PDParallelAlgorithmsAnalysisandProgrammingTopicSpeed
    \item \PDParallelAlgorithmsAnalysisandProgrammingTopicNaturally
    \item \PDParallelAlgorithmsAnalysisandProgrammingTopicParallel
    \item \PDParallelAlgorithmsAnalysisandProgrammingTopicParallelGraph
    \item \PDParallelAlgorithmsAnalysisandProgrammingTopicParallelMatrix
    \item \PDParallelAlgorithmsAnalysisandProgrammingTopicProducer
    \item \PDParallelAlgorithmsAnalysisandProgrammingTopicExamples
\end{topics}
\begin{learningoutcomes}%
        \item \PDParallelAlgorithmsAnalysisandProgrammingLODefineCritical~[\Familiarity] %
        \item \PDParallelAlgorithmsAnalysisandProgrammingLOComputeTheSpan~[\Usage] %
        \item \PDParallelAlgorithmsAnalysisandProgrammingLODefineSpeed~[\Familiarity] %
        \item \PDParallelAlgorithmsAnalysisandProgrammingLOIdentifyIndependent~[\Usage] %
        \item \PDParallelAlgorithmsAnalysisandProgrammingLOCharacterizeFeatures~[\Familiarity] %
        \item \PDParallelAlgorithmsAnalysisandProgrammingLOImplementAAnd~[\Usage] %
        \item \PDParallelAlgorithmsAnalysisandProgrammingLODecompose~[\Usage] %
        \item \PDParallelAlgorithmsAnalysisandProgrammingLOProvideAn~[\Usage] %
        \item \PDParallelAlgorithmsAnalysisandProgrammingLOGiveExamplesWhere~[\Usage] %
        \item \PDParallelAlgorithmsAnalysisandProgrammingLOImplementAAlgorithm~[\Usage] %
        \item \PDParallelAlgorithmsAnalysisandProgrammingLOIdentifyIssuesIn~[\Usage] %
\end{learningoutcomes}%
\end{unit}

\begin{unit}{\PDParallelPerformance}{}{peterpacheco,matloff,wenmei,sanders,introPerforComp}{18}{a,b,c}
\begin{topics}%
    \item \PDParallelPerformanceTopicLoad
    \item \PDParallelPerformanceTopicPerformance
    \item \PDParallelPerformanceTopicScheduling
    \item \PDParallelPerformanceTopicEvaluating
    \item \PDParallelPerformanceTopicData
    \item \PDParallelPerformanceTopicPower
\end{topics}
\begin{learningoutcomes}%
    \item \PDParallelPerformanceLODetect~[\Usage] %
    \item \PDParallelPerformanceLOCalculateThe~[\Usage] %
    \item \PDParallelPerformanceLODescribeHowLayout~[\Familiarity] %
    \item \PDParallelPerformanceLODetectAnd~[\Usage] %
    \item \PDParallelPerformanceLOExplainTheScheduling~[\Familiarity] %
    \item \PDParallelPerformanceLOExplainPerformance~[\Familiarity] %
    \item \PDParallelPerformanceLOExplainTheTrade~[\Familiarity] %
\end{learningoutcomes}%
\end{unit}

\begin{coursebibliography}
\bibfile{Computing/CS/CS3P1}
\end{coursebibliography}

\end{syllabus}
