\begin{syllabus}

\course{CS312. Estructuras de Datos Avanzadas}{Obligatorio}{CS312} % Common.pm

\begin{justification}
Algorithms and data structures are a fundamental part of computer science that allow us to organize information in a more efficient way, so it is important for every professional in the area to have a solid background in this regard.

In the course of advanced data structures our goal is for the student to know and analyze complex structures, such as Multidimensional Access Methods, Space-Time Access Methods and Metric Access Methods, etc.
\end{justification}

\begin{goals}
\item That the student understands, designs, implements, applies and
Propose innovative data structures to solve problems related to the handling of multidimensional data, retrieval of information by similarity, search engines and
other computational problems.
\end{goals}

\begin{outcomes}{V1}
   \item \ShowOutcome{a}{1}
   \item \ShowOutcome{b}{2}
   \item \ShowOutcome{c}{1}
   \item \ShowOutcome{j}{1}
\end{outcomes}

\begin{competences}{V1}
    \item \ShowCompetence{C1}{a,j} 
    \item \ShowCompetence{C20}{b}
    \item \ShowCompetence{CS2}{c}
\end{competences}

\begin{unit}{Técnicas Básicas de Implementación de Estructuras de Datos}{}{Cuadros2004Implementing,Knuth2007TAOCP-V-I,Knuth2007TAOCP-V-II,Gamma94}{16}{C1}
   \begin{topics}
         \item Structured Programming
         \item Object-oriented programming
         \item Abstract Data Types
         \item Independence of the user programming language of the structure
         \item Platform Independence
         \item Concurrency control
         \item Data Protection
         \item Encapsulation levels (struct, class, namespace, etc)
   \end{topics}
   \begin{learningoutcomes}
         \item That the student understands the basic differences that involve the different techniques of implementation of data structures[\Usage]
         \item That the student analyze the advantages and disadvantages of each of the existing techniques[\Usage]
   \end{learningoutcomes}
\end{unit}

\begin{unit}{Métodos de Acceso Multidimensionales}{}{Samet2004SAM-MAM,Gaede98multidimensional}{16}{C20}
   \begin{topics}
         \item Access Methods for Point Data
         \item Access Methods for non-point data
         \item Problems with dimension enhancement
   \end{topics}
   \begin{learningoutcomes}
         \item That the student understands to know and implement some Access Methods for multidimensional data and temporal space[\Usage]
         \item That the student understands the potential of these Access Methods in the future of commercial databases[\Usage]
   \end{learningoutcomes}
\end{unit}

\begin{unit}{Métodos de Acceso Métrico}{}{Samet2004SAM-MAM,Chavez:01,Traina00SlimTree,Zezula07}{20}{C24}
   \begin{topics}
         \item Metric Access Methods for discrete distances
         \item Metric Access Methods for Continuous Distances
   \end{topics}
   \begin{learningoutcomes}
         \item That the student understands to know and implement some methods of metric access[\Usage]
         \item That the student understands the importance of these Access Methods for Information Retrieval by similarity[\Usage]
   \end{learningoutcomes}
\end{unit}

\begin{unit}{Métodos de Acceso Aproximados}{}{Indyk06,Zezula07,Samet2004SAM-MAM}{20}{C1}
   \begin{topics}
         \item Space Filling Curves
         \item Locality Sensitive Hashing
   \end{topics}
   \begin{learningoutcomes}
         \item That the student understands to know and implement some approximate access methods[\Usage]
         \item That the student understands the importance of these Access Methods for Information Retrieval by Similarity in environments where Scalability is a very important factor [\Usage]
   \end{learningoutcomes}
\end{unit}

\begin{unit}{Seminarios}{}{Samet2004SAM-MAM,Chavez:01}{8}{C20}
	\begin{topics}
         \item Access Methods Temporary Space
         \item Generic Data Structures
   \end{topics}
   \begin{learningoutcomes}
         \item That the student can discuss the latest advances in access methods for different domains of knowledge [\Usage]
   \end{learningoutcomes}
\end{unit}

\begin{coursebibliography}
\bibfile{Computing/CS/CS312}
\end{coursebibliography}

\end{syllabus}
