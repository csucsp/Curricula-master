\begin{syllabus}

\course{GH0005. Communication Laboratory}{Obligatorio}{CB101}

\begin{justification}
Through this course, the student will improve and strengthen his abilities to communicate both oral and written in an academic context. To do this, the student will exercise in the composition of texts, taking into account the requirements of a formal academic language: characteristics of academic writing (rules of punctuation, spelling, grammatical lexical competence, normative) and correct use of information. In turn, the course promotes a comprehensive reading that is not limited to the descriptive level, but also encompasses the conceptual and metaphorical, because only in this way will the student develop his critical and analytical capacity. The student will take on academic readings and scientific outreach that will allow him to distinguish between the objectives set out in the different types of texts and to recognize the oral and written text as a coherent and cohesive unit in terms of form and content. Once these objectives have been achieved, the student will understand that oral and written 
communication skills are central competences of his / her university life and, later, his / her professional life. 
\end{justification}

\begin{goals}
\item With this course the student develops and strengthens his oral and written communication skills in the context of an academic context. In addition, it comprehends conceptually and metaphorically expository texts, and identifies the objectives, hierarchy of ideas and structure of those texts. At the end of the course, the student is able to produce descriptive and informative expository texts. Likewise, it develops its capacity of openness and tolerance towards the diversity of points of view thanks to the continuous group work, self-evaluations and peer evaluations that will be faced during the course cycle.
\end{goals}

\begin{outcomes}   
    \item \ShowOutcome{e}{2}
    \item \ShowOutcome{f}{2}
    \item \ShowOutcome{i}{2}
    \item \ShowOutcome{n}{2}
\end{outcomes}

\begin{competences}
    \item \ShowCompetence{C17}{f,h,n}
    \item \ShowCompetence{C20}{f,n}
    \item \ShowCompetence{C24}{f,h}
\end{competences}


\begin{unit}{Communication Laboratory I}{}{Cassany93}{12}{4}
   \begin{topics}
      \item Approach to some characteristics of formal writing.
      \item Academic writing features
      \item Reading strategies.
      \item Structure of text.
      \item Structure of paragraphs.
      \item Characteristics of the paragraph.
      \item Argumentative vs.expository text.
      \item Writing process: delimitation of topic and outline production.
      \item Citations:function and types
      \item Approach to characteristic of the oral presentacion.
      \item Conference:formal presentation
      \item Writing full tet with citations.
   \end{topics}
   \begin{learningoutcomes}
      \item Selt-assessment: the students are able to recognize their own strenghts and weaknesses to make constructive criticism of their own work .
   \end{learningoutcomes}
\end{unit}




\begin{coursebibliography}
\bibfile{GeneralEducation/GH1005}
\end{coursebibliography}

\end{syllabus}
