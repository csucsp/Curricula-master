\begin{syllabus}

\course{CS1D3. Álgebra Abstracta}{Obligatorio}{CS1D3}

\begin{justification}
En algebra abstracta se explotará las nociones de teoria de números, grupos, anillos y campos para comprender en profundidad temas de computación como criptografía y teoría de la codificación.
\end{justification}

\begin{goals}
	\item Entender los conceptos de estructuras algebraicas como anillos, dominios, cuerpos y grupos.
	\item Utilizar las propiedades de las estructuras algebraicas para resolver problemas  
	\item  Conocer las técnicas y métodos de sistemas criptográficos y como los teoremas permiten la realización de cálculos rápidos y eficientes.
\end{goals}
\begin{outcomes}{V1}
    \item \ShowOutcome{a}{3}
    \item \ShowOutcome{i}{2}
    \item \ShowOutcome{j}{3}
 \end{outcomes}

\begin{competences}
    \item \ShowCompetence{C1}{a}
    \item \ShowCompetence{C8}{a}
    \item \ShowCompetence{C16}{j}
    \item \ShowCompetence{CS2}{i}
\end{competences}

\begin{unit}{}{Teoría de Números}{Rosen2011,Grimaldi03, Koshy07}{16}{C1,CS2}
   \begin{topics}
    	\item Número enteros, algoritmos de la división, máximo común divisor, algoritmo de Euclides y algoritmo extendido de Euclides. Ecuaciones diofánticas
    	\item Aritmética Modular y Operaciones en Zn: suma, resta, multiplicación, inversa y exponenciación.
	\item Congruencia, conjunto de residuos, congruencia lineal, teorema chino del resto.
	\item Generadores de números primos y pseudo-aleatorios, función phi de Euler, teorema pequeño de Fermat, teorema de Euler, teorema fundamental de la aritmética y factorización.			
   \end{topics}
   \begin{learningoutcomes}
      	\item  Realizar cálculos que involucren aritmética modular [\Usage]
	\item  Describir algoritmos numérico teóricos básicos eficientes, incluyendo el algoritmo de Euclides y el algoritmo extendido de Euclides. [\Assessment]
	\item Establecer la importancia del estudio de la teoría de números. [\Familiarity]
	\item \IASCryptographyLODiscussThePrime [\Familiarity]
   \end{learningoutcomes}
\end{unit}

\begin{unit}{}{Estructuras Algebraicas}{Grimaldi03,Gallian2012, Koshy07}{14}{C1, C16}
   \begin{topics}
    	\item Grupos: propiedades, operaciones, homomorfismos e isomorfismo, orden de un grupo, grupos cíclicos, teorema de Lagrange y raíces primitivas.
    	\item Anillos y cuerpos: propiedades, sub-anillos, dominios de integridad.
   \end{topics}
   \begin{learningoutcomes}
	\item  Adquirir habilidad en la resolución de problemas abstractos y en la formulación de conjeturas . [\Familiarity]
	\item Argumentar como los principales teoremas y algoritmos permiten resolver problemas criptográficos. [\Assessment]
   \end{learningoutcomes}
\end{unit}

\begin{unit}{\IASCryptography}{}{Menezes96, Paar2011, Forouzan08}{20}{C8, C16}
\begin{topics}
    \item \IASCryptographyTopicBasic
    \item \IASCryptographyTopicCipher
    \item \IASCryptographyTopicPublic
    \item \IASCryptographyTopicMathematical
    \item \IASCryptographyTopicCryptographic
    \item \IASCryptographyTopicSymmetric
    \item \IASCryptographyTopicPublicKey
    \item \IASCryptographyTopicAuthenticated
    \item \IASCryptographyTopicCryptographicProtocols
    \item \IASCryptographyTopicMotivate
    \item \IASCryptographyTopicSecurity
    \item \IASCryptographyTopicCryptographicStandards
    \item \IASCryptographyTopicQuantum
\end{topics}

\begin{learningoutcomes}
    \item \IASCryptographyLODescribeTheCryptography [\Familiarity]
    \item \IASCryptographyLODefineTheCipher [\Familiarity]
    \item \IASCryptographyLODiscussThePrime [\Familiarity]
    \item \IASCryptographyLOExplainHowInfrastructure [\Familiarity]
    \item \IASCryptographyLOUseCryptographic [\Familiarity]
    \item \IASCryptographyLOIllustrateHow [\Familiarity]
    \item \IASCryptographyLOUsePublic [\Familiarity]
    \item \IASCryptographyLOExplainHowProtocols [\Familiarity]
    \item \IASCryptographyLODiscussCryptographic [\Familiarity]
    \item \IASCryptographyLODescribeReal [\Familiarity]
    \item \IASCryptographyLOSummarizePrecise [\Familiarity]
    \item \IASCryptographyLOApplyAppropriate [\Familiarity]
    \item \IASCryptographyLOAppreciate [\Familiarity]
    \item \IASCryptographyLODescribeQuantum [\Familiarity]
\end{learningoutcomes}
\end{unit}

\begin{unit}{}{Teoria Algebraica de la Codificación}{Grimaldi03, Trappe05}{10}{CS2}
   \begin{topics}
    	\item  Elementos, proceso de transmitir una palabra
    	\item Esquemas de codificación: paridad, triple repetición, verificación de paridad y generación de códigos de grupo.
	 \end{topics}
   \begin{learningoutcomes}
      	\item  Utilizar las propiedades de las estructuras algebraicas en el estudio  de la teoria algebraica de los códigos. [\Familiarity]
	 \item Aplicar técnicas que permitan la detección de errores, y si es necesario, proveer de métodos para reconstruir palabras originales. [\Usage]
   \end{learningoutcomes}
\end{unit}




\begin{coursebibliography}
\bibfile{Computing/CS/CS1D3}
\end{coursebibliography}

\end{syllabus}


