% Responsable : Luis Díaz Basurco
% Sumilla de  : Estadísticas y Probabilidades
% Versión     : 1

\begin{syllabus}

\course{MA308. Análisis Exploratorio de Datos Espaciales}{Obligatorio}{MA308} % Common.pm

\begin{justification}
Provee de una introducción a la teoría de las probabilidades e inferencia estadistica  con aplicaciones, necesarias en el análisis de datos, diseño de modelos aleatorios y toma de decisiones.
\end{justification}

\begin{goals}
\item Que el alumno aprenda a utilizar las herramientas de la estadistica  para tomar decisiones ante situaciones de incertidumbre.
\item Que el alumno aprenda a obtener conclusiones a partir de datos experimentales.
\item Que el alumno pueda extraer conslusiones útiles sobre la totalidad de una población basándose en información. recolectada
\end{goals}

\begin{outcomes}{V1}
   \item \ShowOutcome{a}{2}
   \item \ShowOutcome{i}{2}
   \item \ShowOutcome{j}{3}
\end{outcomes}

\begin{competences}{V1}
    \item \ShowCompetence{C1}{a} 
    \item \ShowCompetence{CS6}{i}
    \item \ShowCompetence{CS2}{j}
\end{competences}


\begin{unit}{}{Estadísticas descriptiva}{Mendenhall97}{6}{C1}
\begin{topics}
      \item Presentación de datos
      \item Medidas de localización central
      \item Medidas de dispersión
   \end{topics}

   \begin{learningoutcomes}
      \item Presentar resumir y describir datos. [\Usage]
   \end{learningoutcomes}
\end{unit}

\begin{unit}{}{Probabilidades}{Meyer70}{6}{C1}
\begin{topics}
      \item Espacios muestrales y eventos
      \item Axiomas y propiedades de probabilidad
      \item Probabilidad condicional
      \item Independencia,
      \item Teorema de Bayes
   \end{topics}
   \begin{learningoutcomes}
      \item Identificar espacios aleatorios [\Usage]
      \item diseñar  modelos probabilísticos [\Usage]
      \item Identificar eventos como resultado de un  [\Usage]experimento aleatorio [\Usage]
      \item Calcular la probabilidad de ocurrencia de un evento [\Usage]
      \item Hallar la probabilidad usando condicionalidad, independencia y Bayes [\Usage]
   \end{learningoutcomes}
\end{unit}

\begin{unit}{}{Variable aleatoria}{Meyer70,Devore98}{6}{CS6}
\begin{topics}
      \item Definición y tipos de variables aleatorias 
      \item Distribución de probabilidades
      \item Funciones densidad
      \item Valor esperado
      \item Momentos
   \end{topics}

   \begin{learningoutcomes}
      \item Identificar variables aleatorias que describan un espacio muestra [\Usage]
      \item Construir la distribución o función de densidad. [\Usage]
      \item Caracterizar distribuciones o funciones densidad conjunta. [\Usage]
   \end{learningoutcomes}
\end{unit}

\begin{unit}{}{Distribución de probabilidad discreta y continua}{Meyer70,Devore98}{6}{CS6}
\begin{topics}
      \item Distribuciones de probabilidad básicas
      \item Densidades de probabilidad básicas
      \item Funciones de variable aleatoria
   \end{topics}

   \begin{learningoutcomes}
      \item Calcular probabilidad de una variable aleatoria con distribución o función densidad [\Usage]
      \item Identificar la distribución o función densidad que describe un problema aleatorio [\Usage]
      \item Probar propiedades de distribuciones o funciones de densidad [\Usage]
   \end{learningoutcomes}
\end{unit}

\begin{unit}{}{Distribución de probabilidad conjunta}{Meyer70,Devore98}{6}{CS2}
\begin{topics}
      \item Variables aleatorias distribuidas conjuntamente 
      \item Valores esperados, covarianza y correlación
      \item Las estadistica s y sus distribuciones
      \item Distribución de medias de muestras
      \item Distribución de una combinación lineal

   \end{topics}
   \begin{learningoutcomes}
      \item Encontrar la distribución conjunta de dos variables aleatorias discretas o continuas [\Usage]
      \item Hallar las distribuciones marginales o condicionales de variables aleatorias conjuntas [\Usage]
      \item Determinar dependencia o independencia de variables aleatorias [\Usage]
      \item Probar propiedades que son consecuencia del teorema  del límite central [\Usage]
   \end{learningoutcomes}
\end{unit}

\begin{unit}{}{Inferencia estadistica }{Meyer70,Devore98}{6}{CS2}
\begin{topics}
      \item Estimación estadistica 
      \item Prueba de hipótesis
      \item Prueba de hipótesis usando ANOVA
   \end{topics}

   \begin{learningoutcomes}
      \item Probar si un estimador es insesgado, consistente o suficiente [\Usage]
      \item Hallar intervalo intervalos de confianza para estimar parámetros [\Usage]
      \item Tomar decisiones de parámetros en base a pruebas de hipótesis [\Usage]
      \item Probar hipótesis usando ANOVA [\Usage]
   \end{learningoutcomes}
\end{unit}





\begin{coursebibliography}
\bibfile{BasicSciences/MA308}
\end{coursebibliography}

\end{syllabus}
