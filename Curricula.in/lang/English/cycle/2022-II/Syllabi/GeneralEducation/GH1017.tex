\begin{syllabus}

\course{GH0017. Introduction to Quechua}{Electivo}{GH0017}
% Source file: ../Curricula.in/lang/English/cycle/2020-I/Syllabi/GeneralEducation/GH1017.tex

\begin{justification}
The Quechua communicative course allows students to approach the practical use of the Andean language in their Chanca variety. This is one of the varieties of greater diffusion and model to approach other varieties of denominated Quechua southern or Quechua II. In addition, it is simple in its learning to share sounds with Castilian. Also, the course seeks to familiarize the student with the basic structures of this language, as well as with the translation and production of texts. The ultimate goal is to provide the basic learning tools so that the student can express himself at a basic and functional level, as well as lead and develop his own language learning.
We believe that speaking Quechua in certain situations where UTEC engineers have to develop is a very important advantage: native Quechua speakers practice a differentiated treatment with people who speak it because they feel that their tradition is being respected and, at the same time, Making an effort to engage in dialogue in their own language. This represents very specific operational advantages in the treatment and the agreement of interests. 
\end{justification}

\begin{goals}
\item Grant basic tools to introduce and converse in the Quechua language, in the Chanca variety.
\item Approach the student to the basic structures of the language in order to direct his study and self-learning.
\item Train the student in the translation and production of texts in the native language.
\item Provide tools for the student to develop the knowledge of this language individually.
\item Give tools to recognize the origin of Quechua that they face through elements of linguistic analysis.
\end{goals}

\begin{outcomes}{V1}
    \item \ShowOutcome{d}{2} % Multidiscip teams
    \item \ShowOutcome{e}{2} % ethical, legal, security and social implications
    \item \ShowOutcome{f}{2} % communicate effectively
    \item \ShowOutcome{n}{2} % Apply knowledge of the humanities
    \item \ShowOutcome{p}{2} % TASDSH
\end{outcomes}

\begin{competences}{V1}
    \item \ShowCompetence{C10}{d,n,o}
    \item \ShowCompetence{C17}{f}
    \item \ShowCompetence{C18}{f}
    \item \ShowCompetence{C21}{e}
\end{competences}

\begin{unit}{Introduction to Quechua.}{}{Cerron76,Press82}{12}{4}
   \begin{topics}
      \item General Dialectology of Quechua
      \item Phonologic system of Chanca Quechua: phonemes, syllables, accent, pronunciatio
      \item Introduction, basic questions, basic orders
      \item Noun Phrases: pronouns, personal possessive, plural, grammatical cases, and interrogative pronouns
      \item Verbal Phrases: verb tenses, persons
      \item Deverbative and denominative derivation
      \item Sentence themes: syntax
      \item Discourse particles: validators, reporting, etc.
   \end{topics}

   \begin{learningoutcomes}
      \item Use of basic communicative resources in the Quechuan language.
   \end{learningoutcomes}
\end{unit}

\begin{coursebibliography}
\bibfile{GeneralEducation/GH1017}
\end{coursebibliography}

\end{syllabus}
