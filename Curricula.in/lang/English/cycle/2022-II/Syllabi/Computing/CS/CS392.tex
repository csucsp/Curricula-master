\begin{syllabus}

\course{CS392. Tópicos en Ingenieríade Software}{Electivo}{CS392} % Common.pm

\begin{justification}
El desarrollo de software requiere del uso de mejores prácticas de desarrollo, gestión de proyectos de TI, manejo de equipos
y uso eficiente y racional de frameworks de aseguramiento de la calidad y de Gobierno de Portfolios, estos elemento son pieza 
clave y transversal para el éxito del proceso productivo.

Este curso explora el diseño, selección, implementación y gestión de soluciones TI en las Organizaciones. El foco está en 
las aplicaciones y la infraestructura y su aplicación en el negocio.
\end{justification}

\begin{goals}
\item Entender una variedad de frameworks para el análisis de arquitectura empresarial y la toma de decisiones
\item Utilizar técnicas para la evaluación y gestión del riesgo en el portfolio de la empresa
\item Evaluar y planificar la integración de tecnologías emergentes
\item Entender el papel y el potencial de las TI para a apoyar la gestión de procesos empresariales
\item Entender los difentes enfoques para modelar y mejorar los procesos de negocio
\item Describir y comprender modelos de aseguramiento de la calidad como marco clave para el éxitos de los proyectos de TI.
\item Comprender y aplicar el framework de IT Governance como elemento clave para la gestión del portfolio de aplicaciones Empresariales
\end{goals}



\begin{unit}{\SESoftwareDesign}{}{Sommerville2010,Pressman2014}{18}{c,d,i,j,m,o}
\begin{topics}
      \item \SESoftwareDesignTopicSystem
      \item \SESoftwareDesignTopicDesign
      \item \SESoftwareDesignTopicStructural
      \item \SESoftwareDesignTopicDesignPatterns
      \item \SESoftwareDesignTopicRelationships
      \item \SESoftwareDesignTopicSoftware
      \item \SESoftwareDesignTopicThe
      \item \SESoftwareDesignTopicInternal
      \item \SESoftwareDesignTopicInternalDesign
      \item \SESoftwareDesignTopicMeasurement
      \item \SESoftwareDesignTopicTradeoffs
      \item \SESoftwareDesignTopicApplication
      \item \SESoftwareDesignTopicMiddleware
      \item \SESoftwareDesignTopicPrinciples       
\end{topics}
\begin{learningoutcomes}
    \item \SESoftwareDesignLOArticulateDesign [\Usage]
    \item \SESoftwareDesignLOUseAToSimple  [\Usage]
    \item \SESoftwareDesignLOConstructModels [\Usage]
    \item \SESoftwareDesignLOWithin [\Usage]
    \item \SESoftwareDesignLOForASuitable [\Usage]
    \item \SESoftwareDesignLOCreateAppropriate [\Usage]
    \item \SESoftwareDesignLOExplainTheTheA [\Usage]
    \item \SESoftwareDesignLOForThe [\Usage]
    \item \SESoftwareDesignLOGiven [\Usage]
    \item \SESoftwareDesignLOInvestigateThe [\Usage]
    \item \SESoftwareDesignLOApplySimple [\Usage]
    \item \SESoftwareDesignLODescribeARefactoring [\Usage]
    \item \SESoftwareDesignLOSelectSuitable [\Usage]
    \item \SESoftwareDesignLOExplainHowMight [\Usage]
    \item \SESoftwareDesignLODesignAA [\Usage]
    \item \SESoftwareDesignLODiscussAnd [\Usage]
    \item \SESoftwareDesignLOApplyModels [\Usage]
    \item \SESoftwareDesignLOAnalyzeAFrom [\Usage]
    \item \SESoftwareDesignLOAnalyzeAFromOf [\Usage]
    \item \SESoftwareDesignLOExplainTheObjects [\Usage]
    \item \SESoftwareDesignLOApplyComponent [\Usage]
    \item \SESoftwareDesignLORefactorAn [\Usage]
    \item \SESoftwareDesignLOStateAnd [\Usage]
\end{learningoutcomes}
\end{unit}


\begin{unit}{\SESoftwareProjectManagement}{}{Sommerville2010,Pressman2014}{14}{c,d,i,j,m,o}
\begin{topics}
    \item \SESoftwareProjectManagementTopicTeam
    \item \SESoftwareProjectManagementTopicEffort
    \item \SESoftwareProjectManagementTopicRisk
    \item \SESoftwareProjectManagementTopicTeamManagement
    \item \SESoftwareProjectManagementTopicProject
    \item \SESoftwareProjectManagementTopicSoftwareMeasurement
    \item \SESoftwareProjectManagementTopicSoftwareQuality
    \item \SESoftwareProjectManagementTopicRisk
    \item \SESoftwareProjectManagementTopicSystem
\end{topics}
\begin{learningoutcomes}
    \item \SESoftwareProjectManagementLODiscussCommon [\Usage]
    \item \SESoftwareProjectManagementLOCreateAndAgenda [\Usage]
    \item \SESoftwareProjectManagementLOIdentifyAndRoles [\Usage]
    \item \SESoftwareProjectManagementLOUnderstandTheAnd [\Usage]
    \item \SESoftwareProjectManagementLOApplyAStrategy [\Usage]
    \item \SESoftwareProjectManagementLOUseAn [\Usage]
    \item \SESoftwareProjectManagementLOListSeveral [\Usage]
    \item \SESoftwareProjectManagementLODescribeTheRiskSoftware [\Usage]
    \item \SESoftwareProjectManagementLODescribeDifferent [\Usage]
    \item \SESoftwareProjectManagementLODemonstrateThrough [\Usage]
    \item \SESoftwareProjectManagementLODescribeHowOfAffects [\Usage]
    \item \SESoftwareProjectManagementLOCreateAIdentifying [\Usage]
    \item \SESoftwareProjectManagementLOAssessAnd [\Usage]
    \item \SESoftwareProjectManagementLOUsing [\Usage]
    \item \SESoftwareProjectManagementLOTrack [\Usage]
    \item \SESoftwareProjectManagementLOCompareSimple [\Usage]
    \item \SESoftwareProjectManagementLOUseATool [\Usage]
    \item \SESoftwareProjectManagementLODescribeTheRiskThe [\Usage]
    \item \SESoftwareProjectManagementLOIdentifyRisks [\Usage]
    \item \SESoftwareProjectManagementLOExplainHowDecisions [\Usage]
    \item \SESoftwareProjectManagementLOIdentifySecurity [\Usage]
    \item \SESoftwareProjectManagementLODemonstrateA [\Usage]
    \item \SESoftwareProjectManagementLOApplyTheOf [\Usage]
    \item \SESoftwareProjectManagementLOConductAAnalysis [\Usage]
    \item \SESoftwareProjectManagementLOIdentifyAndOfFor [\Usage]
\end{learningoutcomes}
\end{unit}

\begin{unit}{}{ITIL}{Sommerville2010,Pressman2014}{14}{c,d,i,j,m}
\begin{topics}
    \item Administración del servicio como práctica.
    \item Ciclo de vida del servicio.
    \item Definiciones y conceptos genéricos.
    \item Modelos y principios claves.
    \item Procesos.
    \item Tecnología y arquitectura.
    \item Competencia y entrenamiento.
\end{topics}
\begin{learningoutcomes}
  \item Utilizar y aplicar correctamente ITIL en el proceso de software. [\Usage]
\end{learningoutcomes}
\end{unit}

\begin{unit}{}{COBIT}{Sommerville2010,Pressman2014}{14}{c,d,i,j,m}
\begin{topics}
    \item Fundamentos e Introducción.
    \item Frameworks de Control y IT Governance.
\end{topics}
\begin{learningoutcomes}
\item Utilizar y aplicar correctamente COBIT en el proceso de software. [\Usage]	
\end{learningoutcomes}
\end{unit}




\begin{coursebibliography}
\bibfile{Computing/CS/CS392}
\end{coursebibliography}

\end{syllabus}
