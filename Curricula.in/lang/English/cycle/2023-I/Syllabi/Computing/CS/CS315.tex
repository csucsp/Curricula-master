\begin{syllabus}

\course{CS315. Estructuras de Datos Avanzadas}{Obligatorio}{CS315}

\begin{justification}
Algorithms and data structures are a fundamental part of computer science that allow us to organize information more efficiently, so it is important for every professional in the area to have a solid background in this regard.

In the course of advanced data structures our goal is for the student to know and analyze
complex structures, such as Multidimensional Access Methods, Spatio-Temporal Access Methods and
Metric Access Methods, Compact Data Structures, etc.
\end{justification}

\begin{goals}
\item That the student understands, designs, implements, applies and
Propose innovative data structures to solve problems related to the handling of multidimensional data, retrieval of information by similarity, search engines and
other computational problems.
\end{goals}

\begin{outcomes}
\ExpandOutcome{a}{3}
\ExpandOutcome{b}{4}
\ExpandOutcome{c}{3}
\ExpandOutcome{g}{4}
\ExpandOutcome{j}{3}
\ExpandOutcome{k}{3}
\end{outcomes}

\begin{unit}{Técnicas Básicas de Implementación de Estructuras de Datos}{}{Cuadros2004Implementing,Knuth2007TAOCP-V-I,Knuth2007TAOCP-V-II,Gamma94}{16}{a,b,c}
   \begin{topics}
         \item Programación estructurada
         \item Programación Orientada a Objetos
         \item Tipos Abstractos de Datos
         \item Independencia del lenguaje de programación del usuario de la estructura
         \item Independencia de Plataforma
         \item Control de concurrencia
         \item Protección de Datos
         \item Niveles de encapsulamiento (struct, class, namespace, etc)
   \end{topics}

   \begin{learningoutcomes}
         \item Que el alumno entienda las diferencias básicas que involucran las distintas técnicas de implementación de estructuras de datos
         \item Que el alumno analice las ventajas y desventajas de cada una de las técnicas existentes
   \end{learningoutcomes}
\end{unit}

\begin{unit}{Métodos de Acceso Multidimensionales}{}{Samet2004SAM-MAM,Gaede98multidimensional}{16}{a,b,c}
   \begin{topics}
         \item Point Access Methods
         \item Métodos de Acceso para datos no puntuales
         \item Problemas relacionados con el aumento de dimensión
   \end{topics}

   \begin{learningoutcomes}
         \item Que el alumno entienda conozca e implemente algunos Métodos de Acceso para datos multidimensionales y espacio temporales
         \item Que el alumno entienda el potencial de estos Métodos de Acceso en el futuro de las bases de datos comerciales
   \end{learningoutcomes}
\end{unit}

\begin{unit}{Métodos de Acceso Métrico}{}{Samet2004SAM-MAM,Chavez:01,Traina00SlimTree,Zezula07}{20}{a,b,c,i}
   \begin{topics}
         \item Métodos de Acceso Métrico para distancias discretas
         \item Métodos de Acceso Métrico para distancias continuas
   \end{topics}

   \begin{learningoutcomes}
         \item Que el alumno entienda conozca e implemente algunos métodos de acceso métrico
         \item Que el alumno entienda la importancia de estos Métodos de Acceso para la Recuperación de Información por Similitud
   \end{learningoutcomes}
\end{unit}

\begin{unit}{Métodos de Acceso Aproximados}{}{Samet2004SAM-MAM,Indyk06,Zezula07}{20}{a,b,c,i}
   \begin{topics}
         \item Space Filling Curves
         \item Locality Sensitive Hashing
   \end{topics}

   \begin{learningoutcomes}
         \item Que el alumno entienda conozca e implemente algunos métodos de acceso aproximados
         \item Que el alumno entienda la importancia de estos Métodos de Acceso para la Recuperación de Información por Similitud en entornos donde la Escalabilidad sea una factor muy importante
   \end{learningoutcomes}
\end{unit}

\begin{unit}{Seminarios}{}{Samet2004SAM-MAM,Navarro:CompactDS:2016,Chavez:01}{8}{a,b,c,i}
   \begin{topics}
         \item Métodos de Acceso Espacio Temporal
         \item Estructuras de Datos con programación genérica
   \end{topics}

   \begin{learningoutcomes}
         \item Que el alumno pueda discutir sobre los últimos avances en métodos de acceso para distintos dominios de conocimiento
   \end{learningoutcomes}
\end{unit}

\begin{coursebibliography}
\bibfile{Computing/CS/CS315}
\end{coursebibliography}

\end{syllabus}
