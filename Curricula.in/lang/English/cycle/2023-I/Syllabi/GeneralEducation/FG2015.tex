\begin{syllabus}

\course{FG2015. Conflict and Community Relations}{Electivo}{FG2015}
% Source file: ../Curricula.in/lang/English/cycle/2020-I/Syllabi/GeneralEducation/FG2015.tex

\begin{justification}
This course provides theoretical and methodological tools to handle social conflicts in the extractive and energy industries. In the first part, the students will approach to the dimensions of a social conflict, its possible causes and the various motivations of the actors involved. The second part provides tools and techniques to handle conflicts and resolve them peacefully. The course includes theoretical discussion and case studies.

\end{justification}

\begin{goals}
\item Capacidad de interpretar información.
\end{goals}

\begin{outcomes}{V1}
    \item \ShowOutcome{d}{2}
    \item \ShowOutcome{e}{2}
    \item \ShowOutcome{n}{2}
    
\end{outcomes}

\begin{competences}{V1}
    \item \ShowCompetence{C10}{d,n}
    \item \ShowCompetence{C17}{d}
    \item \ShowCompetence{C18}{n}
    \item \ShowCompetence{C21}{e}
\end{competences}

\begin{unit}{Culturas de Gobernanza y Distribución de Poder}{}{Lessig15}{12}{4}
   \begin{topics}
      \item ?`Cómo se relaciona la economía con la política?.
      \item El rol de las Instituciones.
      \item Análisis de casos.
   \end{topics}
   \begin{learningoutcomes}
      \item Desarrollo del innterés por conocer sobre temas actuales en la sociedad peruana y el mundo.
   \end{learningoutcomes}
\end{unit}

\begin{coursebibliography}
\bibfile{GeneralEducation/FG2015}
\end{coursebibliography}

\end{syllabus}
