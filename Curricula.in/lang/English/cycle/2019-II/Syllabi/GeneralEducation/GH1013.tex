\begin{syllabus}

\course{GH0013. Critique of Modernity}{Obligatorio}{GH0013}
% Source file: ../Curricula.in/lang/English/cycle/2019-II/Syllabi/GeneralEducation/GH1013.tex

\begin{justification}
The preparation course for pre-vocational practice I is the first of two courses in the area of personal and professional competence development. This course provides
Opportunity for the student to self-assess and recognize the opportunities for improvement, to feel capable of accomplishing each of the challenges proposed to him at the
Personal and professional and to carry out an adequate analysis of their situation, execution and proposed resolution.
Learning through experience will help you to evaluate from your own perspective, different ways of thinking and the ability to contribute individually or as a team to the achievement of a specific goal; Through the exchange of ideas, the
Evaluation of proposals and the execution of the ideal solution alternative.

\end{justification}

\begin{goals}
    \item Understands professional and ethical responsibilities.
    \item Oral communication skills.
    \item Written communication skills.
    \item Recognizes the need for lifelong learning.
\end{goals}

\begin{outcomes}{V1}
    \item \ShowOutcome{d}{2} % Multidiscip teams
    \item \ShowOutcome{e}{2} % ethical, legal, security and social implications
    \item \ShowOutcome{f}{2} % communicate effectively
    \item \ShowOutcome{n}{2} % Apply knowledge of the humanities
\end{outcomes}

\begin{competences}{V1}
    \item \ShowCompetence{C10}{d,n}
    \item \ShowCompetence{C17}{f}
    \item \ShowCompetence{C18}{f}
    \item \ShowCompetence{C21}{e}
\end{competences}

\begin{unit}{Introduction to the course.}{}{Robbins05}{12}{4}
   \begin{topics}
      \item Introduction to the course. Presentation of the methodology to be applied (types of evaluation, class schedule, workshops).
      \item Presentation to the competencies that are sought to develop (leadership, teamwork, critical thinking, among others). Theory of competencies and what the market wants.
      \item DISC Shipping - online.
   \end{topics}
   \begin{learningoutcomes}
      \item Develop assertive and proactive methods for project presentation.
   \end{learningoutcomes}
\end{unit}

\begin{unit}{Self-awareness.}{}{Gomez09}{24}{3}
   \begin{topics}
      \item Review of class commitments and agreements.
      \item Assessment center of self-evaluation.
      \item Dynamics on self-knowledge, personal SWOT identification and personal vision
   \end{topics}

   \begin{learningoutcomes}
      \item Recognize strengths and improvement points.
      \end{learningoutcomes}
\end{unit}

\begin{unit}{Personal branding.}{}{Robbins05}{24}{3}
   \begin{topics}
      \item Theory. Personal branding. Plan of positioning in the market (as I will make my competences can enter the market).
      \item How one should communicate, the use of voice to enhance their skills and achieve security and effectiveness in their communication.
      \item Theory What is a CV? How to create an innovative CV? Implementation of projects; Data updating, project portfolio building; Virtual communication.
      \item Creation of a CV by group.
   \end{topics}

   \begin{learningoutcomes}
      \item Learn to show your personal brand through different professional and social media.
   \end{learningoutcomes}
\end{unit}

\begin{unit}{Interviews}{}{Robbins05}{30}{3}
   \begin{topics}
      \item Talk: Labor Market Expectations. What does the market want and want?
      \item Types of interviews and evaluations in the recruitment and selection process. Use of persuasion strategies; Forms and techniques for interviews, tips and recommendations.
      \item Delivery of Challenge 1: Sending the CV
      \item VOLCAN Talk: "Interview Tips and Recruitment and Selection Evaluations on Knowledge Thursday
      \item Task. After knowing and knowing what the market wants, the elements are created to design each student's own strategy
   \end{topics}

   \begin{learningoutcomes}
      \item Ability to face a job interview.
   \end{learningoutcomes}
\end{unit}

\begin{unit}{Virtual Platforms.}{}{Gomez09}{24}{3}
   \begin{topics}
      \item Conversation of the Volcan talk and the previous class.
      \item Virtual job platforms: review of the main virtual platforms (CSM), correct use of the UTEC stock exchange.
      \item Linkedin as creator of relationships: introduction to the social network; Utility and transcendence at present; LinkedIn usage rules and tools. LinkedIn Student Exposure and Similar Tools
      \item Explanation of networkingUTEC.   
   \end{topics}

   \begin{learningoutcomes}
      \item Learn the job search techniques and criteria for a good performance in the professional evaluations.
   \end{learningoutcomes}
\end{unit}

\begin{unit}{Networking}{}{Robbins05}{24}{3}
   \begin{topics}
      \item Networking UTEC: Challenge 2: Get an interview. Send your CV to 40 companies. Take a picture with your clothes     
   \end{topics}

   \begin{learningoutcomes}
      \item Develop professional competences oriented to the assertiveness of the search for job opportunities.
   \end{learningoutcomes}
\end{unit}

\begin{unit}{Feedback.}{}{Bolles15}{24}{3}
   \begin{topics}
      \item Survey-Feedback of UTEC Networking.
      \item Dynamics - Challenge 3: Interviews Circle by competencies with professionals.
   \end{topics}

   \begin{learningoutcomes}
      \item Foment a positive attitude towards work and a passion for excel Pre-professional practices through the detection of opportunities for improvement.
   \end{learningoutcomes}
\end{unit}

\begin{unit}{Partial Exam.}{}{Gomez09}{24}{3}
   \begin{topics}
      \item Partial exams (Competency interviews, CV and LinkedIn review)
   \end{topics}

   \begin{learningoutcomes}
      \item Develop in the student the ability to approach situations or problems assertively, with an integrative look.
   \end{learningoutcomes}
\end{unit}

\begin{unit}{Assessment Center.}{}{Robbins05}{24}{3}
   \begin{topics}
      \item Delivery and discussion of assessment center results.
      \item Assessment center in the classroom, with the application of real cases, in the measurement of competences: Proactivity, problem analysis, analytical thinking and planning and organization, teamwork and leadership, adaptability, assertive communication, ethics.
    \end{topics}

   \begin{learningoutcomes}
      \item Develop the ability to recognize and propose solutions to opportunities for improvement within the work environment, using their own resources, skills and interpersonal relationships.
   \end{learningoutcomes}
\end{unit}

\begin{unit}{Debate}{}{Bolles15}{24}{3}
   \begin{topics}
      \item Conversation about the results of the previous class.
      \item Out-of-class workshop: Proactivity, problem analysis, analytical thinking and planning and organization, teamwork and leadership, adaptability, assertive communication, ethics. You will have the feedback.
   \end{topics}

   \begin{learningoutcomes}
      \item Encourage a positive attitude towards work.
   \end{learningoutcomes}
\end{unit}

\begin{unit}{Labor rights and obligations of practitioners}{}{Gomez09}{24}{3}
   \begin{topics}
      \item Charla: Labor rights and obligations of practitioners.
   \end{topics}
   \begin{learningoutcomes}
      \item Ability to recognize rights and responsibilities.
   \end{learningoutcomes}
\end{unit}

\begin{unit}{Interview with experts.}{}{Robbins05}{24}{3}
   \begin{topics}
      \item Experts interview: The real world of work from the vision of the area of human resources - RHHH, with corporate heads of selection as guests.
   \end{topics}
   \begin{learningoutcomes}
      \item Develop professional competences oriented to the assertiveness of the search of job opportunities.
   \end{learningoutcomes}
\end{unit}

\begin{unit}{Recruitment dynamics}{}{Bolles15}{24}{3}
   \begin{topics}
      \item Conversation and presentation of the students about the interview to experts and related topics.
      \item Reinforcement of recruitment and selection evaluations.
      \item Recruiting Dynamics: Challenge 4: How's it going?
   \end{topics}

   \begin{learningoutcomes}
      \item Foment a positive attitude towards work and a passion for excel Pre-professional practices through the detection of opportunities for improvement and the formulation of an engineering project.
   \end{learningoutcomes}
\end{unit}

\begin{unit}{Personal skills}{}{Robbins05}{24}{3}
   \begin{topics}
      \item Case Study
      \item Competencies: Planning and organization and self-confidence, associated with problem solving.
      \item Results feedback on the dynamics of reinforcement.
   \end{topics}

   \begin{learningoutcomes}
      \item develop in the student the ability to approach situations assertively or problems, with an integrative look for later, propose and execute some of the alternatives to the solution of the same
   \end{learningoutcomes}
\end{unit}

\begin{coursebibliography}
\bibfile{GeneralEducation/GH2015}
\end{coursebibliography}

\end{syllabus}
