\begin{syllabus}

\course{EG0009. Interdisciplinary Project III}{Obligatorio}{EG0009}
% Source file: ../Curricula.in/lang/English/cycle/2020-I/Syllabi/Interdisciplinary/XD301.tex

\begin{justification}
Interdisciplinary Projects I is a course in which students work in teams in a research and development or entrepreneurship project, in order to propose a solution to a relevant problem. The development of the project focuses on the use of engineering, technology and computer science tools to propose solutions to technical, technological, scientific and / or social problems. The integration of
Knowledge and multidisciplinary and interdisciplinary aspects is an essential element for the success of the project. Throughout the course, the student learns about the design process, to apply the contents of his career to a real context; To identify and acquire new relevant knowledge; And to collaborate interdisciplinarily. In this third course of Interdisciplinary Projects, the student is exposed to problems of moderate complexity, with low level of uncertainty in the problem and solution, and has the support and close supervision of the project advisor. The course emphasizes the development and reinforcement of effective communication and collaboration skills to foster the formation of high performance teams. 
It learns to manage projects, applying good practices and international standards.
\end{justification}

\begin{goals}
   \item Identify problems
   \item Design a component or process to meet the desired needs within realistic constraints
   
\end{goals}

\begin{outcomes}{V1}
   \item \ShowOutcome{f}{2}
   \item \ShowOutcome{n}{2}
\end{outcomes}

\begin{competences}{V1}
    \item \ShowCompetence{C17}{f}
    \item \ShowCompetence{C19}{n}
\end{competences}

\begin{unit}{Interdisciplinary Project  III }{}{Zobel}{16}{C17}
\begin{topics}
      \item Develop ideas related to the multiple discipiplinas that bring the student to a real idea of a company.
\end{topics}

\begin{learningoutcomes}
   \item Development of critical thinking in decision making in product design processes or conducting research.
\end{learningoutcomes}
\end{unit}

\begin{coursebibliography}
\bibfile{Interdisciplinary/XD101}
\end{coursebibliography}

\end{syllabus}
