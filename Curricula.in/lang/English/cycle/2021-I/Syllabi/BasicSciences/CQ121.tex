\begin{syllabus}

\course{QI0027. General Chemistry}{Obligatorio}{QI0027}
% Source file: ../Curricula.in/lang/English/cycle/2020-I/Syllabi/BasicSciences/CQ121.tex

\begin{justification}
This course is useful in this career so the student learns to show a high degree of mastery of the laws of General Chemistry
\end{justification}

\begin{goals}
\item Train and present to the student the basic principles of chemistry as a natural science encompassing its most important topics and their relationship with everyday problems.
\end{goals}

\begin{outcomes}{V1}
\ShowOutcome{d}{2}
\ShowOutcome{h}{2}
\end{outcomes}

\begin{competences}{V1}
    \item \ShowCompetence{C20}{d,h}
\end{competences}

\begin{unit}{QU1. Thermodynamics}{}{Raymond99,Kennet92}{4}{C20}
\begin{topics}
     
      \item Thermodynamic systems and their classification. Thermodynamic variables and state functions.
      \item States of a system. States of equilibrium. Extensive and intensive variables.
      \item Thermal equilibria. Zero Principle of Thermodynamics
      \item First Law of thermodynamics. Heat capacity. Reversible processes and maximum work.
      \item Internal energy of ideal gases. Adiabatic transformations. Thermo-chemistry. Law of Lavoisier and Place, Law of Hess. Kirchhoff's Law.
      \item Second law of thermodynamics. Entropy. Efficiency of a reversible cycle.
      \item Free energy. Third Law of Thermodynamics.
\end{topics}

   \begin{learningoutcomes}
      \item Understand and work with the principles of thermodynamics.
      \item Abstracting the concepts of gas transformations from nature.
   \end{learningoutcomes}
\end{unit}

\begin{unit}{QU2. Chemical balance}{}{Raymond99,Kennet92}{4}{C20}
\begin{topics}
      \item Concept. Balance constant.
      \item Law of action of the masses.
      \item Homogeneous equilibria. Heterogeneous equilibria. Multiple Balances.
      \item Factors Affecting Chemical Balance.Le Chatelier's principle.
    \end{topics}

   \begin{learningoutcomes}
      \item Describe, know and apply the concepts of chemical balance.
      \item Resolve problems.
   \end{learningoutcomes}
\end{unit}

\begin{unit}{QU3. Studies that Contributed to the Development of Atom Theory}{}{Raymond99}{4}{C20}
\begin{topics}
      \item Properties of waves.
      \item Electromagnetic radiation. Characteristic. Spectrum
      \item Quantum Theory of Max Planck.
      \item Photoelectric effect.
      \item Relation between matter and energy.
      \item X-ray, cathode ray and canals.
      \item Exercises and problems
\end{topics}

   \begin{learningoutcomes}
      \item Describe the behavior and characteristics of waves.
      \item Understand qualitatively and quantitatively the corpuscular behavior of electromagnetic waves.
      \item Solve problems.
   \end{learningoutcomes}
\end{unit}

\begin{unit}{QU4. Theories of the atom}{}{Babor83,Kennet92}{6}{C20}
\begin{topics}
      \item Postulates of Dalton. Atomic model of Thomson.
      \item Rutherford's experiment, Rutherford's atomic model. Inconsistency.
      \item Bohr's atomic model. Spectrum of emission of the hydrogen atom.
      \item Modern atomic theory. Duality of matter.
      \item Heisenberg uncertainty principle.
      \item Atomic orbitals. Schrodinger equation.
      \item Quantum mechanical description of the hydrogen atom. Quantum numbers.
      \item Electronic configuration. Pauli Exclusion Principle.
      \item Hund's rule. Exceptions.
      \item Paramagnetism and diamagnetism. Screen Effect.
      \item Exercises and problems.
   \end{topics}

   \begin{learningoutcomes}
      \item Understand and interpret classical atomic models.
      \item Understand the fundamentals of modern atomic theory.
      \item Know the basic concepts of quantum mechanics.
      \item Solve problems.

   \end{learningoutcomes}
\end{unit}

\begin{unit}{QU5. Periodic Table}{Kennet92,Mahan92}{}{4}{C20}
\begin{topics}	
	\item Periodic law.
	\item Periodic law. Description of the periodic table. Period and group. Location of an item.
	\item Periodic properties: Atomic radius, ionic radius, ionization energy, electron affinity. Electronegativity.
	\item Variation of chemical properties.
	\item Exercises and problems.

   \end{topics}

   \begin{learningoutcomes}
      \item Understand the structure of the periodic table.
      \item Know the properties of the elements.
      \item Solve problems.

   \end{learningoutcomes}
\end{unit}

\begin{unit}{QU6. Chemical bond}{}{Mahan92,Ander83}{3}{C20}
   \begin{topics}
        \item Valence theory. Evolution.
	\item Octet Rule.
	\item Lewis Theory.
	\item Ion and electrovalent bond.
	\item Formation of the element between the elements $ s $ and the elements $ p $. The ionic energies of the crystalline netting.
	\item Born Haber Cycle.
	\item Covalent bond. Sharing electron pairs.
	\item Formal charge and Lewis structure. Concept of resonance.
	\item Exceptions to the octet rule. Forces in covalent bond
	\item Theory of the repulsion of electronic pairs of the valence level (RPENV).
	\item Concept of hybridization. Hybridization sp, sp2, sp3 and other types of hybridization.
	\item Molecular Orbital Theory.
	\item Exercises and problems.
   \end{topics}

   \begin{learningoutcomes}

      \item Know and understand the theories of valence and chemical bonds.
      \item Know and understand the molecular orbital theory.
      \item Solve problems
   \end{learningoutcomes}
\end{unit}

\begin{unit}{QU7. Gases}{}{Ander83,Masterton98}{4}{C20}
\begin{topics}
      \item Definition. Pressure of a gas.
      \item Laws of gases: by Boyle, Gay-Lussac and Charles. Equation of an ideal gas.
      \item Dalton's partial pressure law.
      \item Kinetic theory of gases. Distribution of molecular velocities. Average Free Path.
      \item Graham's Law of Diffusion and Effusion.
      \item Real gas. Van der Waals equation.
      \item Exercises and problems.
   \end{topics}

   \begin{learningoutcomes}

      \item Know the basic concepts of ideal gases.
      \item Understand and apply the kinetic theory of gases.
      \item know concepts of diffusion and effusion of gases.
      \item Understand the concepts of real gases.
      \item Solve problems.

   \end{learningoutcomes}
\end{unit}

\begin{unit}{QU8. Intermolecular Forces and Liquids}{}{Masterton98,Babor83}{3}{C20}
\begin{topics}

      \item Definition. Evaporation and vapor pressure at steady state.
      \item Measurement of vapor pressure and heat of vaporization. Boiling point and latent heat of vaporization.
      \item Intermolecular forces; Dipole-dipole forces, ion-dipole, stray, force and van der Waals radius. Hydrogen bond.
      \item Viscosity. Surface tension and capillary action.
      \item Phase changes.
      \item Ejercicios y problemas.
    \end{topics}

   \begin{learningoutcomes}
      \item Understand basic concepts of intermolecular forces.
      \item Know and apply concepts of vaporization and boiling.
      \item Know and apply concepts of surface tension and phase changes.
      \item Solve Problems.
   \end{learningoutcomes}
\end{unit}

\begin{unit}{QU9. Solids}{}{Masterton98,Babor83}{3}{C20}
\begin{topics}

      \item Definition. Packing of spheres. Efficiency of packaging. Compact packing.
      \item Use of X-rays in the study of the structure of crystals.
      \item Classes of crystal structures: ionic crystals. Covalent, molecular, metallic. Metal Link Amorphous crystals.
      \item Phase changes. Liquid-vapor balance. Heat of vaporization and boiling point.
      \item Liquid-solid equilibrium. Solid-vapor equilibrium. Phase diagram of water and carbon dioxide.
      \item Exercises and problems.
    \end{topics}

   \begin{learningoutcomes}
      \item Understand basic concepts of crystalline solid structures.
      \item Know and apply concepts of phase changes and equilibrium.
      \item Solve problems .
   \end{learningoutcomes}
\end{unit}

\begin{unit}{QU10. Solutions}{}{Masterton98,Babor83}{3}{C20}
\begin{topics}
      \item Definition. Molecular vision of the dissolution process.
      \item Dissolutions of liquids in liquids. Dissolutions of solids in liquids.
      \item Units of concentration: percentage by mass, molar fraction, molarity, molality Normality.
      \item Effect of temperature on solubility, solubility of solids and temperature, fractional crystallization.
      \item Solubility of gases and temperature. Effect of pressure on the solubility of gases.
      \item Colligative properties of solutions. Colloidal Dispersions.
      \item Exercises and problems.
    \end{topics}

   \begin{learningoutcomes}
      \item Understand basic concepts of molecular  dissolution.
      \item Know and apply concepts of concentration and solubility.
      \item Solve problems
   \end{learningoutcomes}
\end{unit}

\begin{unit}{QU11. Stoichiometry}{}{Masterton98,Babor83}{3}{C20} 
\begin{topics}

      \item Chemical reaction. Expressions of chemical reactions in the form of equations. Characteristics of a chemical equation.
      \item Types of chemical reactions: Precipitation, acid-base, oxide-reduction. Number of reagents and products.
      \item Stoichiometric relations: moles, mass and volume.
      \item Weight and Volume Laws.
      \item Limiting Reagent. Yield of reactions.
      \item Exercises and problems.
    \end{topics}

   \begin{learningoutcomes}

      \item Know basic concepts of chemical reactions
      \item Know and apply weight and volume laws.
      \item Resolve problems .
   \end{learningoutcomes}
\end{unit}

\begin{coursebibliography}
\bibfile{BasicSciences/CQ121}
\end{coursebibliography}
\end{syllabus}
