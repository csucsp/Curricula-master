 \chapter*{Resumen ejecutivo}
\AbstractIntro

Todo el contenido del documento está basado en la propuesta internacional denominada \textit{Computing Curricula}\footnote{http://www.sigcse.org/cc2001/} 
en el área específica de Ciencia de la Computación. Este documento es el resultado de un trabajo conjunto de la 
\textit{Association for Computing Machinery} (ACM) y la Sociedad de Computación de IEEE (IEEE-CS) y 
puede ser accesado a través de la dirección \href{http://www.acm.org/education}{http://www.acm.org/education} 
en sus versions CS2001, CS2008 y \href{cs2013.org}{CS2013}.

Considerando que existen peculiaridades menores al aplicar esta propuesta internacional a nuestros paises, el modelo de \textit{Computing Curricula} 
fue utilizado para proponer el documento base de la presente malla. 

\noindent La computación hoy en día presenta 5 perfiles de formación profesional claramente definidos: 
\begin{itemize}
\item \textbf{Ciencia de la Computación} (\textit{Computer Science} -- CS),
\item Ingeniería de Computación (\textit{Computer Engineering} -- CE),
\item Sistemas de Información (\textit{Information Systems} -- IS),
\item Ingeniería de Software (\textit{Software Engineering} -- SE) y 
\item Tecnología de la Información (\textit{Information Technology} -- IT).
\end{itemize}

Los pilares fundamentales que consideramos en esta propuesta curricular son:
\begin{itemize}
\item Una sólida formación profesional en el área de Ciencia de la Computación,
\item Preparación para la generación de empresas de base tecnológica,
\item Una sólida formación ética y proyección a la sociedad
\end{itemize}

Estos pilares redundarán en la formación de profesionales que se puedan desempeñar en 
cualquier parte del mundo y que ayuden de forma clara al desarrollo de la Industria 
de Software de nuestro país. 

\OtherKeyStones

El resto de este documento está organizado de la siguiente forma: el Capítulo \ref{chap:intro}, 
define y explica el campo de acción de la Ciencia de la Computación, 
además se hace una muy breve explicación de las distintas carreras del área de 
computación propuestas por IEEE-CS y ACM.


\OnlyPeruSPC{
En el Capítulo \ref{chap:cs-market} se presenta el perfil profesional, un análisis de mercado 
que incluye un análisis de la oferta, demanda y tendencias del conocimiento.

Un análisis más detallado del mercado junto con una encuesta a empresarios es presentada en el 
Capítulo \ref{chap:cs-estudio-de-mercado}. Este estudio arrojá datos interesantes con 
relación a la percepción y la necesidad empresarial de este perfil profesional.
}

\OnlyPeruUNSA{
En el Capítulo \ref{chap:cs-market} se presenta el perfil profesional, un análisis de mercado 
reconocidas que incluye un análisis de la oferta, demanda y tendencias del conocimiento.

Un análisis más detallado del mercado junto con una encuesta a empresarios es presentada en el 
Capítulo \ref{chap:cs-estudio-de-mercado}. Este estudio arrojá datos interesantes con relación 
a la percepción y la necesidad empresarial de este perfil profesional.

En el Capítulo \ref{chap:cs-resources} se presentan los recursos de plana docente, infraestructura de 
laboratorios y financieros necesarios para poder crear esta carrera profesional.
}

El Capítulo \ref{chap:BOK}, muestra las áreas de Conocimiento de la Ciencia de la Computación, 
indicando los tópicos y objetivos de aprendizaje de los temas, pertenecientes a estos grupos.

El Capítulo \ref{chap:GeneralInfo} contiene la distribución por semestres, por áreas, por niveles, 
visión gráfica de la malla curricular, comparación con las diversas propuestas internacionales, 
distribución de tópicos por curso así como la distribución de habilidades por materia.

El Capítulo \ref{chap:syllabi} contiene información detallada para cada uno de los cursos 
incluyendo las habilidades con las cuales contribuye, bibliografía por cada unidad así 
como el número de horas mínimas por cada unidad.

\OnlyPeruUCSP{En el Capítulo \ref{chap:equivalence} se presentan las tablas de equivalencias con otros 
planes curriculares.}

En el Capítulo \ref{chap:laboratories} se presenta una sugerencia de los laboratorios 
requeridos para el dictado de clases las mismas que podrían variar de acuerdo al volumen de alumnos que se tenga.

\OnlyPeruSPC{   En el Capítulo~\ref{chap:professors-and-courses} podemos observar un listado de cursos por profesor y de profesor por curso}
\OnlyPeruMINEDU{En el Capítulo~\ref{chap:professors-and-courses} podemos observar un listado de cursos por profesor y de profesor por curso}

\OnlyPeruSPC{   En el Capítulo~\ref{sec:specific-outcomes-by-course} se puede observar un listado de cursos por cada resultado específico.}
\OnlyPeruMINEDU{En el Capítulo~\ref{sec:specific-outcomes-by-course} se puede observar un listado de cursos por cada resultado específico.}

\newpage