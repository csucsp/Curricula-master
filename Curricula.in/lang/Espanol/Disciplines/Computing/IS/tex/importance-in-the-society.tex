\section{Importancia de la carrera en la sociedad}\label{sec:importance-in-the-society}
Tomando en cosideración que estamos en la era de la información, es claro que cualquier 
toma de decisiones organizacionales depende de que se disponga 
de la información adecuada en el momento en que la necesitamos.

En los tiempos en que vivimos la cantidad de información que una organización genera 
es abrumadora y ya no es posible procesar tanta información de forma manual. 
Por esa razón, carreras dedicadas a la administración de las organizaciones 
presentan una seria limitante pues, de forma manual, sólo les es posible analizar 
un volumen reducido de datos para tomar decisiones. 

Por otro lado, existen los profesionales altamente tecnológicos que crean tecnología 
pero no tienen fuertes bases en temas organizacionales. El profesional en 
Sistemas de Información aparece como un excelente punto de equilibrio que sabe llevar la 
tecnología a la organización para hacerla eficiente a gran escala en um ambiente globalizado.

Nuestro profesional también tiene la ventaja de poder tomar decisiones con mayor rapidez y precisión 
que un gerente tradicional pues él está preparado para procesar información voluminosa de forma directa. 
Es esta información procesada la que finalmente le da los elementos necesarios para
para una adecuada toma de decisiones organizacionales.

Finalmente, debemos tener siempre en mente que los alumnos que ingresan hoy 
saldrán al mercado dentro de 5 años aproximadamente y, en un mundo que cambia tan rápido, 
no podemos ni debemos enseñarles tomando en cuenta solamente el mercado actual y/local. 
Nuestros profesionales deben estar preparados para resolver los problemas 
que habrá dentro de 10 o 15 años y eso sólo es posible a través de formación de la habilidad 
de aprender a aprender por su propia cuenta. Esta habilidad no sólo es necesaria para para una larga 
vida profesional activa sino que de eso depende directamente que sea un motor de impulso en las 
organizaciones donde se desempeñe.
