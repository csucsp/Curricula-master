\newcommand{\ContribInitMsg}{Esta disciplina contribuye al logro de los siguientes resultados de la carrera\xspace}
\newcommand{\CompetencesInitMsg}{Esta disciplina contribuye a la formación de las siguientes competencias\xspace}

% ...

\newcommand{\Familiarity}{Familiarizarse}
\DefineCompetenceLevel{1}{\Familiarity}
\newcommand{\Usage}{Usar}
\DefineCompetenceLevel{2}{\Usage}
\newcommand{\Assessment}{Evaluar}
\DefineCompetenceLevel{3}{\Assessment}

\newcommand{\LearningOutcomesTxtEsFamiliarity}{El estudiante \textbf{entiende} lo que un concepto es o qué significa. Este nivel de dominio \textbf{se refiere a un conocimiento básico} de un concepto en lugar de esperar instalación real con su aplicación. Proporciona una respuesta a la pregunta: \textbf{?`Qué sabe usted de esto?}}
\newcommand{\LearningOutcomesTxtEsUsage}{El alumno es capaz de \textbf{utilizar o aplicar} un concepto de una manera concreta. El uso de un concepto puede incluir, por ejemplo, apropiadamente usando un concepto específico en un programa, utilizando una técnica de prueba en particular, o la realización de un análisis particular. Proporciona una respuesta a la pregunta: \textbf{?`Qué sabes de cómo hacerlo?}}
\newcommand{\LearningOutcomesTxtEsAssessment}{El alumno es capaz de \textbf{considerar un concepto de múltiples puntos de vista} y/o \textbf{justificar la selección de un determinado enfoque} para resolver un problema. Este nivel de dominio implica más que el uso de un concepto; se trata de la posibilidad de seleccionar un enfoque adecuado de las alternativas entendidas. Proporciona una respuesta a la pregunta: \textbf{?`Por qué hiciste eso?}}

\newcommand{\LearningOutcomesTxtEnFamiliarity}{The student understands what a concept is or what it means. This level of mastery concerns a basic awareness of a concept as opposed to expecting real facility with its application. It provides an answer to the question: What do you know about this?}
\newcommand{\LearningOutcomesTxtEnUsage}{The student is able to use or apply a concept in a concrete way. Using a concept may include, for example, appropriately using a specific concept in a program, using a particular proof technique, or performing a particular analysis. It provides an answer to the question: What do you know how to do?}
\newcommand{\LearningOutcomesTxtEnAssessment}{The student is able to consider a concept from multiple viewpoints and/or justify the selection of a particular approach to solve a problem. This level of mastery implies more than using a concept; it involves the ability to select an appropriate approach from understood alternatives. It provides an answer to the question: Why would you do that?}

%DefineOutcome{1}{Analizar un problema computacional complejo y aplicar principios de computación y otras disciplinas para identificar soluciones.\xspace}
%DefineOutcome{1Short}{Aplicar conocimientos de computación y otras disciplinas.\xspace}
\DefineOutcome{1}{Evalúa las necesidades del cliente y del entorno e identificar los requisitos de software para generar soluciones integrales e innovadoras optimizando los recursos tecnológicos, de capital humano, costo y tiempo.\xspace}
\DefineOutcome{1Short}{Evalua las necesidades del cliente.\xspace}
\DefineSpecificOutcome{1}{1}{1.1}{Aplicar el pensamiento computacional de manera efectiva en la solución de problemas cotidianos.\xspace}
\DefineSpecificOutcome{1}{2}{1.2}{Evaluar diferentes propuestas pensamiento computacional para un mismo problema.\xspace}
\DefineSpecificOutcome{1}{3}{1.3}{Aplicar la robótica como un medio para desarrollar pensamiento computacional.\xspace}
\DefineSpecificOutcome{1}{4}{1.4}{Identificar y aplicar de forma eficiente diversas estrategias algorítmicas y estructuras de datos para la solución de un problema dadas ciertas restricciones de espacio y tiempo.\xspace}
\DefineSpecificOutcome{1}{5}{1.5}{Identificar y aplicar de forma eficiente diversas estrategias algorítmicas y estructuras de datos para la solución de un problema en ambientes paralelos y distribuidos.\xspace}
\DefineSpecificOutcome{1}{6}{1.6}{Implementar soluciones distribuídas utilizando MapReduce.\xspace}
\DefineSpecificOutcome{1}{7}{1.7}{Implementar soluciones distribuídas utilizando bases de datos NoSql.\xspace}
\DefineSpecificOutcome{1}{8}{1.8}{Aplicar técnicas de aprendizaje de máquina sobre grandes volumenes de datos.\xspace}
\DefineSpecificOutcome{1}{9}{1.9}{Aplicar técnicas de aprendizaje de máquina para el procesamiento y análisis de grandes volumenes obtenidos en tiempo real.\xspace}
\DefineSpecificOutcome{1}{10}{1.10}{Implementar soluciones distribuidas usando bases de datos de grafos.\xspace}
\DefineSpecificOutcome{1}{11}{1.11}{Entender la diferencia entre un problema NP-difícil y uno que tiene solución polinomial.\xspace}
\DefineSpecificOutcome{1}{12}{1.12}{Dado un problema con solución polinomial, identificar si es posible resolverlo mediante una estrategia voraz, mediante una estrategia de programación dinámica o una de división y conquista tomando en cuenta el tamaño de la entrada.\xspace}
\DefineSpecificOutcome{1}{13}{1.13}{Modelar base de datos a traves de modelos ER, MR, optimización, transacciones y recuperación de la información.\xspace}
\DefineSpecificOutcome{1}{14}{1.14}{Diseñar los recursos de hardware apropiados para computar.\xspace}
\DefineSpecificOutcome{1}{15}{1.15}{Aprender y aplicar técnicas y mecanismos para realizar cómputo eficiente en hardware.\xspace}
\DefineSpecificOutcome{1}{16}{1.16}{Entender la implementación de un Sistema Operativo para realizar un uso eficiente del hardware disponible en un sistema.\xspace}
\DefineSpecificOutcome{1}{17}{1.17}{Estudiar y aplicar diversas técnicas de control eficiente de recursos para lograr el procesamiento de información.\xspace}
\DefineSpecificOutcome{1}{18}{1.18}{Definir requerimientos en un proyecto final integrado.\xspace}
\DefineSpecificOutcome{1}{19}{1.19}{Entender la diferencia entre un problema sin solución (problema indecidible) y un problema que sí posee solución.\xspace}
\DefineSpecificOutcome{1}{20}{1.20}{Identificar y resolver un problema que es resoluble mediante teoría de autómatas y reconocer cuál es el tipo de automata más simple que resuelve el problema.\xspace}
\DefineSpecificOutcome{1}{21}{1.21}{Realizar el presupuesto requerido para ejecutar una solución basada en software.\xspace}
\DefineSpecificOutcome{1}{22}{1.22}{Escribir programas a partir de una especificación práctica y producir gráficos realísticos.\xspace}
\DefineSpecificOutcome{1}{23}{1.23}{Escribir programas orientados a resolver problemas de nuestro entorno utilizando computación gráfica.\xspace}
\DefineSpecificOutcome{1}{24}{1.24}{Analizar el poder computacional necesario para poder descifrar y calcular el valor de una llave.\xspace}
\DefineSpecificOutcome{1}{25}{1.25}{Analizar y entender el contexto de un problema para solucionarlo a través de la robótica.\xspace}
\DefineSpecificOutcome{1}{26}{1.26}{Describir la rotación de un cuerpo rígido en términos de coordenada angular, velocidad angular y aceleración angular.\xspace}
\DefineSpecificOutcome{1}{27}{1.27}{Analizar la rotación de un cuerpo rígido cuando la aceleración angular es constante.\xspace}
\DefineSpecificOutcome{1}{28}{1.28}{Relacionar la rotación de un cuerpo rígido con la velocidad y la aceleración lineales de un punto en el cuerpo.\xspace}
\DefineSpecificOutcome{1}{29}{1.29}{El significado del momento de inercia del cuerpo en torno a un eje y cómo se relaciona con la energía cinética rotacional.\xspace}

%DefineOutcome{2}{Diseñar, implementar y evaluar una solución basada en computación para cumplir con un conjunto dado de requerimientos computacionales en el contexto de la disciplina del programa.\xspace}
%DefineOutcome{2Short}{Diseñar, implementar y evaluar una solución basada en computación.\xspace}
\DefineOutcome{2}{Aplica tópicos de investigación, metodologías, técnicas y mejores prácticas de la Ingeniería de Software para la construcción de soluciones en base al diseño, desarrollo, pruebas, implementación, documentación y mejora continua del Software..\xspace}
\DefineOutcome{2Short}{Aplica investigación en solución de problemas de Ing de Software.\xspace}

%DefineOutcome{3}{Comunicar efectivamente en varios contextos profesionales.\xspace}
%DefineOutcome{3Short}{Comunicar efectivamente en varios contextos profesionales.\xspace}
\DefineOutcome{3}{Justifica un desempeño individual, como parte de equipos de trabajo o como líder de proyectos de grupos multidisciplinarios en entornos globales con el fin de asegurar la calidad de software, aplicando normas, configuraciones, regulaciones y métricas.\xspace}
\DefineOutcome{3Short}{Justifica un desempeño individual, como parte de equipos de trabajo.\xspace}

\DefineSpecificOutcome{3}{1}{3.1}{Transmitir de forma clara propuestas técnicas a audiencias de otras áreas.\xspace}
\DefineSpecificOutcome{3}{2}{3.2}{Transmitir propuestas técnicas del area de computación en inglés\xspace}
\DefineSpecificOutcome{3}{3}{3.3}{Transmitir propuestas técnicas en Inglés a audiencias de otras áreas.\xspace}
\DefineSpecificOutcome{3}{4}{3.4}{Presentar un plan de negocio a potenciales inversionistas.\xspace}
\DefineSpecificOutcome{3}{5}{3.5}{Escuchar. Puede seguir un discurso muy lento y cuidadosamente articulado para asimilar el significado.\xspace}
\DefineSpecificOutcome{3}{6}{3.6}{Hablar. Puede producir frases simples y aisladas sobre personas y lugares.\xspace}
\DefineSpecificOutcome{3}{7}{3.7}{Leer. Puede entender textos muy cortos y simples, una sola frase a la vez, recogiendo nombres familiares, palabras y frases básicas.\xspace}
\DefineSpecificOutcome{3}{8}{3.8}{Escribir. Puede escribir frases y oraciones simples y aisladas.\xspace}
\DefineSpecificOutcome{3}{9}{3.9}{Escuchar. Puede entender un discurso claro y articulado lentamente relacionado con las áreas de prioridad más inmediata.\xspace}
\DefineSpecificOutcome{3}{10}{3.10}{Hablar. Puede dar una descripción o presentación simple como una serie corta de frases y oraciones sencillas enlazadas en una lista.\xspace}
\DefineSpecificOutcome{3}{11}{3.11}{Leer. Puede entender textos cortos y simples que contienen el vocabulario de mayor frecuencia.\xspace}
\DefineSpecificOutcome{3}{12}{3.12}{Escribir. Puede escribir una serie de frases y oraciones sencillas unidas con conectores simples.\xspace}
\DefineSpecificOutcome{3}{13}{3.13}{Escuchar. Puede comprender lo suficiente para poder satisfacer necesidades de tipo concreto.\xspace}
\DefineSpecificOutcome{3}{14}{3.14}{Leer. Puede entender textos cortos y sencillos sobre asuntos familiares de tipo concreto que consisten en una alta frecuencia de lenguaje cotidiano o relacionado con el trabajo.\xspace}
\DefineSpecificOutcome{3}{15}{3.15}{Escuchar. Puede entender los puntos principales de un discurso claro y estándar sobre asuntos familiares, incluyendo narraciones cortas.\xspace}
\DefineSpecificOutcome{3}{16}{3.16}{Hablar. Puede sostener con razonable fluidez una descripción directa de uno de los diversos temas dentro de su campo de interés.\xspace}
\DefineSpecificOutcome{3}{17}{3.17}{Leer. Puede leer textos factuales sencillos sobre temas relacionados con su campo e interés con un nivel de comprensión satisfactorio.\xspace}
\DefineSpecificOutcome{3}{18}{3.18}{Escribir. Puede escribir textos directamente conectados en una gama de temas familiares, uniendo una serie de elementos discretos más cortos en una secuencia lineal.\xspace}
\DefineSpecificOutcome{3}{19}{3.19}{Escuchar. Puede entender información factual directa con un acento generalmente familiar.\xspace}
\DefineSpecificOutcome{3}{20}{3.20}{Hablar. Puede sostener con razonable fluidez una descripción directa de uno de los diversos temas dentro de su campo de interés, presentándolo como una secuencia lineal de puntos.\xspace}
\DefineSpecificOutcome{3}{21}{3.21}{Manejo de habilidades intelectuales como memoria, concentración, agilidad mental y creatividad.\xspace}
\DefineSpecificOutcome{3}{22}{3.22}{Manejo de la voz, la expresión corporal y facial.\xspace}
\DefineSpecificOutcome{3}{23}{3.23}{Fortalecer la auto estima, la seguridad y superar la timidez (no entendida como introversión).\xspace}
\DefineSpecificOutcome{3}{24}{3.24}{Desarrollar claridad, precisión, corrección lingüística, concisión y elegancia al hablar en un discurso.\xspace}
\DefineSpecificOutcome{3}{25}{3.25}{Desarrollar cualidades de la voz tales como: volumen, velocidad, tono, flexibilidad y entonación por medio de ejercicios de respiración.\xspace}
\DefineSpecificOutcome{3}{26}{3.26}{Desarrollar habilidades knésicas y proxémicas tales como: gestos, uso de las manos (mímicas y ademanes), mirada, posturas corporales y desplazamientos.\xspace}
\DefineSpecificOutcome{3}{27}{3.27}{Estructurar discursos para diferentes ámbitos como: científicos, académicos y sociales.\xspace}
\DefineSpecificOutcome{3}{28}{3.28}{Aplicar herramientas de liderazgo de equipos tales como: comunicación efectiva, inteligencia emocional, gestión del tiempo, toma de decisiones, creatividad e innovación, mentoría.\xspace}

%DefineOutcome{4}{Reconocer responsailidades profesionales y hacer juicios informados en prácticas computacionales basados en principios éticos y legales.\xspace}
%DefineOutcome{4Short}{Reconocer responsabilidades profesionales y hacer juicios informados.\xspace}
\DefineOutcome{4}{Diseña soluciones de Software de acuerdo a los estándares y políticas de seguridad de la información en uno o varios dominios de aplicación siendo socialmente responsables y demostrando ética profesional.\xspace}
\DefineOutcome{4Short}{Diseña soluciones de Software de acuerdo a los estándares.\xspace}

%DefineOutcome{5}{Funcionar efectivamente como un miembro o líder de equipo involucrado en actividades apropiadas para la disciplina del programa.\xspace}
%DefineOutcome{5Short}{Funcionar efectivamente como un miembro o líder de equipo.\xspace}
\DefineOutcome{5}{Valora la necesidad del desarrollo profesional permanente y la capacidad para encararlo en el más amplio contexto de los cambios tecnológicos.\xspace}
\DefineOutcome{5Short}{Valora la necesidad aprendizaje permanente.\xspace}

\DefineOutcome{6}{Aplicar fundamentos de teoria de ciencias de la computación y desarrollo de software para producir soluciones basados en computación.\xspace}
\DefineOutcome{6Short}{Aplicar fundamentos de teoria de ciencias de la computación y desarrollo de software.\xspace}

\DefineOutcome{7}{Desarrollar principios investigación en el área de computación con niveles de competividad internacional.\xspace}
\DefineOutcome{7Short}{Desarrollar principios de investigación con nivel internacional.\xspace}

\DefineOutcome{8}{Transformar sus conocimientos del área de Ciencia de la Computación en emprendimientos tecnológicos.\xspace}
\DefineOutcome{8Short}{Transformar sus conocimientos en emprendimientos tecnológicos.\xspace}

\DefineOutcome{9}{Aplicar conocimientos de humanidades en su labor profesional.\xspace}
\DefineOutcome{9Short}{Aplicar conocimientos de humanidades en su labor profesional.\xspace}

% DefineOutcome{10}{Comprender que la formación de un buen profesional no se desliga ni se opone sino mas bien contribuye al auténtico crecimiento personal. Esto requiere de la asimilación de valores sólidos, horizontes espirituales amplios y una visión profunda del entorno cultural.\xspace}
% DefineOutcome{10Short}{Comprender que la formación humana contribuye al auténtico crecimiento personal.\xspace}

\DefineOutcome{10}{Mejorar las condiciones de la sociedad poniendo la tecnología al servicio del ser humano.\xspace}
\DefineOutcome{10Short}{Poner la tecnología al servicio del ser humano.\xspace}

\DefineOutcome{CG1}{Comunicación integral.\xspace}
\DefineOutcome{CG1Short}{Comunicación integral.\xspace}

\DefineOutcome{CG2}{Comunicación bilingüe.\xspace}
\DefineOutcome{CG2Short}{Comunicación bilingüe.\xspace}

\DefineOutcome{CG3}{Investigación.\xspace}
\DefineOutcome{CG3Short}{Investigación.\xspace}

\DefineOutcome{CG4}{Gestiona recursos.\xspace}
\DefineOutcome{CG4Short}{Gestiona recursos de manera eficiente.\xspace}

\DefineOutcome{CG5}{Desarrollo humano.\xspace}
\DefineOutcome{CG5Short}{Desarrollo humano.\xspace}

\DefineOutcome{CG6}{Desarrollo humano.\xspace}
\DefineOutcome{CG6Short}{Desarrollo humano.\xspace}