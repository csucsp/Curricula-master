\begin{syllabus}

\course{GH1002. Arte y Tecnología}{Obligatorio}{GH1002}
% Source file: ../Curricula.in/lang/Espanol/cycle/2020-I/Syllabi/GeneralEducation/GH1002.tex

\begin{justification}
El curso busca  dar  una visión  global, histórica  y crítica de las transformaciones y sinergias del arte contemporáneo. Donde los alumnos se aproximan a dos componentes del arte y diseño contemporáneo: las prácticas interdisciplinarias  y a los puntos de contacto entre las artes y los procesos tecnológicos y de ingeniería.
\end{justification}

\begin{goals}
    \item Desarrollar la capacidad de analizar información .
    \item Desarrollar la capacidad de interpretar información.
    \item Desarrollar la Capacidad de trabajo en equipo.
    \item Desarrollar la Capacidad de comunicación oral.
    \item Reconocer la necesidad del aprendizaje permanente.
\end{goals}

\begin{outcomes}{V1}
    \item \ShowOutcome{d}{2} % Multidiscip teams
    \item \ShowOutcome{e}{2} % ethical, legal, security and social implications
    \item \ShowOutcome{f}{2} % communicate effectively
    \item \ShowOutcome{n}{2} % Apply knowledge of the humanities
    \item \ShowOutcome{p}{2} % TASDSH
\end{outcomes}

\begin{competences}{V1}
    \item \ShowCompetence{C10}{d,n,o}
    \item \ShowCompetence{C17}{f}
    \item \ShowCompetence{C18}{f}
    \item \ShowCompetence{C21}{e}
\end{competences}

\begin{unit}{Arte y Técnologia}{}{Wilson02}{12}{4}
   \begin{topics}
      \item ?`Qué es el arte y para qué sirve?
      \item El discurso artístico: identidad, territorio, política y sociedad.
   \end{topics}
   \begin{learningoutcomes}
      \item Impulsar el interés por conocer sobre temas actuales de la sociedad peruana y del mundo.
   \end{learningoutcomes}
\end{unit}

\begin{unit}{Arte Digita}{}{Maeda12}{24}{3}
   \begin{topics}
      \item Arte Generativo.
      \item Net Art.
      \item Realidad Virtual.
   \end{topics}

   \begin{learningoutcomes}
      \item Desarrollo de habilidades como: creatividad, pensamiento crítico, observación y síntesis.
      \end{learningoutcomes}
\end{unit}

\begin{unit}{Prototipado, análisis y creación}{}{Wilson02}{24}{3}
   \begin{topics}
      \item Fabricación Digital.
      \item Intervención: Acción y espacio público.
      \item Presentación: Montaje, portafolio.
      \end{topics}

   \begin{learningoutcomes}
      \item Los alumnos entienden la importancia y efectividad del trabajo en equipo tanto en la vida académica como profesional. Durante el semestre los estudiantes realizan actividades grupales e individuales cuyo objetivo común es la generación de un
            proyecto que vincule conceptos de arte, tecnología e ingeniería.
     
   \end{learningoutcomes}
\end{unit}

\begin{coursebibliography}
\bibfile{GeneralEducation/GH1002}
\end{coursebibliography}
\end{syllabus}
