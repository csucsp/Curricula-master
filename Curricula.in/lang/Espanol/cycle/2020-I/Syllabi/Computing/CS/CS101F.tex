\begin{syllabus}

\course{CS101F. Introducción a la Programación}{Obligatorio}{CS101F}

\begin{justification}
Este es el primer curso en la secuencia de los cursos introductorios a la informática. En este curso se pretende cubrir los conceptos señalados por la \textit{Computing Curricula} IEEE-CS/ACM 2008, bajo el enfoque \textit{functional-first}.

La programación es uno de los pilares de la informática; cualquier profesional del área, necesitará programar para concretizar sus modelos y propuestas.

Este curso introducirá a los participantes en los conceptos fundamentales de este arte. Lo tópicos incluyen tipos de datos, estructuras de control, funciones, listas, recursividad y la mecánica de la ejecución, prueba y depuración.

El curso también ofrecerá una introducción al contexto histórico y social de la informática y una revisión del ámbito de esta disciplina.
\end{justification}

\begin{goals}
\item Introducir los conceptos fundamentales de programación y estructuras de datos utilizando un lenguaje funcional.
\item Desarrollar su capacidad de abstracción, utilizar un lenguaje de programación funcional.
\end{goals}




\begin{unit}{\SPHistoryOfComputingDef}{}{brookshear,thompson2011,guttag2013,zelle2010}{4}{a}
    \SPHistoryOfComputingAllTopics
    \SPHistoryOfComputingAllObjectives
\end{unit}

\begin{unit}{\PLOverviewDef}{}{thompson2011,guttag2013,zelle2010}{1}{a}
   \begin{topics}
      \item \PLOverviewTopicHistory
      \item Paradigmas de programación.
   \end{topics}

   \begin{learningoutcomes}
      \item \PLOverviewObjONE
      \item \PLOverviewObjTWO
   \end{learningoutcomes}
\end{unit}

\begin{unit}{\PLDeclarationsAndTypesDef}{}{thompson2011,guttag2013,zelle2010}{1}{a}
    \begin{topics}%
	\item \PLDeclarationsAndTypesTopicThe%
	\item \PLDeclarationsAndTypesTopicOverview%
    \end{topics}%
    \PLDeclarationsAndTypesAllObjectives
\end{unit}

\begin{unit}{\PFFundamentalConstructsDef}{}{thompson2011,guttag2013,zelle2010}{2}{a,b}
  \PFFundamentalConstructsAllTopics
  \PFFundamentalConstructsAllObjectives
\end{unit}

\begin{unit}{\PLFunctionalProgrammingDef}{}{thompson2011,guttag2013,zelle2010}{1}{a,b}
    \begin{topics}%
	\item \PLFunctionalProgrammingTopicOverview%
	\item \PLFunctionalProgrammingTopicRecursion%
	\item \PLFunctionalProgrammingTopicPragmatics%
    \end{topics}%
   \PLFunctionalProgrammingAllObjectives
\end{unit}

\begin{unit}{\PFRecursionDef}{}{thompson2011,guttag2013,zelle2010}{6}{a,b}
    \begin{topics}%
	\item \PFRecursionTopicTheconcept%
	\item \PFRecursionTopicRecursive%
	\item \PFRecursionTopicSimple%
	\item \PFRecursionTopicDiveAndConquer%
    \end{topics}%

    \begin{learningoutcomes}%
	\item \PFRecursionObjONE%
	\item \PFRecursionObjTWO%
	\item \PFRecursionObjTHREE%
	\item \PFRecursionObjFOUR%
	\item \PFRecursionObjFIVE%
	\item \PFRecursionObjSIX%
	\item \PFRecursionObjEIGHT%
    \end{learningoutcomes}%
\end{unit}

\begin{unit}{\ALFundamentalAlgorithmsDef}{}{thompson2011,guttag2013,zelle2010}{4}{a,b,c}
    \begin{topics}%
	\item \ALFundamentalAlgorithmsTopicSimple%
	\item \ALFundamentalAlgorithmsTopicSequential%
	\item \ALFundamentalAlgorithmsTopicQuadratic%
	\item \ALFundamentalAlgorithmsTopicBinary%
	\item \ALFundamentalAlgorithmsTopicDepth%
    \end{topics}%
    \ALFundamentalAlgorithmsAllObjectives
\end{unit}

\begin{unit}{\PLAbstractionMechanismsDef}{}{thompson2011,guttag2013,zelle2010}{4}{a,b,c}
   \begin{topics}
      \item \PLAbstractionMechanismsTopicProcedures%
      \item \PLAbstractionMechanismsTopicParameterization%
      \item \PLAbstractionMechanismsTopicType%
      \item \PLAbstractionMechanismsTopicModules%
   \end{topics}
   \begin{learningoutcomes}
      \item \PLAbstractionMechanismsObjONE%
      \item \PLAbstractionMechanismsObjTWO%
      \item \PLAbstractionMechanismsObjTHREE%
   \end{learningoutcomes}
\end{unit}

\begin{unit}{\PFAlgorithmsAndProblemSolvingDef}{}{thompson2011,guttag2013,zelle2010}{10}{a,b,c,i}
    \PFAlgorithmsAndProblemSolvingAllTopics
    \PFAlgorithmsAndProblemSolvingAllObjectives
\end{unit}

\begin{unit}{\PLVirtualMachinesDef}{}{thompson2011,guttag2013,zelle2010}{1}{2}
   \begin{topics}
      \item \PLVirtualMachinesTopicETheconcept
   \end{topics}
   \begin{learningoutcomes}
      \item \PLVirtualMachinesObjONE
   \end{learningoutcomes}
\end{unit}

\begin{unit}{\PLObjectOrientedProgrammingDef}{}{thompson2011,guttag2013,zelle2010}{4}{a,b,c,i}
    \begin{topics}%
	\item \PLObjectOrientedProgrammingTopicClasses%
	\item \PLObjectOrientedProgrammingTopicPolymorphism%
	\item \PLObjectOrientedProgrammingTopicClasshierarchies%
    \end{topics}%

    \begin{learningoutcomes}%
	\item \PLObjectOrientedProgrammingObjFOUR%
	\item \PLObjectOrientedProgrammingObjFIVE%
    \end{learningoutcomes}%
\end{unit}

\begin{unit}{\SEUsingAPIsDef}{}{thompson2011,guttag2013,zelle2010}{2}{a,b,c,i}
   \begin{topics}
      \item \SEUsingAPIsTopicProgramming
   \end{topics}

   \begin{learningoutcomes}
      \item \SEUsingAPIsObjONE
   \end{learningoutcomes}
\end{unit}

\begin{coursebibliography}
\bibfile{Computing/CS/CS101F}
\end{coursebibliography}

\end{syllabus}
