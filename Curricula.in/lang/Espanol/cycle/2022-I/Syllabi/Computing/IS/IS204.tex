\begin{syllabus}

\course{IT2004. Hardware y Software de TI}{Obligatorio}{IT2004}
% Source file: ../Curricula.in/lang/Espanol/cycle/2020-I/Syllabi/Computing/IS/IS204.tex

\begin{justification}
    Esta disciplina es importante porque introduce al estudiante a los principios y aplicación de software y hardware de computadores con una experiencia práctica de instalación y configuración.
    \end{justification}
    
    \begin{goals}
    \item Que el alumno entienda los puntos de equilibrio en una arquitectura de computadores para su uso efectivo en un entorno organizacional.
    \item Que el alumno entienda los conceptos fundamentales de Arquitectura de Computadores, memoria, registros, buses, entornos multiprocesadores vs. aquellos con un solo procesador, dispositivos de almacenamiento, controladores de dispositivos, 
    \item Que el alumno conozca los módulos del sistema operativo, procesos y gerenciamiento de procesos en sistema de archivos.
    \end{goals}
    
    \begin{outcomes}{V1}
        \item \ShowOutcome{b}{1}
        \item \ShowOutcome{c}{1}
        \item \ShowOutcome{g}{1}
        \item \ShowOutcome{i}{1}
    \end{outcomes}
    
    \begin{unit}{\LUSIXDef}{}{\LUSIXBib}{6}{b,c,g,i}
    
    \end{unit}
    
    \begin{unit}{\LUSEVENDef}{}{\LUSEVENBib}{3}{b,c,g,i}
    
    \end{unit}
    
    
    \begin{coursebibliography}
    \bibfile{Computing/IS/IS}
    \end{coursebibliography}
    
    \end{syllabus}
    