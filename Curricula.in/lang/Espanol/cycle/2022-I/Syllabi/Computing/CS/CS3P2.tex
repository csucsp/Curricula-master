\begin{syllabus}

\course{CS3P02. Cloud Computing}{Obligatorio}{CS3P02}
% Source file: ../Curricula.in/lang/Espanol/cycle/2020-I/Syllabi/Computing/CS/CS3P2.tex

\begin{justification}
Para entender las técnicas computacionales avanzadas, los estudiantes deberán tener un fuerte conocimiento de las
diversas estructuras discretas, estructuras que serán implementadas y usadas en laboratorio en el lenguaje de programación.
\end{justification}

\begin{goals}
\item Que el alumno sea capaz de modelar problemas de ciencia de la computación usando grafos y árboles relacionados con estructuras de datos.
\item Que el alumno aplicar eficientemente estrategias de recorrido para poder buscar datos de una manera óptima.
\end{goals}



\begin{unit}{\PDDistributedSystems}{}{coulouris}{15}{a,b}
\begin{topics}%
    \item \PDDistributedSystemsTopicFaults
    \item \PDDistributedSystemsTopicDistributed
    \item \PDDistributedSystemsTopicDistributedSystem
    \item \PDDistributedSystemsTopicDistributedService
    \item \PDDistributedSystemsTopicCore
\end{topics}
\begin{learningoutcomes}%
    \item \PDDistributedSystemsLODistinguishNetwork~[\Familiarity] %
    \item \PDDistributedSystemsLOExplainWhySuch~[\Familiarity] %
    \item \PDDistributedSystemsLOWriteAPerforms~[\Usage] %
    \item \PDDistributedSystemsLOMeasure~[\Usage] %
    \item \PDDistributedSystemsLOExplainWhySystem~[\Familiarity] %
    \item \PDDistributedSystemsLOImplementAForSpell~[\Usage] %
    \item \PDDistributedSystemsLOExplainTheOverhead~[\Familiarity] %
    \item \PDDistributedSystemsLODescribeTheAssociated~[\Familiarity] %
    \item \PDDistributedSystemsLOGiveExamplesFor~[\Usage] %
\end{learningoutcomes}%
\end{unit}

\begin{unit}{\PDCloudComputing}{}{dongarra, buyya}{15}{a,b}
\begin{topics}
    \item Visión global de \textit{Cloud Computing}.
    \item Historia.
    \item Visión global de las tecnologias que envuelve.
    \item Beneficios, riesgos y aspectos económicos.
    \item \PDCloudComputingTopicCloud
    \item \PDCloudComputingTopicInternet
\end{topics}
\begin{learningoutcomes}
    \item Explicar el concepto de Cloud Computing. [\Familiarity]
    \item Listar algunas tecnologias relacionadas con Cloud Computing. [\Familiarity]
    \item \PDCloudComputingLOExplainStrategies~[\Familiarity] %
    \item Discutir las ventajas y desventajas del paradigma de Cloud Computing.  [\Familiarity]
    \item Expresar los beneficios económicos así como las carácteristicas y riesgos del paradigma de Cloud para negocios y proveedores de cloud.   [\Familiarity]
    \item Diferenciar entre los modelos de servicio.   [\Usage]
\end{learningoutcomes}
\end{unit}

\begin{unit}{Centros de Procesamiento de Datos}{}{dongarra, buyya}{10}{g,i}
\begin{topics}
    \item Visión global de un centro de procesamiento de datos.
    \item Consideraciones en el diseño.
    \item Comparación de actuales grandes centros de procesamiento de datos.
\end{topics}
\begin{learningoutcomes}
    \item Describir la evolución de los Data Centers. [\Familiarity]
    \item Esbozar la arquitectura de un data center en detalle. [\Familiarity]
    \item Indicar consideraciones de diseño y discutir su impacto.  [\Familiarity]
\end{learningoutcomes}
\end{unit}

\begin{unit}{\PDCloudComputing}{}{dongarra, buyya}{20}{i,j}
\begin{topics}
    \item \PDCloudComputingTopicVirtualization
    \item Seguridad, recursos y isolamiento de fallas.
    \item Almacenamiento como servicio.
    \item Elasticidad.
    \item Xen y WMware.
    \item Amazon EC2.
\end{topics}
\begin{learningoutcomes}
    \item \PDCloudComputingTopicVirtualization. [\Familiarity]
    \item \PDCloudComputingLOExplainTheDisadvantages. [\Familiarity]
    \item Identificar las razones por qué la virtualización está llegando a ser enormente útil, especialmente en la cloud. [\Familiarity]
    \item Explicar diferentes tipos de isolamiento como falla, recursos y seguridad proporcionados por la virtualización y utilizado por la cloud. [\Familiarity]
    \item Explicar la complejidad que puede tener el administrar en términos de niveles de abstracción y interfaces bien definidas y su aplicabilidad para la virtualización en la cloud.  [\Familiarity]
    \item Definir virtualización y identificar diferentes tipos de máquinas virtuales. [\Familiarity]
    \item Identificar condiciones de virtualización de CPU, reconocer la diferencia entre {\em full virtualization} y {\em paravirtualization}, explicar emulación como mayor técnica para virtualización del CPU y examinar planificación virtual del CPU en Xen. [\Familiarity]
    \item Esbozar la diferencia entre la clásica memoria virtual del SO y la virtualización de memoria. Explicar los múltiplos niveles de mapeamiento de páginas en oposición a la virtualización de la memoria. Definir memoria {\em over-commitment} e ilustrar sobre WMware {\em memory ballooning} como técnica de reclamo para sistemas virtualizados con memoria {\em over-committed}. [\Familiarity]
\end{learningoutcomes}
\end{unit}

\begin{unit}{\PDCloudComputing}{}{dongarra, buyya}{12}{i,j}
\begin{topics}
    \item \PDCloudComputingTopicCloudBased
    \item Visión global sobre tecnologías de almacenamiento.
    \item Conceptos fundamentales sobre almacenamiento en la cloud.
    \item Amazon S3 y EBS.
    \item Sistema de archivos distribuidos.
    \item Sistema de bases de datos NoSQL.
\end{topics}
\begin{learningoutcomes}
    \item Describir la organización general de datos y almacenamiento. [\Familiarity]
    \item Identificar los problemas de escalabilidad y administración de la big data. Discutir varias abstracciones en almacenamiento. [\Familiarity]
    \item Comparar y contrastar diferentes tipos de sistema de archivos. Comparar y contrastar el Sistema de Archivos Distribuido de Hadoop (HDFS) y el Sistema de Archivos Paralelo Virtual (PVFS).  [\Usage]
    \item Comparar y contrastar diferentes tipos de bases de datos. Discutir las ventajas y desventajas sobre las bases de datos NoSQL. [\Usage]
    \item Discutir los conceptos de almacenamiento en la cloud. [\Familiarity]
\end{learningoutcomes}
\end{unit}

\begin{unit}{Modelos de Programación}{}{dongarra, buyya, graphlab, pregel, giraph}{12}{g,j}
\begin{topics}
    \item Visión global de los modelso de programación basados en cloud computing.
    \item Modelo de Programación MapReduce.
    \item Modelo de programación para aplicaciones basadas en Grafos.
\end{topics}
\begin{learningoutcomes}
    \item Explicar los aspectos fundamentales de los modelos de programación paralela y distribuida. [\Familiarity]
    \item Diferencias entre los modelos de programación: MapReduce, Pregel, GraphLab y Giraph. [\Usage]
    \item Explicar los principales conceptos en el modelo de programación MapReduce.  [\Usage]
\end{learningoutcomes}
\end{unit}

\begin{coursebibliography}
\bibfile{Computing/CS/CS3P2}
\end{coursebibliography}

\end{syllabus}
