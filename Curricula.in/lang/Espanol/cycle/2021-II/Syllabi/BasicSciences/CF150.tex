\begin{syllabus}

\course{CF150. Fundamentos de Física}{Obligatorio}{CF150} % Common.pm

\begin{justification}
El curso está orientado principalmente al tratamiento conceptual de la mecánica de la part'../../../2020-I copy/Syllabi/BasicSciences'ícula. Para este fin, se desarrollan cuatro temas básicos: entre ellos, lo concerniente a las cantidades físicas, el Sistema Internacional de Unidades, las magnitudes escalares y vectoriales; Cinemática, en la cual se incluyen temas de movimiento rectilíneo, casos de velocidad y de aceleración constantes, y su representación gráfica en el tiempo, cuerpos en caída libre y el movimiento circular uniforme; Dinámica, que comprende las tres leyes de Newton, los conceptos de masa y peso, y el tema de rozamiento; y los conceptos básicos de Trabajo y Energía, con alguna de sus aplicaciones.
\end{justification}

\begin{goals}
\item Al término del semestre, el estudiante será capaz de resolver problemas de vectores en dos dimensiones por métodos gráficos o analíticos. Será capaz de utilizar los sistemas de unidades internacional (SI) e Inglés. El alumno será capaz de utilizar magnitudes físicas escalares y vectoriales, para ello, emplear operaciones de dos ecuaciones con dos incógnitas y dar su respuesta en notación científica.
\item En cinemática de la part'../../../2020-I copy/Syllabi/BasicSciences'ícula, el alumno será capaz de describir el movimiento en una dimensión de la part'../../../2020-I copy/Syllabi/BasicSciences'ícula con velocidad constante y con aceleración constante. Para ello, el alumno será capaz de emplear las ecuaciones de movimiento y de velocidad, así como, las gráficas de posición-tiempo y velocidad-tiempo. Aplicar los conceptos de distancia, velocidad, rapidez y aceleración y utilizar adecuadamente el sistema de referencia. Utilizar los conceptos de cinemática para la solución de problemas de dos y tres móviles, con el uso las ecuaciones de movimiento y velocidad y gráficas velocidad-tiempo y posición-tiempo.
\item Al término del semestre el alumno será capaz de emplear los conceptos de derivada e integral definida para su uso en la determinación de las ecuaciones de movimiento, de velocidad y de aceleración. Será capaz de usar la fórmula general para la derivada y para la integral de una función de la forma atn, en ejercicios y problemas aplicados a cinemática de la part'../../../2020-I copy/Syllabi/BasicSciences'ícula en una dimensión.
\item En dinámica de la part'../../../2020-I copy/Syllabi/BasicSciences'ícula, el alumno será capaz de aplicar las leyes de Newton, los conceptos de masa y peso. Utilizar las unidades en el Sistema Internacional. Comprender la ley de gravitación universal. Emplear el diagrama de cuerpo libre, reconocer correctamente las fuerzas normales y tensiones, las fuerzas de acción y reacción. Equilibrio de una part'../../../2020-I copy/Syllabi/BasicSciences'ícula. Resolver problemas de dinámica con planos inclinados, poleas y cuerdas ideales. El alumno será capaz de resolver ejemplos y problemas sin fricción, de uno, dos y tres bloques unidos por cuerdas ideales que pasan por poleas ideales. Al término del semestre, el alumno será capaz de reconocer y emplear las fuerzas de fricción, fricción estática y cinética (tendencia al movimiento y movimiento relativo). Fricción estática máxima, coeficientes de rozamiento estático y cinético, en experimentos, ejemplos y problemas con fricción: de un bloque sobre una superficie horizontal y en plano inclinado, dos y tres bloques unidos por cuerdas ideales y que pasan por poleas ideales.
\end{goals}

\begin{outcomes}
\item \ShowOutcome{a}{2}
\item \ShowOutcome{i}{2}
\end{outcomes}

\begin{competences}
    \item \ShowCompetence{C1,C20}{a,i}
\end{competences}

\begin{unit}{Cantidades físicas y vectores en una y dos dimensiones}{}{Douglas2006,Hewitt2004,Wilson2003}{6}{C1,C20}
\begin{topics}
      \item Cantidades físicas. Magnitudes fundamentales y derivadas. Magnitudes derivadas adimensionales: radian. Prefijos de múltiplos y submúltiplos. Sistema internacional, sistema inglés y técnico. Magnitudes escalares y vectoriales. Conversión de unidades. Homogeneidad dimensional. Ejemplos y problemas.
      \item Definición de vector. Vectores en dos dimensiones. Adición de vectores en dos dimensiones por métodos gráficos: polígono y paralelogramo, métodos analíticos: descomposición rectangular, ley de senos y ley de cosenos. Definición de vector opuesto. Resta de vectores en dos dimensiones. Preguntas conceptuales, ejemplos y problemas.
   \end{topics}

\end{unit}

\begin{unit}{Cinemática de la partícula en una dimensión}{}{Douglas2006,Hewitt2004,Wilson2003}{14}{C1,C20}
\begin{topics}
      \item Sistema de referencia: eje x, origen de coordenadas e instante inicial de tiempo. Vector posición (una componente), vector desplazamiento, distancia recorrida, vector velocidad media y rapidez. Preguntas conceptuales.
      \item Movimiento rectilíneo con velocidad constante. Velocidad media. Definición de velocidad constante a partir de la velocidad media. Gráfica velocidad-tiempo (v-t). Análisis la velocidad y del desplazamiento en la gráfica v-t. Determinación de la posición como una función del tiempo: ley de movimiento, x(t). Gráfica posición-tiempo (x-t). Análisis de la posición, desplazamiento y velocidad en la gráfica x-t. Preguntas conceptuales, ejemplos y problemas.
      \item Movimiento rectilíneo con aceleración constante. Aceleración media. Definición de aceleración constante a partir de la aceleración media. Gráfica aceleración-tiempo (a-t). Análisis de la aceleración y variación de la velocidad en la gráfica a-t. Determinación de la velocidad como una función del tiempo: ley de velocidad, v(t). Gráfica v-t. Análisis del desplazamiento, la velocidad y la aceleración en la gráfica v-t. Determinación de la posición como una función del tiempo: ley de movimiento, x(t). Gráfica x-t. Análisis del desplazamiento, la velocidad y la aceleración en la gráfica x-t. Determinación de la ecuación eliminando la aceleración y determinación de la ecuación eliminando el tiempo, a partir de las leyes de movimiento y de velocidad. Preguntas conceptuales, ejemplos y problemas.
      \item Caída libre. Convención: sistema de referencia con el eje positivo siempre hacia arriba (eje y), la gravedad: aceleración de valor 9,8 m/s2 siempre negativa. La velocidad como función del tiempo: ley de velocidad, v(t), y posición como función del tiempo: ley de movimiento, y(t). Ecuaciones eliminando la aceleración y eliminando el tiempo, a partir de las leyes de movimiento y de velocidad. Características del movimiento: el tiempo de subida y el tiempo de bajada, respecto a un punto. Rapidez de subida y de bajada, respecto a un punto. Gráficas de aceleración-tiempo (a-t), de velocidad-tiempo (v-t) y de posición-tiempo (y-t). Preguntas conceptuales, ejemplos y problemas.
    \end{topics}
\end{unit}

\begin{unit}{Conceptos de derivada e integral definida}{}{Douglas2006,Hewitt2004,Wilson2003}{6}{C1,C20}
\begin{topics}
      \item Concepto de derivada. Ejercicios y problemas aplicados a cinemática en una dimensión. Fórmula general para la derivada de una función de la forma $at^{n}$. Concepto de integral definida. Ejercicios y problemas aplicados a cinemática en una dimensión. Fórmula general para la integral de una función de la forma $at^{n}$.
\end{topics}

\end{unit}

\begin{unit}{Dinámica de la partícula en una y dos dimensiones}{}{Douglas2006,Hewitt2004,Wilson2003}{16}{C1,C20}
\begin{topics}
      \item Leyes de Newton. Concepto de masa y peso. Unidades en el Sistema Internacional. Ley de gravitación universal. Fuerzas de acción y reacción. Fuerzas normales y tensiones. Diagrama de cuerpo libre y sistema de referencia. Equilibrio de una part'../../../2020-I copy/Syllabi/BasicSciences'ícula. Plano inclinado poleas y cuerdas ideales. Preguntas conceptuales, ejemplos y problemas sin fricción, de uno, dos y tres bloques unidos por cuerdas que pasan por poleas.
      \item Rozamiento (fricción). Tendencia y movimiento relativo: fricción estática y cinética. Fricción estática máxima, coeficientes de rozamiento estático y cinético. Relación entre la fuerza aplicada y la fricción estática y cinética a través de una gráfica. Experimentos. Preguntas conceptuales, ejemplos y problemas con fricción: de un bloque sobre una superficie horizontal y en plano inclinado, dos y tres bloques unidos por cuerdas y que pasan por poleas ideales.
   \end{topics}
\end{unit}



\begin{coursebibliography}
\bibfile{BasicSciences/CF150}
\end{coursebibliography}

\end{syllabus}
