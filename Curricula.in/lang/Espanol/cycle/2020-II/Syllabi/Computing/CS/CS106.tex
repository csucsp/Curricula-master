\begin{syllabus}

\course{CB1006. Estructuras Discretas II}{Obligatorio}{CB1006}
% Source file: ../Curricula.in/lang/Espanol/cycle/2020-I/Syllabi/Computing/CS/CS106.tex

\begin{justification}
   Para entender las técnicas computacionales avanzadas, los estudiantes deberán tener un fuerte conocimiento de las
   diversas estructuras discretas, estructuras que serán implementadas y usadas en laboratorio en el lenguaje de programación.
   \end{justification}
   
   \begin{goals}
   \item Que el alumno sea capaz de modelar problemas de ciencia de la computación usando grafos y árboles relacionados con estructuras de datos
   \item Que el alumno aplicar eficientemente estrategias de recorrido para poder buscar datos de una manera óptima
   \end{goals}
   
   \begin{outcomes}{V1}
      \item \ShowOutcome{a}{3}
      \item \ShowOutcome{b}{4}
      \item \ShowOutcome{i}{3}
      \item \ShowOutcome{j}{3}
    \end{outcomes}
   
   \begin{unit}{\DSBasicsofCounting}{}{Grimaldi97}{25}{a}
      \DSBasicsofCountingAllTopics
      \DSBasicsofCountingAllLearningOutcomes
   \end{unit}
   
   \begin{unit}{\DSGraphsandTrees}{}{Johnsonbaugh99}{25}{a,b}
      \DSGraphsandTreesAllTopics
      \DSGraphsandTreesAllLearningOutcomes
   \end{unit}
   
   \begin{unit}{\DSDiscreteProbability}{}{Micha98,Rosen2004}{10}{a,b,j}
      \DSDiscreteProbabilityAllTopics
      \DSDiscreteProbabilityAllLearningOutcomes
   \end{unit}
   
   \begin{coursebibliography}
   \bibfile{Computing/CS/CS105}
   \end{coursebibliography}
   
   \end{syllabus}
   
   %\end{document}
   
