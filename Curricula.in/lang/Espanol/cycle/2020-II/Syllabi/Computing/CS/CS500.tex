
\begin{syllabus}

\course{CS5000. Simulación y Diseño de Experimentos}{Obligatorio}{CS5000}
% Source file: ../Curricula.in/lang/Espanol/cycle/2020-I/Syllabi/Computing/CS/CS500.tex

\begin{justification}
Este curso presenta a los estudiantes conceptos y técnicas de diseño clásico y bayesiano: unidades experimentales, aleatorización, tratamientos, bloqueos y restricciones a la aleatorización, y la utilidad de los diseños. Muestra  la determinación óptima del tamaño de la muestra para la estimación y las pruebas. Los temas incluyen pruebas A-B simples, diseños factoriales factoriales y factoriales fraccionados, métodos de superficie de respuesta, diseños conjuntos, diseños secuenciales, diseño y modelado de experimentos informáticos complejos, y diseños para objetivos múltiples. 
\end{justification}

\begin{goals}
\item Capacidad.
\end{goals}

\begin{outcomes}{V1}
   \item \ShowOutcome{a}{2}
\end{outcomes}

\begin{competences}{V1}
    \item \ShowCompetence{C1}{a} 
\end{competences}


\begin{unit}{Introducción}{}{Sheldon,Menden}{6}{a}
\begin{topics}
      \item Introducción I.
   \end{topics}

   \begin{learningoutcomes}
      \item Aprender
   \end{learningoutcomes}
\end{unit}





\begin{coursebibliography}
\bibfile{BasicSciences/CS500}
\end{coursebibliography}

\end{syllabus}
