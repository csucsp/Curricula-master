\begin{syllabus}

\course{CS401. Metodología de la Investigación en Computación}{Obligatorio}{CS401}
% Source file: ../Curricula.in/lang/Espanol/cycle/2020-I/Syllabi/Computing/CS/CS401.tex

\begin{justification}
Este curso tiene por objetivo que el alumno aprenda a realizar una
investigación de carácter científico en el área de computación. Los docentes del curso determinarán un área de estudio para cada alumno, y se le hará entrega de bibliografía para analizar y a partir de la misma, y de fuentes bibliográficas adicionales (investigadas por el alumno), el alumno deberá ser capaz de construir un artículo del tipo survey del tema asignado.
\end{justification}

\begin{goals}
\item Que el alumno aprenda como se inicia una investigación científica en el área de computación.
\item Que el alumno conozca las principales fuentes para obtener bibliografía relevante para trabajos de investigación en el área de computacion: Researchindex, IEEE-CS\footnote{http://www.computer.org}, ACM\footnote{http://www.acm.org}.
\item Que el alumno sea capaz de analizar las propuestas existentes sobre un determinado tópico y relacionarlos de forma coherente en una revisión bibliográfica.
\item Que el alumno pueda redactar documentos técnicos en computación utilizando \LaTeX.
\item Que el alumno sea capaz de \underline{reproducir} los resultados ya existentes en un determinado tópico a través de la experimentación.
\item Los entregables de este curso son:
	\begin{description}
	\item [Avance parcial:] Dominio del tema del artículo y bibliografía preliminar en formato de artículo \LaTeX.
	\item [Final:] Entendimiento del artículo del tipo survey, documento concluído donde se contenga, opcionalmente, 
	      los resultados experimentales de la(s) técnica(s) estudiada(s).
	\end{description}
\end{goals}



\begin{unit}{Iniciación científica en el área de computación}{}{ieee,acm,citeseer}{60}{a,b,c,i,h}
  \begin{topics}
      \item Búsqueda bibliográfica en computación.
      \item Redacción de artículos técnicos en computación.
  \end{topics}
  \begin{learningoutcomes}
      \item Aprender a hacer una investigación correcta en el área de computación[\Usage]
      \item Conocer las fuentes de bibliografía adecuada para esta área[\Usage]
      \item Saber redactar un documento de acorde con las carácteristicas que las conferencias de esta área exigen[\Usage]
  \end{learningoutcomes}
\end{unit}

\begin{coursebibliography}
\bibfile{Computing/CS/CS401}
\end{coursebibliography}

\end{syllabus}
