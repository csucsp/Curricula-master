\begin{syllabus}

\course{. }{}{} % Common.pm

\begin{justification}
	El propósito de este curso es proporcionar al alumno una base histórica 
	que le permita comprender críticamente los principales procesos culturales 
	que han configurado la Civilización Occidental desde sus orígenes remotos en 
	las civilizaciones antiguas principalmente en el mundo clásico grecorromano 
	introduciéndonos en la Edad Media mediante la formación, desarrollo y crisis 
	de la cristiandad, analizando los principales hechos que han configurado la 
	historia moderna de Occidente, como el Renacimiento, el Protestantismo, 
	la Ilustración, la Revolución Industrial, la emergencia de las utopías sociales 
	y el humanismo ateo, o la Postmodernidad.
	
	La asignatura asume que el mundo occidental constituye históricamente una 
	unidad cultural con sus diferencias continentales y particularidades nacionales, 
	regionales y locales.
	
	Como se ve, no se trata de dar a conocer los principales hechos históricos de la 
	Historia Universal, sino fundamentalmente de introducir al estudiante en una 
	visión crítica general de los procesos culturales que generan nuevas mentalidades, 
	conceptos, modos de ver la vida, costumbres, políticas y normas. 
	Por lo dicho, tampoco nos referimos a la nueva historia cultural moderna que, 
	historiográficamente hablando, se ocupa solo del tratamiento, producción y 
	socialización de servicios y bienes culturales.
	
	A lo largo del curso serán fundamentales dos textos que tomaremos como referencia. 
	El primero será el libro de Florencio Hubeñak, Historia integral de Occidente (2008) 
	y el segundo de Alfredo Sáenz, La cristiandad y su cosmovisión (1992). 
	Ambos textos servirán de guía para comprender los principales cambios culturales 
	de fondo y los acontecimientos políticos, sociales y económicos más importantes.
	La finalidad es que los alumnos puedan obtener una base de información fáctica 
	relevante de las lecturas y que las clases sean lugares no solo de exposición 
	magistral sino de diálogo que permita consolidar su aprendizaje.
	
	El curso ofrecerá una visión panorámica de la formación y desarrollo de 
	Occidente, desde sus orígenes en el mundo clásico grecorromano, la posterior 
	influencia sociocultural que tuvo el cristianismo, poniendo énfasis en la 
	conformación de la Cristiandad como estructura fundamental política 
	económica y social, analizando el fenómeno cultural de la Modernidad, 
	sus orígenes, formación, desarrollo y crisis hasta la Postmodernidad.  
	Entendiendo el término cultura 
	en el sentido amplio de civilización, el curso ofrece una perspectiva de las 
	principales inflexiones culturales en Occidente.
	Teniendo en cuenta la conjunción de factores políticos, sociales y económicos, 
	se buscará abordar los principales cambios de perspectiva y mentalidad respecto 
	de los problemas filosófico-teológicos fundamentales, los ordenamientos 
	políticos y normativos, así como los presupuestos morales, costumbres y 
	valores predominantes de cada periodo histórico.
\end{justification}

\begin{goals}
\item Analizar y comprender  los principales procesos históricos por los que ha atravesado el mundo de Occidente, dotando al alumno de una información real y equilibrada de los acontecimientos más relevantes que han ido conformando nuestro mundo actual a fin de criticar y tomar conciencia sobre los mismos y sobre la influencia que han tenido en la sociedad así como en la política y economía a través del tiempo. [\Familiarity]
\end{goals}

\begin{outcomes}
    \item \ShowOutcome{n}{2}
	\item \ShowOutcome{o}{2}
\end{outcomes}

\begin{competences}
    \item \ShowCompetence{C20}{n,ñ}
\end{competences}

\begin{unit}{}{Mundo grecorromano y el cristianismo}{Hubenak2007,Dawson2007,Krebs2006}{12}{C20}
\begin{topics}
	\item Nociones fundamentales de Historia de la Cultura Occidental.
	     \begin{subtopics}
		\item Concepto.
		\item Importancia.
		\item Cuestiones metodológicas básicas.
		\item Alcances y límites.
	    \end{subtopics}
	\item Civilización Helénica.
	    \begin{subtopics}
		\item Evolución de la polis griega.
		\item Apogeo y legado cultural helénico.
		\item Cosmovisión y paradigmas culturales.
		\item El helenismo.
	    \end{subtopics}
	\item Civilización Romana.
	    \begin{subtopics}
		\item Importancia.
		\item La cosmovisión del romano.
		\item Helenización y Romanización.
		\item Consolidación del Imperio.
	    \end{subtopics}
	\item Cristianismo en el Imperio Romano.
	    \begin{subtopics}
		\item La revolución cultural del cristianismo.
		\item De la Romanidad a la Cristiandad.
	    \end{subtopics}
\end{topics}
\begin{learningoutcomes}
	\item Comprender la importancia del estudio de la historia como parte de la formación integral del estudiante universitario. [\Familiarity]
	\item Identificar el aporte de las civilizaciones antiguas en la formación del mundo occidental. [\Familiarity]
	\item Analizar el aporte del cristianismo en la configuración de la civilización occidental, el proceso de expansión y consolidación en el marco de la unidad política y cultural grecorromana.[\Familiarity]
\end{learningoutcomes}
\end{unit}

\begin{unit}{}{Formación y desarrollo de la Cristiandad}{Hubenak2007,Garcia1960,saenz1992}{9}{C20}
\begin{topics}
	\item Labor civilizadora de la Iglesia.
	    \begin{subtopics}
		\item Los reinos romano-germánicos.
		\item La romanización de los bárbaros y la cristianización de la Romanidad.
		\item El monacato occidental.
		\item La legislación romano-germánica.
	    \end{subtopics}
	\item El Imperio Bizantino y el Islam.
	    \begin{subtopics}
		\item Origen, importancia y etapas del Imperio Bizantino.
		\item Cisma de Oriente.
		\item Conceptos fundamentales de la religión musulmana.
		\item Expansión y amenaza a la Cristiandad.
	    \end{subtopics}
	\item Iglesia y Estado en la Edad Media.
	\item La Cristiandad y su cosmovisión.
\end{topics}
\begin{learningoutcomes}
	\item Identificar los procesos históricos de la formación de la Cristiandad, el desarrollo de la cristiandad oriental, el sistema feudal y la estructura política que lo sustentaba. [\Familiarity]
\end{learningoutcomes}
\end{unit}

\begin{unit}{}{Absolutismo e Ilustración}{Hubenak2007}{9}{C20}
\begin{topics}
	\item Concepto, origen, desarrollo y crisis históricos de la Modernidad.
	\item Cambios sociales, económicos y culturales en la Cristiandad
	      \begin{subtopics}
		\item Legado medieval.
		\item Eclosión del mundo moderno.
	      \end{subtopics}
	\item Renacimiento.
	\item Protestantismo.
	\item Reforma católica.
	\item Monarquías nacionales.
	\item La paz de Westfalia.
	\item La Ilustración.
	\item Fin del Antiguo Régimen.
\end{topics}
\begin{learningoutcomes}
	\item Identificar y analizar los procesos históricos que dan origen y configuran la Modernidad. [\Familiarity]
\end{learningoutcomes}
\end{unit}

\begin{unit}{}{Siglo XIX y XX}{Hubenak2007}{9}{C20}
\begin{topics}
	\item Consolidación del Estado-nación.
	\item Revolución Industrial y pensamiento científico.
	\item Eras de las utopías sociales.
	      \begin{subtopics}
		\item Comunismo.
		\item Fascismo.
		\item Nazismo.
	      \end{subtopics}
	\item La guerra fría y la postmodernidad.
\end{topics}
\begin{learningoutcomes}
	\item Analizar los procesos históricos asociados a los radicales cambios ocurridos durante el siglo XIX y consecuente siglo XX. [\Familiarity]
\end{learningoutcomes}
\end{unit}

\begin{unit}{}{Siglo XIX y XX}{Hubenak2007}{9}{C20}
\begin{topics}
	\item Consolidación del Estado-nación.
	\item Revolución Industrial y pensamiento científico.
	\item Eras de las utopías sociales.
	      \begin{subtopics}
		\item Comunismo.
		\item Fascismo.
		\item Nazismo.
	      \end{subtopics}
	\item La guerra fría y la postmodernidad.
\end{topics}
\begin{learningoutcomes}
	\item Analizar los procesos históricos asociados a los radicales cambios ocurridos durante el siglo XIX y consecuente siglo XX. [\Familiarity]
\end{learningoutcomes}
\end{unit}



\begin{coursebibliography}
\bibfile{GeneralEducation/FG205}
\end{coursebibliography}

\end{syllabus}
