\begin{syllabus}

\course{CF151. Laboratorio de Física 1}{Obligatorio}{CF151} % Common.pm

\begin{justification}
Mediante las sesiones de laboratorio, el curso permite a los estudiantes comprobar los conocimientos teóricos a través de experimentos previamente diseñados con el propósito de desarrollar habilidades y destreza en el uso de instrumental de laboratorio, adquirir datos y procesarlos mediante software, comunicar resultados trabajando en equipo. Desarrolla actitudes y habilidades para el logro de las siguientes competencias: aprender a aprender, capacitando al estudiante para la resolución de problemas y realizar operaciones básicas del trabajo experimental; comunicación, al presentar el estudiante informes técnicos claros y precisos utilizando correctamente el idioma; desempeño personal y académico, cuando el estudiante muestra un comportamiento basado en valores y ética profesional, aplica sus habilidades de pensamiento crítico y participa dinámicamente en equipos de trabajo.

Abarca los siguientes contenidos: mediciones y teoría de incertidumbres, mecánica de la partícula que comprende cinemática, dinámica, trabajo y energía, mecánica de sistema de partículas que comprende impulso y cantidad de movimiento, y mecánica del cuerpo rígido.
\end{justification}

\begin{goals}
\item Al término del semestre, el estudiante será capaz de aplicar la teoría correspondiente al curso de Física 1 y emplearla en la resolución de situaciones reales bajo las consideraciones propias del laboratorio, relacionándolas además a situaciones del mundo cotidiano en los temas de mecánica de la partícula, mecánica de sistemas de partículas y cuerpo rígido. Así mismo desarrollar habilidades y destreza en el uso de instrumental de laboratorio, adquirir datos y procesarlos mediante software, comunicar resultados trabajando en equipo.
\end{goals}

\begin{outcomes}
\item \ShowOutcome{a}{2}
\item \ShowOutcome{i}{2}
\end{outcomes}

\begin{competences}
    \item \ShowCompetence{C1,C20}{a,i}
\end{competences}

\begin{unit}{Mediciones y teoría de incertidumbres}{}{GuiaLab2018,Sears2009,Resnick2005,Tipler2010}{2}{C1,C20}
\begin{topics}
      \item Determinación del volumen de un objeto aplicando la teoría de incertidumbres.
\end{topics}
\end{unit}

\begin{unit}{Cinemática}{}{GuiaLab2018,Sears2009,Resnick2005,Tipler2010}{2}{C1,C20}
\begin{topics}
      \item Capacitación en el uso de las herramientas del software Pasco Capstone.
      \item Determinación de la ley de movimiento en un MRUV.
\end{topics}
\end{unit}

\begin{unit}{Dinámica y Rozamiento}{}{GuiaLab2018,Sears2009,Resnick2005,Tipler2010}{2}{C1,C20}
\begin{topics}
      \item Determinación de la aceleración y verificación de la segunda ley de Newton. 
      \item Determinación del coeficiente de rozamiento estático.
\end{topics}
\end{unit}

\begin{unit}{Trabajo y energía}{}{GuiaLab2018,Sears2009,Resnick2005,Tipler2010}{2}{C1,C20}
\begin{topics}
      \item Verificación del Teorema del trabajo y la energía.
   \end{topics}
\end{unit}

\begin{unit}{Colisiones}{}{GuiaLab2018,Sears2009,Resnick2005,Tipler2010}{2}{C1,C20}
\begin{topics}
	\item Verificación del Teorema del impulso y la variación de la cantidad de movimiento lineal. 
	\item Determinación del coeficiente de restitución de una colisión.
   \end{topics}
\end{unit}

\begin{unit}{Máquinas simples}{}{GuiaLab2018,Sears2009,Resnick2005,Tipler2010}{2}{C1,C20}
   \begin{topics}
	\item Estudio de la palanca.
   \end{topics}
\end{unit}



\begin{coursebibliography}
\bibfile{BasicSciences/CF151}
\end{coursebibliography}

\end{syllabus}
