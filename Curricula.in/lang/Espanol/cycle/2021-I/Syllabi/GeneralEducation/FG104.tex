\begin{syllabus}

\course{. }{}{} % Common.pm

\begin{justification}
El estudio de la Filosofía en la universidad, se presenta como un espacio de reflexión constante sobre el ser y el quehacer del ser humano en el mundo. Así mismo, proporciona las herramientas académicas necesarias para la adquisición del pensamiento formal y la actitud crítica frente a las corrientes relativistas que nos alejan de la Verdad.
La formación filosófica aporta considerablemente al cultivo de los saberes, capacidades y potencialidades humanas, de tal manera que facilita al ser humano encontrar el camino hacia la Verdad plena.
\end{justification}

\begin{goals}
\item Estructurar en los alumnos los fundamentos filosóficos buscando desarrollar en ellos las capacidades superiores de pensamiento, a través del estudio crítico de los contenidos temáticos, que les permita despertar la avidez por el saber, el buscar la Verdad y el conocer la realidad de manera objetiva, de tal forma que puedan orientar su vida con criterios claros y razonables.
\end{goals}



\begin{unit}{}{Primera Unidad: Aproximación a la noción de Filosofía }{Pieper}{12}{C20}
\begin{topics}
	\item Importancia de la Filosofia
	\item Filosofía: definición etimológica y real.
	\item El asombro como comienzo del filosofar.
	\item El ocio como condición para la filosofía.
    \item La filosofía como sabiduría natural.
    \item Condiciones morales del filosofar.
    \item Filosofía y otros conocimientos.
    \item Aproximación histórica: Antigua, media, moderna y contemporánea.
\end{topics}

\begin{learningoutcomes}
	\item Investigar y valorar la naturaleza de la filosofía [\Usage].
	\item Identificar y analizar las distintas corrientes filosóficas en la historia [\Usage].
\end{learningoutcomes}
\end{unit}

\begin{unit}{}{Segunda Unidad: El hombre}{Amerio, Acodesi}{6}{C21}
\begin{topics}
	\item Características generales de la Antropología filosófica. Definiciones, objetos, métodos y relación con otros saberes.
	\item Visiones reduccionistas: materialismo y espiritualismo.
	\item Visión integral del ser humano.
    \item La persona humana. Definición. Unidad sustancial del cuerpo y el espíritu.
\end{topics}

\begin{learningoutcomes}
	\item Distinguir las nociones fundamentales de la antropología realista [\Usage].
	\item Valorar críticamente las diversas posturas antropológicas [\Usage].
	\item Identificar y valorar al hombre como un ser personal [\Usage].
\end{learningoutcomes}
\end{unit}

\begin{unit}{}{Tercera Unidad : El conocimiento humano}{Zanotti, Platon, Pieper}{6}{C20}
\begin{topics}
	\item Características generales del conocimiento humano.
    \item Discusión con otras posturas: el escepticismo y el relativismo; racionalismo y empirismo.
    \item La verdad: lógica y ontológica.
\end{topics}
\begin{learningoutcomes}
	\item Examinar las dificultades que plantea el conocimiento humano y las diversas soluciones que se dan a las mismas [\Usage].
    \item Explicar los conceptos fundamentales del realismo gnoseológico [\Usage].
\end{learningoutcomes}
\end{unit}

\begin{unit}{}{Cuarta Unidad : El obrar humano}{AristotelesE}{6}{C20}
\begin{topics}
	\item Características generales: etimología, moral y ética, objeto, tipo de conocimiento.
	\item Criterios de moralidad.
	\item Fuentes de la moralidad.
	\item Relativismo ético.
	\item El bien. El fin último. La felicidad.
	\item Virtudes
\end{topics}
\begin{learningoutcomes}
	\item Distinguir las nociones esenciales de la ética filosófica desde sus fundamentos [\Usage].
    \item Explicar el valor de la vida virtuosa y asumirla como camino a la felicidad [\Usage].
\end{learningoutcomes}
\end{unit}

\begin{unit}{}{Quinta Unidad : El ser}{Gomez, Alvira}{6}{C20}
\begin{topics}
	\item La metafísica como estudio del ser.
	\item Los trascendentales
	\item La estructura del ente finito.Sustancia-Accidente, Materia-Forma, Acto-Potencia, Esencia-Acto de ser.
	\item La causalidad. La existencia de Dios. La creación y sus implicancias.
\end{topics}
\begin{learningoutcomes}
	\item Examinar las características fundamentales de la metafísica y valorar su primacía en el pensamiento filosófico [\Usage].
    \item Comprender las nociones metafísicas fundamentales [\Usage].
    \item Explicar la posibilidad de acceder filosóficamente a Dios como creador [\Usage].
\end{learningoutcomes}
\end{unit}



\begin{coursebibliography}
\bibfile{GeneralEducation/FG101}
\end{coursebibliography}

\end{syllabus}
