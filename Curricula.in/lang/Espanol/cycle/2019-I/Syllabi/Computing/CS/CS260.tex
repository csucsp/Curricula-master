\begin{syllabus}

\course{CS260. Lógica Computacional}{Obligatorio}{CS260}

\begin{justification}
El presente es un curso avanzado de lógica para informáticos. De
entre las distintas aplicaciones de la lógica en la informática, se
pueden destacar, entre otras, las técnicas de verificación formal de
programas, la programación lógica o la inteligencia artificial. Como
complemento a los fundamentos teóricos del curso, se introduce el
problema de la demostración automática de teoremas. Se presentan
diferentes heurísticas para la demostración automática de teoremas,
así como distintos sistemas implementados con los que comprobar la
potencia de las técnicas expuestas. Los sistemas de demostración
automática de teoremas resultan particularmente útiles en el
desarrollo de métodos formales en la ingeniería del software.
\end{justification}

\begin{goals}
\item Conocer los métodos de la lógica (lógica de predicados y de la lógica modal) que más se utilizan hoy en día en ciencia de la computación, ingeniería del software e inteligencia artificial. 
\item Desarrollar habilidades y aptitudes para la representación formal del conocimiento, la operación simbólica sobre sistemas formales, la demostración de teoremas y la interpretación semántica.
\item Habilitar al alumno para saber pensar de forma lógica, analítica, crítica y estructurada y con ello argumentar e inferir correctamente.
\item Comprender los mecanismos computacionales asociados a las problemáticas de la demostración automática y la programación lógica, y descubrir la importancia del control en su resolución.
\end{goals}

\begin{outcomes}
\ExpandOutcome{a}{4}
\ExpandOutcome{h}{3}
\ExpandOutcome{j}{4}
\end{outcomes}

\begin{unit}{Lógica de Predicados de Primer Orden}{Iranzo05,Paniagua03}{20}{3}
   \begin{topics}
         \item  Sintaxis y Semántica
         \item  El método axiomático y el método interpretativo
         \item  Demostración automática de teoremas
         \item  Los agentes inteligentes y la lógica
   \end{topics}
   \begin{learningoutcomes}
         \item Fundamentar que la lógica constituye la base matemática del software
         \item Desarrollar sólidas bases formales mediante la lógica: en el proceso de representación del conocimiento, así como en el proceso deductivo.
   \end{learningoutcomes}
\end{unit}

\begin{unit}{Intensificación en Programación}{Lloyd93, Bratko91}{20}{3}
   \begin{topics}
         \item Programación Lógica
         \item Programación Lógica Avanzada
         \item La síntesis de programas a partir de especificaciones
   \end{topics}

   \begin{learningoutcomes}
        \item Presentar los conceptos fundamentales del paradigma de programación lógica
        \item Presentar algunas técnicas de Análisis y Depuración de programas lógicos
        \item Presentar una introducción a la programación automática.
   \end{learningoutcomes}
\end{unit}

\begin{unit}{Extensiones y otras Lógicas}{Fernandez03, Klir95}{20}{5}
   \begin{topics}
      \item Lógicas Multivalentes
      \item Lógica Hoare
      \item Lógica Modal
      \item Lógica Temporal
   \end{topics}

   \begin{learningoutcomes}
      \item Representar aspectos complejos de la realidad en la que no es factible asignar asignar un rango de dos valores de verdad a los enunciados( lógica trivalente y lógica difusa)
      \item Establecer las nociones fundamentales de especificación formal y verificación de programas
   \end{learningoutcomes}
\end{unit}



\begin{coursebibliography}
\bibfile{Computing/CS/CS105}
\end{coursebibliography}

\end{syllabus}

%\end{document}
