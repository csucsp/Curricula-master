\begin{syllabus}

\course{CS2201. Arquitectura de Computadores}{Obligatorio}{CS2201}
% Source file: ../Curricula.in/lang/Espanol/cycle/2019-I/Syllabi/Computing/CS/CS221.tex

\begin{justification}
Es necesario que el profesional en Ciencia de la Computación tenga sólido  conocimiento de la organización y funcionamiento de los diversos sistema de cómputo actuales en los cuales gira se instala el entorno de programación. Con ello también sabrá establecer los alcances y límites de las aplicaciones que se desarrollen de acuerdo a la plataforma siendo usada.

Se tratarán los siguientes temas: componentes de lógica digital básicos en un sistema de computación, diseño de conjuntos de instrucciones, microarquitectura del procesador y ejecución en \textit{pipelining}, organización de la memoria: caché y memoria virtual, protección y compartición, sistema I/O e interrupciones, arquitecturas super escalares y ejecución fuera de orden, computadoras vectoriales, arquitecturas para \textit{multithreading}, multiprocesadores simétricos, modelo de memoria y sincronización, sistemas integrados y computadores en paralelo.

\end{justification}

\begin{goals}
\item Este curso tiene como propósito ofrecer al estudiante una base sólida de la evolución de las arquitecturas de computadores y los factores que influenciaron en el diseño de los elementos de \textit{hardware} y \textit{software} en sistemas de computación actuales. 
\item Garantizar la comprensión de cómo es el \textit{hardware} en sí y cómo interactuan \textit{hardware} y \textit{software} en un sistema de cómputo actual.
\item Tratar los siguientes temas: componentes de lógica digital básicos en un sistema de computación, diseño de conjuntos de instrucciones, microarquitectura del procesador y ejecución en \textit{pipelining}, organización de la memoria: caché y memoria virtual, protección y compartición, sistema I/O e interrupciones, arquitecturas super escalares y ejecución fuera de orden, computadoras vectoriales, arquitecturas para \textit{multithreading}, multiprocesadores simétricos, modelo de memoria y sincronización, sistemas integrados y computadores en paralelo.
\end{goals}

\begin{outcomes}{V1}
    \item \ShowOutcome{b}{2}
    \item \ShowOutcome{g}{2}
    \item \ShowOutcome{i}{3}
\end{outcomes}

\begin{outcomes}{V2}
    \item \ShowOutcome{1}{2}
    \item \ShowOutcome{2}{2}
\end{outcomes}

\begin{competences}{V1}
    \item \ShowCompetence{C4}{i} 
    \item \ShowCompetence{C8}{b,i,g}
    \item \ShowCompetence{C9}{b,g}
\end{competences}

\begin{competences}{V2}
    \item \ShowCompetence{C4}{1} 
    \item \ShowCompetence{C8}{2,1}
    \item \ShowCompetence{C9}{2}
\end{competences}

\begin{unit}{\ARDigitallogicanddigitalsystems}{}{Parhami2005,Patterson2004}{18}{C8}
\begin{topics}%
    \item \ARDigitallogicanddigitalsystemsTopicOverview
    \item \ARDigitallogicanddigitalsystemsTopicCombinational
    \item \ARDigitallogicanddigitalsystemsTopicMultiple
    \item \ARDigitallogicanddigitalsystemsTopicComputer
    \item \ARDigitallogicanddigitalsystemsTopicRegister
    \item \ARDigitallogicanddigitalsystemsTopicPhysical
\end{topics}
\begin{learningoutcomes}
    \item \ARDigitallogicanddigitalsystemsLODescribeTheComputer [\Familiarity]
    \item \ARDigitallogicanddigitalsystemsLOComprehend [\Usage]
    \item \ARDigitallogicanddigitalsystemsLOExplainTheThe [\Usage]
    \item \ARDigitallogicanddigitalsystemsLOArticulate [\Familiarity]
    \item \ARDigitallogicanddigitalsystemsLODesignThe [\Usage]
    \item \ARDigitallogicanddigitalsystemsLOUseCad [\Familiarity]
    \item \ARDigitallogicanddigitalsystemsLOEvaluate [\Assessment]
\end{learningoutcomes}
\end{unit}

\begin{unit}{\ARMachinelevelrepresentationofdata}{}{Parhami2005,Stallings2010}{8}{C9}
\begin{topics}
    \item \ARMachinelevelrepresentationofdataTopicBits
    \item \ARMachinelevelrepresentationofdataTopicNumeric
    \item \ARMachinelevelrepresentationofdataTopicFixed
    \item \ARMachinelevelrepresentationofdataTopicSigned
    \item \ARMachinelevelrepresentationofdataTopicRepresentation
    \item \ARMachinelevelrepresentationofdataTopicRepresentationOf
\end{topics}

\begin{learningoutcomes}
    \item \ARMachinelevelrepresentationofdataLOExplainWhyData [\Assessment]
    \item \ARMachinelevelrepresentationofdataLOExplainTheUsing [\Familiarity]
    \item \ARMachinelevelrepresentationofdataLODescribeHowAre [\Usage]
    \item \ARMachinelevelrepresentationofdataLOExplainHowNumber [\Usage]
    \item \ARMachinelevelrepresentationofdataLODescribeTheOf [\Usage]
    \item \ARMachinelevelrepresentationofdataLOConvertNumerical [\Usage]
\end{learningoutcomes}
\end{unit}

\begin{unit}{\ARAssemblylevelmachineorganization}{}{Parhami2005,Patterson2004,Hennessy2006}{8}{C4,CS3}
\begin{topics}
  \item \ARAssemblylevelmachineorganizationTopicBasic
  \item \ARAssemblylevelmachineorganizationTopicControl
  \item \ARAssemblylevelmachineorganizationTopicInstruction
  \item \ARAssemblylevelmachineorganizationTopicAssembly
  \item \ARAssemblylevelmachineorganizationTopicInstructionFormats
  \item \ARAssemblylevelmachineorganizationTopicAddressing
  \item \ARAssemblylevelmachineorganizationTopicSubroutine
  \item \ARAssemblylevelmachineorganizationTopicI
  \item \ARAssemblylevelmachineorganizationTopicHeap
\end{topics}

\begin{learningoutcomes}
  \item \ARAssemblylevelmachineorganizationLOExplainTheTheNeumann [\Familiarity]
  \item \ARAssemblylevelmachineorganizationLODescribeHowIs [\Familiarity]
  \item \ARAssemblylevelmachineorganizationLODescribeInstruction [\Familiarity]
  \item \ARAssemblylevelmachineorganizationLOSummarize [\Familiarity]
  \item \ARAssemblylevelmachineorganizationLODemonstrateHow [\Usage]
  \item \ARAssemblylevelmachineorganizationLOExplainDifferent [\Usage]
  \item \ARAssemblylevelmachineorganizationLOExplainHowAre [\Usage]
  \item \ARAssemblylevelmachineorganizationLOExplainTheOf [\Familiarity]
  \item \ARAssemblylevelmachineorganizationLOWriteSimple [\Usage]
  \item \ARAssemblylevelmachineorganizationLOShow  [\Usage]
\end{learningoutcomes}
\end{unit}

%% Unidad Organización Funcional para ejecución de instrucciones  
\begin{unit}{\ARFunctionalorganization}{}{Parhami2005,Hennessy2006}{8}{C9}
\begin{topics}
      \item \ARFunctionalorganizationTopicImplementation
      \item \ARFunctionalorganizationTopicControl
      \item \ARFunctionalorganizationTopicInstruction
      \item \ARFunctionalorganizationTopicIntroductionTo
\end{topics}

\begin{learningoutcomes}
\item \ARFunctionalorganizationLOCompareAlternative [\Assessment]
\item \ARFunctionalorganizationLODiscussTheControl [\Familiarity]
\item \ARFunctionalorganizationLOExplainBasic [\Usage]
\item \ARFunctionalorganizationLODesignAnd [\Usage]
\item \ARFunctionalorganizationLODetermineFor [\Assessment]
\end{learningoutcomes}
\end{unit}

%% Organización y Arquitectura de la Memoria
\begin{unit}{\ARMemorysystemorganizationandarchitecture}{}{Parhami2005,Patterson2004,Denning2005}{8}{CS3}
\begin{topics}
  \item \ARMemorysystemorganizationandarchitectureTopicStorage
  \item \ARMemorysystemorganizationandarchitectureTopicMemory
  \item \ARMemorysystemorganizationandarchitectureTopicMain
  \item \ARMemorysystemorganizationandarchitectureTopicLatency
  \item \ARMemorysystemorganizationandarchitectureTopicCache
  \item \ARMemorysystemorganizationandarchitectureTopicMultiprocessor
  \item \ARMemorysystemorganizationandarchitectureTopicVirtual
  \item \ARMemorysystemorganizationandarchitectureTopicFault
  \item \ARMemorysystemorganizationandarchitectureTopicError
\end{topics}

\begin{learningoutcomes}
  \item \ARMemorysystemorganizationandarchitectureLOIdentify [\Familiarity]
  \item \ARMemorysystemorganizationandarchitectureLOExplainTheMemory [\Familiarity]
  \item \ARMemorysystemorganizationandarchitectureLODescribeHowOf [\Usage]
  \item \ARMemorysystemorganizationandarchitectureLODescribeTheMemory [\Usage] 
  \item \ARMemorysystemorganizationandarchitectureLOExplainTheA [\Usage] 
  \item \ARMemorysystemorganizationandarchitectureLOCompute [\Assessment]
\end{learningoutcomes}
\end{unit}

%% Unidad I/O: Interfacing and Communications
\begin{unit}{\ARInterfacingandcommunication}{}{Parhami2005,Stallings2010}{8}{C4,C9,CS3}
\begin{topics}
	\item \ARInterfacingandcommunicationTopicI
	\item \ARInterfacingandcommunicationTopicInterrupt
	\item \ARInterfacingandcommunicationTopicExternal
	\item \ARInterfacingandcommunicationTopicBuses
	\item \ARInterfacingandcommunicationTopicIntroduction
	\item \ARInterfacingandcommunicationTopicMultimedia
	\item \ARInterfacingandcommunicationTopicRaid
 \end{topics}
 
\begin{learningoutcomes}
	\item \ARInterfacingandcommunicationLOExplainHowUsed [\Familiarity]
	\item \ARInterfacingandcommunicationLOIdentifyVarious [\Familiarity]
	\item \ARInterfacingandcommunicationLODescribeData [\Usage]
	\item \ARInterfacingandcommunicationLOCompare [\Assessment]
	\item \ARInterfacingandcommunicationLOIdentifyThe [\Familiarity]
	\item \ARInterfacingandcommunicationLODescribeTheLimitations [\Familiarity]
\end{learningoutcomes}
\end{unit}

%% Unidad Arquitecturas Alternativas y en Paralelo
\begin{unit}{\ARMultiprocessingandalternativearchitectures}{}{Parhami2005,Parhami2002,ElRewini2005}{8}{C9}
\begin{topics}
	\item \ARMultiprocessingandalternativearchitecturesTopicPower
	\item \ARMultiprocessingandalternativearchitecturesTopicExample
	\item \ARMultiprocessingandalternativearchitecturesTopicInterconnection
	\item \ARMultiprocessingandalternativearchitecturesTopicShared
	\item \ARMultiprocessingandalternativearchitecturesTopicMultiprocessor
\end{topics}

\begin{learningoutcomes}
	\item \ARMultiprocessingandalternativearchitecturesLODiscussTheParallel [\Assessment]
	\item \ARMultiprocessingandalternativearchitecturesLODescribeAlternative [\Familiarity]
	\item \ARMultiprocessingandalternativearchitecturesLOExplainTheInterconnection [\Usage]
	\item \ARMultiprocessingandalternativearchitecturesLODiscussTheThat [\Familiarity]
	\item \ARMultiprocessingandalternativearchitecturesLODescribeTheMemoryMemory [\Assessment]
\end{learningoutcomes}
\end{unit}

%% Unidad Mejoras en Desempeño
\begin{unit}{\ARPerformanceenhancements}{}{Parhami2005,Parhami2002,Patterson2004,Dongarra2006,Johnson1991}{8}{C8,C9}
\begin{topics}
	\item \ARPerformanceenhancementsTopicSuperscalar
	\item \ARPerformanceenhancementsTopicBranch
	\item \ARPerformanceenhancementsTopicPrefetching
	\item \ARPerformanceenhancementsTopicVector
	\item \ARPerformanceenhancementsTopicHardware
	\item \ARPerformanceenhancementsTopicScalability
	\item \ARPerformanceenhancementsTopicAlternative
\end{topics}

\begin{learningoutcomes}
  \item \ARPerformanceenhancementsLODescribeSuperscalar [\Familiarity]
  \item \ARPerformanceenhancementsLOExplainTheBranch [\Usage]
  \item \ARPerformanceenhancementsLOCharacterize [\Assessment]
  \item \ARPerformanceenhancementsLOExplainSpeculative [\Assessment]
  \item \ARPerformanceenhancementsLODiscussTheThatIn [\Assessment]
  \item \ARPerformanceenhancementsLODescribeTheScalability [\Assessment]
\end{learningoutcomes}
\end{unit}

\begin{coursebibliography}
\bibfile{Computing/CS/CS221}
\end{coursebibliography}

\end{syllabus}
