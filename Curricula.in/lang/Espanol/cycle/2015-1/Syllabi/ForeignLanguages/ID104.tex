\begin{syllabus}

\course{ID202. Lengua Extranjera IV}{Obligatorio}{ID104}

\begin{justification}
Parte fundamental de la formación integral de un profesional es la habilidad 
de comunicarse en un idioma extranjero además del propio idioma nativo. 
No solamente amplía su horizonte cultural sino que permite una visión 
más humana y comprensiva de la vida. En el caso de los idiomas extranjeros, 
indudablemente el Inglés es el más práctico porque es hablado alrededor de 
todo el mundo. No hay país alguno donde éste no sea hablado. Además de ser 
vital para su carrera profesional.
\end{justification}

\begin{goals}
\item Incrementar el nivel de conversación en diferentes temas, en los alumnos. Así como la capacidad de escribir y leer documentación de todo tipo.
\item Llevar al alumno a una expresión más intensa en el dominio del idioma.
\end{goals}

\begin{outcomes}
\ExpandOutcome{f}{4}
\ExpandOutcome{HU}{4}
\end{outcomes}

\begin{unit}{Do and don't!}{Soars022S,Soars022W,Soars022T, Cambridge06, MacGrew99}{0}{4}
   \begin{topics}
      \item Auxiliares de Modo should, must y have got to.
      \item Oraciones afirmativas, negativas e interrogativas con modals.
      \item Términos para cartas formales.
      \item Partes de las respuestas cortas.
      \item Expresiones para ocupaciones.
   \end{topics}

   \begin{learningoutcomes}
      \item Al terminar la octava unidad, cada una de las alumnas, comprendiendo la gramática de los auxiliares should y must  es capaz de expresar una mayor cantidad de acciones en forma obligación y sugerencia.  Además es capaz de expresar ideas describiendo ocupaciones.  Asume la necesidad de escribir cartas formales. 
   \end{learningoutcomes}
\end{unit}

\begin{unit}{Going places!}{Soars022S,Soars022W,Soars022T, Cambridge06, MacGrew99}{0}{4}
   \begin{topics}
      \item Tiempo Presente Simple y Futuro con Will
      \item Primer Condicional
      \item Colocaciones
      \item Vocabulario de preposiciones de lugar y de tiempo
      \item Expresiones de conexión de ideas
   \end{topics}

   \begin{learningoutcomes}
      \item Al terminar la novena unidad, los alumnos habiendo identificado la forma de expresar presente reconocen la diferencia entre las formas de futuro y las aplican adecuadamente. Describen condiciones acuciosamente.  Asumen expresiones para demostrar ubicación de lugar. Utilizan expresiones de tiempo y conectores para unir ideas varias.
   \end{learningoutcomes}
\end{unit}

\begin{unit}{Scared to death!}{Soars022S,Soars022W,Soars022T, Cambridge06, MacGrew99}{0}{4}
   \begin{topics}
      \item Patrones de Verbos Infinitivos y gerundios
      \item What + Infinitivo
      \item Something + infinitive
      \item Expresiones de sentimientos
      \item Exclamaciones de sorpresa 
   \end{topics}

   \begin{learningoutcomes}
      \item Al terminar la décima unidad los alumnos serán capaces de reconocer y utilizar patrones tiempos en pasado los utilizan adecuadamente. Utilizan expresiones de exclamación. Y describen sentimientos. Utilizarán conjunciones para unir ideas tipo.
   \end{learningoutcomes}
\end{unit}

\begin{unit}{Things that changed the world!}{Soars022S,Soars022W,Soars022T, Cambridge06, MacGrew99}{0}{4}
   \begin{topics}
      \item Voz Pasiva
      \item Oraciones Afirmativas, Negativas y Preguntas
      \item Uso de Participios, verbos y sustantivos que van unidos
      \item Señales. Signos y notas
      \item Resúmenes
      \item Expresiones para indicar prohibición
   \end{topics}

   \begin{learningoutcomes}
      \item Al terminar la décimo primera primera unidad, los alumnos habiendo identificado la idea de acciones pasivas describen acciones adecuadamente en diversas situaciones que la involucran. Reconocen y aplican participios. Asumen la idea de respetar signos y señales públicas. Expresan ideas de hábitos. Hacen resúmenes.
   \end{learningoutcomes}
\end{unit}

\begin{unit}{Dreams and reality!}{Soars022S,Soars022W,Soars022T, Cambridge06, MacGrew99}{0}{4}
   \begin{topics}
      \item Segundo Condicional
      \item Auxiliar de Modo Might
      \item Verbos de Frase
      \item Vocabulario de expresiones sociales
      \item Adverbios
      \item Expresiones para dar consejo
   \end{topics}

   \begin{learningoutcomes}
      \item Al finalizar la décimo segunda unidad, los alumnos, a partir de la comprensión de la idea de Condicionales y de expresar posibilidad elaborarán oraciones utilizando los elementos necesarios. Asimilarán además la necesidad de frases verbales (verbos de 2 palabras). Adquirirán vocabulario para describir expresiones sociales.
   \end{learningoutcomes}
\end{unit}

\begin{unit}{Making a living!}{Soars022S,Soars022W,Soars022T, Cambridge06, MacGrew99}{0}{4}
   \begin{topics}
      \item Present Perfect Continuous
      \item Present Continuous
      \item Ocupaciones 
      \item Formación de palabras
      \item Adverbios
      \item Expresiones de uso en el teléfono
   \end{topics}

   \begin{learningoutcomes}
      \item Al finalizar la décimo tercera unidad estructuran oraciones con acciones que incluyen presente y pasado en contextos adecuados. Enfatizan la diferencia entre tipos de ocupaciones. Utilizan expresiones adecuadas para conversaciones telefónicas
   \end{learningoutcomes}
\end{unit}

\begin{unit}{All you need is love!}{Soars022S,Soars022W,Soars022T, Cambridge06, MacGrew99}{0}{4}
   \begin{topics}
      \item Pasado Perfecto y Pasado Simple
      \item Expresiones de Reporte
      \item Expresiones de palabras en contextos diferentes 
      \item Despedidas cortas y formales 
      \item Historias de amor
   \end{topics}

   \begin{learningoutcomes}
      \item Al finalizar la décimo cuarta unidad, los alumnos habiendo conocido los fundamentos de la estructuración del tiempo pasado perfecto, lo diferencian del pasado simple. Enfatizan la diferencia entre palabras en contextos diferentes. Describen ideas de despedidas. Utilizan expresiones para escribir historias de amor. Asumen la idea de dar y hacer entrevistas.
   \end{learningoutcomes}
\end{unit}



\begin{coursebibliography}
\bibfile{ForeignLanguages/ID101}
\end{coursebibliography}

\end{syllabus}
%\end{document}
