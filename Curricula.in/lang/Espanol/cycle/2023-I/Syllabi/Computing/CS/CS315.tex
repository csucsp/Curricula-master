\begin{syllabus}

\course{CS315. Estructuras de Datos Avanzadas}{Obligatorio}{CS315}

\begin{justification}
Los algoritmos y estructuras de datos son una parte fundamental de la ciencia de la computación que nos 
permiten organizar la información de una manera más eficiente, por lo que es importante para todo 
profesional del área tener una sólida formación en este aspecto.

En el curso de estructuras de datos avanzadas nuestro objetivo es que el alumno conozca y analice 
estructuras complejas como los Métodos de Acceso Multidimensional, 
Métodos de Acceso Espacio-Temporal y 
Métodos de Acceso Métrico, Estructuras de Datos Compactas, etc.
\end{justification}

\begin{goals}
\item Que el alumno entienda, diseñe, implemente, aplique y
proponga estructuras de datos innovadoras para solucionar
problemas relacionados al tratamiento de datos multidimensionales,
recuperación de información por similitud, motores de búsqueda y
otros problemas computacionales.
\end{goals}



\begin{unit}{Técnicas Básicas de Implementación de Estructuras de Datos}{}{Cuadros2004Implementing,Knuth2007TAOCP-V-I,Knuth2007TAOCP-V-II,Gamma94}{16}{a,b,c}
   \begin{topics}
         \item Programación estructurada
         \item Programación Orientada a Objetos
         \item Tipos Abstractos de Datos
         \item Independencia del lenguaje de programación del usuario de la estructura
         \item Independencia de Plataforma
         \item Control de concurrencia
         \item Protección de Datos
         \item Niveles de encapsulamiento (struct, class, namespace, etc)
   \end{topics}

   \begin{learningoutcomes}
         \item Que el alumno entienda las diferencias básicas que involucran las distintas técnicas de implementación de estructuras de datos
         \item Que el alumno analice las ventajas y desventajas de cada una de las técnicas existentes
   \end{learningoutcomes}
\end{unit}

\begin{unit}{Métodos de Acceso Multidimensionales}{}{Samet2004SAM-MAM,Gaede98multidimensional}{16}{a,b,c}
   \begin{topics}
         \item Métodos de Acceso para datos puntuales
         \item Métodos de Acceso para datos más complejos
         \item Problemas relacionados con el aumento de dimensión
   \end{topics}

   \begin{learningoutcomes}
         \item Que el alumno entienda conozca e implemente algunos Métodos de Acceso para datos multidimensionales y espacio temporales
         \item Que el alumno entienda el potencial de estos Métodos de Acceso en el futuro de las bases de datos comerciales
   \end{learningoutcomes}
\end{unit}

\begin{unit}{Métodos de Acceso Métrico}{}{Samet2004SAM-MAM,Chavez:01,Traina00SlimTree,Zezula07}{20}{a,b,c,i}
   \begin{topics}
         \item Métodos de Acceso Métrico para distancias discretas
         \item Métodos de Acceso Métrico para distancias continuas
   \end{topics}

   \begin{learningoutcomes}
         \item Que el alumno entienda conozca e implemente algunos métodos de acceso métrico
         \item Que el alumno entienda la importancia de estos Métodos de Acceso para la Recuperación de Información por Similitud
   \end{learningoutcomes}
\end{unit}

\begin{unit}{Métodos de Acceso Aproximados}{}{Samet2004SAM-MAM,Indyk06,Zezula07}{20}{a,b,c,i}
   \begin{topics}
         \item \textit{Space Filling Curves}
         \item \textit{Locality Sensitive Hashing}
   \end{topics}

   \begin{learningoutcomes}
         \item Que el alumno entienda conozca e implemente algunos métodos de acceso aproximados
         \item Que el alumno entienda la importancia de estos Métodos de Acceso para la Recuperación de Información por Similitud en entornos donde la Escalabilidad sea una factor muy importante
   \end{learningoutcomes}
\end{unit}

\begin{unit}{Seminarios}{}{Samet2004SAM-MAM,Navarro:CompactDS:2016,Chavez:01}{8}{a,b,c,i}
	\begin{topics}
         \item Métodos de Acceso Espacio Temporal
         \item Estructuras de Datos Compactas
         \item Estructuras de Datos con programación genérica
   \end{topics}

   \begin{learningoutcomes}
         \item Que el alumno pueda discutir sobre los últimos avances en métodos de acceso para distintos dominios de conocimiento
   \end{learningoutcomes}
\end{unit}

\begin{coursebibliography}
\bibfile{Computing/CS/CS315}
\end{coursebibliography}

\end{syllabus}
