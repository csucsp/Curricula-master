\begin{syllabus}

\course{CB1006. Ingeniería de materiales}{Obligatorio}{CB1006}
% Source file: ../Curricula.in/lang/Espanol/cycle/2020-I/Syllabi/BasicSciences/CB106.tex

\begin{justification}
Este curso presenta a los estudiantes conceptos y técnicas de diseño clásico y bayesiano: unidades experimentales, aleatorización, tratamientos, bloqueos y restricciones a la aleatorización, y la utilidad de los diseños. Muestra  la determinación óptima del tamaño de la muestra para la estimación y las pruebas. Los temas incluyen pruebas A-B simples, diseños factoriales factoriales y factoriales fraccionados, métodos de superficie de respuesta, diseños conjuntos, diseños secuenciales, diseño y modelado de experimentos informáticos complejos, y diseños para objetivos múltiples. 
\end{justification}

\begin{goals}
\item Asimilar
\end{goals}

\begin{outcomes}{V1}
   \item \ShowOutcome{a}{3}
   \item \ShowOutcome{i}{3}
   \item \ShowOutcome{j}{4}
\end{outcomes}

\begin{unit}{Números }{}{Simmons95,Bartle99}{20}{3}
   \begin{topics}
      \item Números 
   \end{topics}

   \begin{learningoutcomes}
      \item Comprender 
      \end{learningoutcomes}
\end{unit}

\begin{coursebibliography}
\bibfile{BasicSciences/CB106}
\end{coursebibliography}

\end{syllabus}
