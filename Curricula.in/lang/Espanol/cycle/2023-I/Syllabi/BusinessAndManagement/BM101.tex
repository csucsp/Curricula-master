\begin{syllabus}

\course{GH0008. Gestión de Empresas}{Obligatorio}{GH0008}
% Source file: ../Curricula.in/lang/Espanol/cycle/2020-I/Syllabi/BusinessAndManagement/BM101.tex

\begin{justification}
Este curso tiene como objetivo proporcionar a los estudiantes con las herramientas necesarias para ir un paso más allá de la idea inicial y modelo de negocio. Aprenderán los primeros pasos hacia la conceptualización de una empresa y la construcción de su equipo. También explorarán los fundamentos de la creación de un plan de negocios
eficaz.Es el segundo de un conjunto de tres cursos diseñados para acompañar a los estudiantes a medida que transforman una idea en un negocio o emprendimiento, desde la ideación, hasta la revisión de la estrategia empresarial actual
\end{justification}

\begin{goals}
\item Entendimiento de  los conceptos básicos del proceso de planificación de negocios y su papel dentro del ciclo de vida empresarial
\end{goals}

\begin{outcomes}{V1}
   \item \ShowOutcome{f}{2}
   \item \ShowOutcome{h}{2}
   \item \ShowOutcome{n}{2}
\end{outcomes}

\begin{competences}{V1}
    \item \ShowCompetence{C17}{f,h,n}
    \item \ShowCompetence{C20}{f,n}
    \item \ShowCompetence{C24}{f,h}
\end{competences}

\begin{unit}{Gestión de Empresas}{}{Maurya12,Kotler03}{16}{C17,C20}
\begin{topics}
      \item El ciclo de vida empresarial:?`Por qué necesito un plan de negocios?
      \item Diferencias entre el modelo de negocio y la planificación empresarial
      \item La importancia de un equipo bien estructurado
      \item Análisis ambiental y principales herramientas de investigación de mercado
      \item Planificación Estratégica: Por qué es necesario y cómo se hace
      \item La importancia del capital: humano, financiero e intelectual
      \item Cómo construir un plan de operaciones
      \item Los fundamentos del marketing: definir estrategia de marketing
      \item Proyecciones financieras: costos y ventas
      \item Asuntos legales
      \item Negocios Responsables: lo básico
\end{topics}

\begin{learningoutcomes}
   \item Entendimiento de la importancia de una planificación eficaz y cómo contribuye al lanzamiento y éxito de una empresa.
\end{learningoutcomes}
\end{unit}

\begin{coursebibliography}
\bibfile{BusinessAndManagement/BM101}
\end{coursebibliography}

\end{syllabus}
