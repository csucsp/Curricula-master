\begin{syllabus}

\course{CB121. QUÍMICA GENERAL}{Obligatorio}{CB121}

\begin{justification}
Un aspecto muy importante en el nivel universitario lo constituye el cálculo diferencial,  aspecto que constituye la piedra angular de las posteriores asignaturas de matemáticas así como de la utilidad de la matemática en la solución de problemas aplicados a la ciencia y la tecnología. Cualquier profesional con rango universitario debe por lo tanto tener conocimiento amplio de esta asignatura, pues se convertirá en su punto de partida para los intereses de su desarrollo profesional; así también será soporte para no tener dificultades en las asignaturas de matemática y física de toda la carrera.
% The course will use case-based learning to introduce chemistry
% concepts of interest for engineering students. The course will be built around lab experiences,
% which will be related to the cases in the first column. There are 12 topics, where one topic will typically be covered in one week.
\end{justification}

\begin{goals}
\item Asimilar y manejar los conceptos de función, sucesión y relacionarlos con los de límites y continuidad.
% Apply the concepts of complex numbers and functions to solve problems related to science and engineering.
\item Describir, analizar, diseñar y formular modelos continuos que dependan de una variable.
%Apply mathematical concepts and techniques of differential calculus of one variable to solve problematic situations of science and engineering.
\item Conocer y manejar la propiedades del cálculo diferencial y aplicarlas a la resolución de problemas.
% Calculate mathematical expressions of indefinite integrals with accuracy, order and clarity in the treatment of the data.
\end{goals}

\begin{outcomes}
    \item \ShowOutcome{a}{3}
    \item \ShowOutcome{j}{3}
\end{outcomes}

\begin{competences}
    \item \ShowCompetence{C1}{a}
    \item \ShowCompetence{C20}{j}
    \item \ShowCompetence{C24}{j}
\end{competences}

\begin{unit}{}{Nuclear energy and fundamentals of electrical conduction}{Leithold2000,Stewart,ThomasFinney}{20}{C1}
% begin{unit}{Nuclear energy and fundamentals of electrical conduction}{MSaE2014}{0}{3}
   \begin{topics}
    \item Atomic structure
    \item Periodic Table and electronic properties
    %item Atomic structure
    %item Periodic Table and electronic properties
    \end{topics}

   \begin{learningoutcomes}
      \item Comprender la importancia del sistema de los números reales (construcción), manipular los axiomas algebraicos y de orden [\Assessment].
      \item Comprender el concepto de función. Manejar dominios, operaciones, gráficas, inversas [\Assessment].
      \end{learningoutcomes}
\end{unit}

\begin{unit}{}{Combustión}{Leithold2000,ThomasFinney}{10}{C20}
% begin{unit}{}{Combustion}{Leithold2000,ThomasFinney}{10}{C20}
  
  \begin{topics}
    \item Stoichiometry
    \item Chemical bonding and formation of molecules and materials.   

    %item Stoichiometry
    %item Chemical bonding and formation of molecules and materials.   
  \end{topics}

   \begin{learningoutcomes}
      \item Entender el concepto de sucesión y su importancia [\Assessment].
      \item Conecer los principales tipos de sucesiones, manejar sus propiedades [\Assessment].
      \item Manejar y calcular límites de sucesiones [\Assessment].
      \end{learningoutcomes}
\end{unit}

\begin{unit}{}{Air pollution}{Leithold2000,Leithold2000,Stewart}{20}{C1}
% begin{unit}{}{Air pollution}{Leithold2000,Leithold2000,Stewart}{20}{C1}
   \begin{topics}
      \item Límites
      \item Continuidad
      \item Aplicaciones de funciones continuas. Teorema del valor intermedio
%       item Properties of gases
%       item Ideal gases and determination of concentrations in gases    
    \end{topics}

   \begin{learningoutcomes}
      \item Comprender el concepto de límite. calcular límites [\Assessment].
      \item Analizar la continuidad de una función [\Assessment].
      \item Aplicar el teorema del valor intermedio [\Assessment].
      \end{learningoutcomes}
\end{unit}

\begin{unit}{}{Contaminación del agua}{Leithold2000,ThomasFinney,Stewart}{22}{C20}
% begin{unit}{}{Water pollution}{Leithold2000,ThomasFinney,Stewart}{22}{C20}
   \begin{topics}
      \item Definición. reglas de derivación
      \item Incrementos y diferenciales
      \item Regla de la cadena. Derivación implícita

%       item Properties of liquids and solids
%       item Preparation of solutions and units of concentration.
%       item Acids and bases
   \end{topics}

   \begin{learningoutcomes}
      \item Comprender el concepto de derivada e interpretarlo [\Assessment].
      \item Manipular las reglas de derivación [\Assessment].
      \end{learningoutcomes}
\end{unit}


\begin{unit}{}{Corrosion and fuel cells}{Leithold2000,ThomasFinney,Stewart}{22}{C20}
% begin{unit}{}{Corrosion and fuel cells}{Leithold2000,ThomasFinney,Stewart}{22}{C20}
   \begin{topics}
      \item Definición. reglas de derivación
      \item Incrementos y diferenciales
      \item Regla de la cadena. Derivación implícita

%       item Redox reactions
%       item Galvanic cells
%       item Electrochemical cells
   \end{topics}

   \begin{learningoutcomes}
      \item Comprender el concepto de derivada e interpretarlo [\Assessment].
      \item Manipular las reglas de derivación [\Assessment].
      \end{learningoutcomes}
\end{unit}



\begin{coursebibliography}
\bibfile{BasicSciences/MA100}
\end{coursebibliography}

\end{syllabus}

%\end{document}
