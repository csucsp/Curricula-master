\begin{syllabus}

\course{CB113. Termodinámica}{Obligatorio}{CB113}

\begin{justification}
Este es un curso general introductorio en Ingenieríatermodinámica. 
Se centra sobretodo en las propiedades termofísicas, las leyes de la termodinámica, 
el concepto del balance de masa y energía, y en los sistemas de conversión de energía estándar.
Aplicaciones ingenieriles y casos pertenecientes a diferentes carreras serán usados a lo largo del curso.
Capacidad de conocimiento sobre los diferentes tópicos del curso, establecerá las bases para los estudiantes
en buscar información más avanzada cuando sea necesario y poder participar en cursos más avanzados en termodinámica.

% 
% This is a general, introductory course in Engineering Thermodynamics. It focus mostly
% on thermophysical properties of substances, the Laws of Thermodynamics, the concept
% of mass and energy balances, and on standard energy conversion systems. Engineering
% applications and cases pertaining to different careers will be used throughout course.
% Knowledgeability about the course topics will set the bases for students to search more
% advanced information when necessary and for participating in more advanced courses on
% Thermodynamics.

\end{justification}

\begin{goals}
\item Capacidad para aplicar los conocimientos de Ingeniería .
\item Capacidad de comunicarse por escrito.
\item Capacidad para resolver problemas de Ingeniería .

% - a3: Capacity to apply knowledge of Engineering (level 1);
% - g2: Capacity to communicate in writing (level 1);
% - e3: Capacity to solve Engineering problems (level 1).

\end{goals}

\begin{outcomes}
\ShowOutcome{a}{3}
\ShowOutcome{i}{3}
\ShowOutcome{j}{3}
\end{outcomes}

\begin{unit}{FI1 Introducción}{Serway2002,Alonso94}{0}{3}
\begin{topics}
      \item Importancia de la termodinámica para las ciencias de la Ingeniería .
      \item Concepto de equilibro (térmica, mecánica y química).
      
% Importance of Thermodynamics for Engineering Sciences Concept of equilibrium (thermal, mechanical and
% chemical)

   \end{topics}

   \begin{learningoutcomes}
      \item Resolver problemas.
   \end{learningoutcomes}
\end{unit}

\begin{unit}{FI1 Propiedades termofísicas}{}{0}{3}
\begin{topics}
      \item Evaluación de las propiedades de las sustancias puras.
      \item Ecuaciones de estado y evaluación de las propiedades mediante el modelo de los gases ideales.
      \item Sistemas cerrados, volúmenes de control y sistemas abiertos.
      \item Estado de las funciones (energía interna, entalpía y entropía).
  
  
% Evaluating properties of pure substances
% Equations of state; evaluating properties using the Ideal Gas Model
% Evaluating properties using computer software (EES)
% Use of EES (Engineering Equation Solver) to compute properties 
% Closed systems, control volumes, open Systems
% State functions (internal energy, enthalpy and entropy)
  
  
   \end{topics}

   \begin{learningoutcomes}
      \item Resolver problemas.
   \end{learningoutcomes}
\end{unit}

\begin{unit}{FI1 1ra Ley y Procesos}{}{0}{3}
\begin{topics}
      \item 1ra Ley de la Termodinámica.
      \item Balance de masa y energía de volúmenes de control.
      
% 1st Law of Thermodynamics
% Mass and energy balances of control volumes
      
      
   \end{topics}

   \begin{learningoutcomes}
      \item Resolver problemas.
   \end{learningoutcomes}
\end{unit}

\begin{unit}{FI1 2da Ley y Sistemas}{}{0}{3}
\begin{topics}
      \item Motores térmicos. Ciclo de Carnot.
      \item 2da Ley de la Termodinámica.
      \item Eficiencia térmica. Coeficiente de rendimiento (COP).
      \item Entropía.
      
% Heat engines. The Carnot cycle.
% 2nd Law of Thermodynamics.
% Thermal efficiency. Coefficient of Performance (COP)
% Entropy
%       
   \end{topics}

   \begin{learningoutcomes}
      \item Resolver problemas.
   \end{learningoutcomes}
\end{unit}

\begin{unit}{FI1 Conversión de Energía}{}{0}{3}
\begin{topics}
      \item Conceptos básicos sobre los ciclos de potencia: Otto, Diesel, Brayton y Rankine.
      \item Ciclos combinados.
      \item Concept básicos sobre los ciclos de refrigeración y bomba de calor.
     
% Basics on power cycles: Otto, Diesel, Brayton \& Rankine
% Combined cycles. Optimization of thermal power plant
% Basics on refrigeration and heat pump cycles
     
   \end{topics}

   \begin{learningoutcomes}
      \item Resolver problemas.
   \end{learningoutcomes}
\end{unit}



\begin{coursebibliography}
\bibfile{BasicSciences/CB111}
\end{coursebibliography}

\end{syllabus}
