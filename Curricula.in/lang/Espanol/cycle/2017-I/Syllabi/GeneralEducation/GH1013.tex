\begin{syllabus}

\course{GH1013. Ciencia, Arte y Sociedad}{Obligatorio}{GH1013} % Common.pm

\begin{justification}
El desarrollo del curso obedece a tres objetivos: primero, comprender la modernidad desde tres aspectos: científico, social y artístico-cultural; segundo, revisar los conceptos fundamentales que constituyeron el imaginario moderno desde la perspectiva de sus obras culturales, científicas e institucionales; y, tercero, discutir y dialogar en torno a las
actuales críticas a la modernidad y sus presupuestos desde las nuevas teorías filosóficas, sociológicas y artísticas del siglo XX y siglo XXI. Estos tres objetivos se organizan alrededor de los tres horizontes que conforman la intención principal de nuestro curso: el
horizonte científico moderno, el horizonte socio-político y el horizonte artístico. Desde este triple horizonte expondremos, dialogaremos y participaremos en la reconstrucción
del imaginario y de la mentalidad moderna, cuyos aportes y desarrollos múltiples constituyen e impactan aún en nuestro imaginario contemporáneo. En efecto, no nos es posible interpretar nuestra actualidad y sus consecuencias si no es a partir de los productos sociales, científicos y artísticos proyectados desde la modernidad. Sin embargo, nuestro curso no solo se limita a la revisión histórica; además de ello ofrece un
componente esencial en cualquier comprensión histórica: el análisis y la revisión de los límites y alcances de la modernidad. Desde esta revisión entenderemos mejor las
consecuencias contemporáneas de los proyectos modernos y cómo ellas, o fueron superadas, o fueron continuadas o accedieron a la reelaboración de sus intenciones desde una perspectiva no moderna, sino postmoderna.
\end{justification}

\begin{goals}
\item Identificar los conceptos fundamentales de la Modernidad
\item Comprender la situación histórico-cultural del siglo XX y XXI denominada Postmodernidad.
\end{goals}

\begin{outcomes}
    \item \ShowOutcome{d}{2} % Multidiscip teams
    \item \ShowOutcome{e}{2} % ethical, legal, security and social implications
    \item \ShowOutcome{f}{2} % communicate effectively
    \item \ShowOutcome{n}{2} % Apply knowledge of the humanities
    \item \ShowOutcome{p}{2} % TASDSH
\end{outcomes}

\begin{competences}
    \item \ShowCompetence{C10}{d,n,o}
    \item \ShowCompetence{C17}{f}
    \item \ShowCompetence{C18}{f}
    \item \ShowCompetence{C21}{e}
\end{competences}

\begin{unit}{Crítica de la Modernidad}{}{Danto99}{12}{4}
   \begin{topics}
      \item ?`Qué significa ser moderno?.
      \item Las Revoluciones Científicas.
      \item El Modelo Matemático De Naturaleza: Copernico (El Matematización) y Newton (Causa-Efecto).
      \item Las nuevas Revoluciones Científicas: El Apogeo del Siglo XX.
      \item La Ciencia desde la Vida.
      \item Naturaleza y Arte.
      \item Nacimiento del Sujeto.
      \item Individuo vs. Sociedad.
      \item Desenmascarar La Sociedad: Freud, Nietzsche y Freud.
      \item Baudelaire Y Benjamin: Estéticas desde la Urb.
      \item El Fin del Arte: De Urinarios y Ruedas De Bicicletas.
   \end{topics}
   \begin{learningoutcomes}
      \item Valorar y revisar críticamente los desarrollos artísticos y culturales en general de la Modernidad desde consideraciones contemporáneas.
   \end{learningoutcomes}
\end{unit}



\begin{coursebibliography}
\bibfile{GeneralEducation/GH1013}
\end{coursebibliography}

\end{syllabus}
