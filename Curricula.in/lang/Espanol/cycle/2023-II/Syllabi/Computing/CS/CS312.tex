\begin{syllabus}

\course{CS3102. Estructuras de Datos Avanzadas}{Obligatorio}{CS3102}
% Source file: ../Curricula.in/lang/Espanol/cycle/2020-I/Syllabi/Computing/CS/CS312.tex

\begin{justification}
Los algoritmos y estructuras de datos son una parte fundamental de la ciencia de la computación que nos 
permiten organizar la información de una manera más eficiente, por lo que es importante para todo 
profesional del área tener una sólida formación en este aspecto.

En el curso de estructuras de datos avanzadas nuestro objetivo es que el alumno conozca y analize 
estructuras complejas, como los Métodos de Acceso Multidimensional, 
Métodos de Acceso Espacio-Temporal y Métodos de Acceso Métrico, etc.
\end{justification}

\begin{goals}
\item Que el alumno entienda, diseñe, implemente, aplique y
proponga estructuras de datos innovadoras para solucionar
problemas relacionados al tratamiento de datos multidimensionales,
recuperación de información por similitud, motores de búsqueda y
otros problemas computacionales.
\end{goals}



\begin{unit}{Multidimensional Data}{}{Cuadros2004Implementing,Knuth2007TAOCP-V-I,Knuth2007TAOCP-V-II,Gamma94}{16}{a,b,c,1,2}
   \begin{topics}
         \item Introducción al curso.
         \item Introducción a datos multidimensionales. 
         \item Maldición de la dimensionalidad.
         
   \end{topics}
   \begin{learningoutcomes}
         \item Introducir la transcendencia de la representación multidimensional de datos. [\Usage]
         \item Entender la complejidad de lidiar con datos multidimensional y de alta dimensión.[\Usage]
         \item Entender la maldición de la dimensionalidad, y su impacto en el indizado de grandes volúmenes de datos.[\Usage]
         \item Presentar y discutir aplicaciones reales de datos multidimensionales en motores de búsqueda.[\Usage]
   \end{learningoutcomes}
\end{unit}

\begin{unit}{Multidimensional Acces Data Structures}{}{Cuadros2004Implementing,Knuth2007TAOCP-V-I,Knuth2007TAOCP-V-II,Gamma94}{16}{a,b,c,1,2}
   \begin{topics}
         \item Introducción a estructuras de datos espaciales.
         \item Estructuras espaciales, Quadtree,Octree y visualización.
         \item Kd-Tree.
         \item Introducción a R-Tress.
         \item R tree (Guttman).
         \item R+ tree.
         \item R* tree.
         \item Variación R*-tree y relación con paginación y tamaño de bloques.
         \item X-tree.
   \end{topics}
   \begin{learningoutcomes}
         \item Introducir los fundamentos teóricos de estructuras de datos espaciales.
         \item Entender los beneficios y limitaciones deestructuras de datos espaciales basadas en árbol.
         \item Implementar diferentes estructuras de datos para el indizado de grandes volumenes de datos.
         \item Entender los fundamentos e implementar estrategias de búsqueda como vecinos mas próximos y búsquedas por rango.
   \end{learningoutcomes}
\end{unit}

\begin{unit}{Approximate Access Methods}{}{Cuadros2004Implementing,Knuth2007TAOCP-V-I,Knuth2007TAOCP-V-II,Gamma94}{20}{a,b,c,1,2}
   \begin{topics}
         \item Métodos de Acceso Métrico para distancias discretas
         \item Métodos de Acceso Métrico para distancias continuas
   \end{topics}
   \begin{learningoutcomes}
         \item Que el alumno entienda conozca e implemente algunos métodos de acceso métrico[\Usage]
         \item Que el alumno entienda la importancia de estos Métodos de Acceso para la Recuperación de Información por Similitud[\Usage]
   \end{learningoutcomes}
\end{unit}

\begin{unit}{Métodos de Acceso Aproximados}{}{Samet2004SAM-MAM,Indyk06,Zezula07}{20}{a,b,c,1,2}
   \begin{topics}
         \item Space Filling Curves: Hilbert curve y Z-order
         \item Proyecciones y complejidad.
         \item Locally sensitive hashing (LSH)
   \end{topics}
   \begin{learningoutcomes}
         \item Entender, conocer e implementar algunos métodos de acceso aproximados.
         \item Entender la importancia de estos métodos de Acceso para la recuperación de información por similitud en entornos donde la escalabilidad sea una factor muy importante.
   \end{learningoutcomes}
\end{unit}

\begin{unit}{Clustering}{}{Cuadros2004Implementing,Knuth2007TAOCP-V-I,Knuth2007TAOCP-V-II,Gamma94}{8}{a,b,c,1,2}
	\begin{topics}
         \item Introducción a Clustering.
         \item Kmeans y DBScan.
         \item Clustering Applications.
         \item Clustering Ensemble.
   \end{topics}
   \begin{learningoutcomes}
         \item Introducir los fundamentos teóricos para el clustering de datos multidimensionales.
         \item Implementar diferentes estrategias para el clustering de datos multidimensionales, como basados en partición, en jerarquía o en densidad.
         \item Entender los fundamentos, aplicaciones e implementar ensambles de métodos de clustering.
         \item Implementar ensambles de métodos de clustering con datos reales.
   \end{learningoutcomes}
\end{unit}

\begin{unit}{Temporal Data Structures}{}{Cuadros2004Implementing,Knuth2007TAOCP-V-I,Knuth2007TAOCP-V-II,Gamma94}{8}{a,b,c,1,2}
   \begin{topics}\
         \item Introducción a Estructuras de datos temporales.
         \item Versionando la estructura de Datos.
         \item Persistencia
         \item Retroactividad
   \end{topics}
   \begin{learningoutcomes}
         \item Introducir los fundamentos teóricos de estructuras de datos temporales.
         \item Entender, discutir e implementar Persistencia y sus tipos.
         \item Entender, discutir e implementar Retroactividad y sus tipos.
         \item Entender y discutir los beneficios y  limitaciones entre persistencia y retroactividad.
   \end{learningoutcomes}
\end{unit}

\begin{unit}{Final Talks}{}{SCuadros2004Implementing,Knuth2007TAOCP-V-I,Knuth2007TAOCP-V-II,Gamma94}{8}{a,b,c,1,2}
	\begin{topics}
         \item Seminarios de trabajo de investigación.
   \end{topics}
   \begin{learningoutcomes}
         \item Investigar sobre nuevos métodos para el indizado de grandes volumenes de datos complejos.
         \item Presentar y dirigir la discusión sobre métodos para indizados de Big Data investigado.
   \end{learningoutcomes}
\end{unit}





\begin{coursebibliography}
\bibfile{Computing/CS/CS312}
\end{coursebibliography}

\end{syllabus}
