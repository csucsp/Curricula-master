\begin{syllabus}

\course{CS100. Introducción de Ciencia de la Computación}{Obligatorio}{CS100}
% Source file: ../Curricula.in/lang/Espanol/cycle/2020-I/Syllabi/Computing/CS/CS100.tex

\begin{justification}
La Ciencia de la Computación es un campo de estudio enorme con muchas especialidades y aplicaciones. Este curso brindará a sus participantes, una visión panorámica de la informática y mostrará sus campos más representativos, como son: Algoritmos, Estructuras de de Datos, Sistemas Operativos, Bases de Datos, etc.
\end{justification}

\begin{goals}
\item Brindar un panorama del área del conocimiento que es cubierta en la ciencia de la computación.
\end{goals}


\begin{unit}{Introducción}{}{brookshear}{2}{a,b,c,1,2}
    \begin{topics}
	\item Introducción a la computación.
	\item Historia de la computación.
   \end{topics}
   \begin{learningoutcomes}
      \item Incentivar a los alumnos el estudio de Computacion como una ciencia. [\Familiarity]
   \end{learningoutcomes}
\end{unit}

\begin{unit}{Almacenamiento de datos}{Almacenamiento de datos}{brookshear2014}{2}{a,b,c,h,1,2,5}
   \begin{topics}
      \item \DSBasicLogicTopicPropositional
      \item \DSBasicLogicTopicLogical
      \item \DSBasicLogicTopicTruth
      \item \DSBasicLogicTopicNormal
      \item \ARMachinelevelrepresentationofdataTopicBits
      \item \ARMachinelevelrepresentationofdataTopicNumeric
   \end{topics}
   \begin{learningoutcomes}
      \item \DSBasicLogicLOConvertLogical[\Familiarity]
      \item \DSBasicLogicLOApplyFormal [\Familiarity]
      \item \ARMachinelevelrepresentationofdataLOExplainWhyData [\Familiarity].
      \item \ARMachinelevelrepresentationofdataLOExplainTheUsing [\Familiarity].
   \end{learningoutcomes}
\end{unit}
\begin{unit}{Tratamiento de Datos}{Tratamiento de Datos}{brookshear2014}{2}{a,b,c,h,1,2,5}
   \begin{topics}
     \item \ARDigitallogicanddigitalsystemsTopicOverview
     \item \ARAssemblylevelmachineorganizationTopicBasic
     \item \ARAssemblylevelmachineorganizationTopicControl
     \item \ARAssemblylevelmachineorganizationTopicInstruction
     \item \ARAssemblylevelmachineorganizationTopicAssembly
     \item \ARMemorysystemorganizationandarchitectureTopicMain
   \end{topics}
   \begin{learningoutcomes}
      \item \DSBasicLogicLOConvertLogical[\Familiarity]
      \item \DSBasicLogicLOApplyFormal [\Familiarity]
      \item \ARAssemblylevelmachineorganizationLOExplainTheTheNeumann [\Familiarity].
      \item \ARAssemblylevelmachineorganizationLODescribeHowIs [\Familiarity].
   \end{learningoutcomes}
\end{unit}

%Sistemas Operativos
\begin{unit}{\OSOverviewofOperatingSystems}{}{brookshear2014}{2}{a,b,c,h,1,2,5}
   \begin{topics}
      \item \OSOverviewofOperatingSystemsTopicRole
      \item \OSOverviewofOperatingSystemsTopicFunctionality
      \item \OSSchedulingandDispatchTopicProcesses
      \item \OSSchedulingandDispatchTopicSchedulers
   \end{topics}
   \begin{learningoutcomes}
      \item \OSOverviewofOperatingSystemsLOExplainTheFunctions [\Familiarity]
   \end{learningoutcomes}
\end{unit}

%Redes e Internet
\begin{unit}{Redes e Internet}{Redes e Internet}{brookshear2014}{2}{a,b,c,h,1,2,5}
   \begin{topics}
      \item \NCIntroductionTopicOrganization
      \item \NCIntroductionTopicPhysical
      \item \NCNetworkedApplicationsTopicNaming
   \end{topics}
   \begin{learningoutcomes}
      \item \NCIntroductionLOArticulateThe [\Familiarity].
      \item \NCIntroductionLOListAndAppropriate  [\Familiarity].
      \item \NCIntroductionLODescribeTheOfNetworked [\Familiarity].
   \end{learningoutcomes}
\end{unit}

%Algoritmos
\begin{unit}{Algoritmos}{Algoritmos}{brookshear2014}{2}{a,b,c,h,1,2,5}
   \begin{topics}
      \item \SDFAlgorithmsandDesignTopicThe
      \item \SDFAlgorithmsandDesignTopicTheRole
      \item \SDFAlgorithmsandDesignTopicProblem
      \item \SDFAlgorithmsandDesignTopicFundamental
   \end{topics}
   \begin{learningoutcomes}
      \item \SDFAlgorithmsandDesignLODiscussTheAlgorithms [\Familiarity].
      \item \SDFAlgorithmsandDesignLODiscussHowMay [\Familiarity].
      \item \SDFAlgorithmsandDesignLOCreateAlgorithms [\Familiarity].
   \end{learningoutcomes}
\end{unit}

%Lenguajes de programcion

\begin{unit}{Lenguajes de Programacion}{Lenguajes de Programacion}{brookshear2014}{2}{a,b,c,h,1,2,5}
   \begin{topics}
      \item \PLProgramRepresentationTopicPrograms
      \item \PLProgramRepresentationTopicAbstract
      \item \PLProgramRepresentationTopicData
   \end{topics}
   \begin{learningoutcomes}
      \item \PLProgramRepresentationLOExplainHowProcess [\Familiarity].
      \item \PLProgramRepresentationLODescribeAnTree [\Familiarity].
      \item \PLProgramRepresentationLODescribeTheHaving [\Familiarity].
   \end{learningoutcomes}
\end{unit}

%Ingenieria del Software
\begin{unit}{\SESoftwareProcesses}{}{brookshear2014}{2}{a,b,c,h,1,2,5}
   \begin{topics}
      \item \SESoftwareProcessesTopicIntroduction
      \item \SESoftwareProcessesTopicSystem
      \item \SESoftwareProcessesTopicSoftware
   \end{topics}
   \begin{learningoutcomes}
      \item \SESoftwareProcessesLODescribeHowInteract [\Familiarity].
      \item \SESoftwareProcessesLODescribeTheAndSeveral [\Familiarity].
   \end{learningoutcomes}
\end{unit}

% Cap Estructura de datos
\begin{unit}{\ALFundamentalDataStructuresandAlgorithms}{}{brookshear2014}{2}{a,b,c,h,1,2,5}
   \begin{topics}
    \item \SDFFundamentalDataStructuresTopicAbstract
    \item \SDFFundamentalDataStructuresTopicLinked
    \item \ALFundamentalDataStructuresandAlgorithmsTopicSimple
    \item \ALFundamentalDataStructuresandAlgorithmsTopicSequential
    \item \ALFundamentalDataStructuresandAlgorithmsTopicBinary
    \item \ALAdvancedDataStructuresAlgorithmsandAnalysisTopicBalanced
    \item \ALAdvancedDataStructuresAlgorithmsandAnalysisTopicAdvanced
   \end{topics}
   \begin{learningoutcomes}
     \item \ALFundamentalDataStructuresandAlgorithmsLOImplement [\Familiarity].
     \item \ALFundamentalDataStructuresandAlgorithmsLOImplementSimple [\Familiarity].
     \item \ALFundamentalDataStructuresandAlgorithmsLODiscussFactors [\Familiarity].
   \end{learningoutcomes}
\end{unit}

%Base de datos
\begin{unit}{Base de Datos}{Base de Datos}{brookshear2014}{2}{a,b,c,h,1,2,5}
   \begin{topics}
      \item \IMDatabaseSystemsTopicApproaches
      \item \IMDatabaseSystemsTopicComponents
      \item \IMDatabaseSystemsTopicDesign
      \item \IMDatabaseSystemsTopicDatabase
      \item \IMDatabaseSystemsTopicUse
   \end{topics}
   \begin{learningoutcomes}
      \item \IMDatabaseSystemsLOExplainTheDistinguish [\Familiarity].
      \item \IMDatabaseSystemsLODescribeTheDesigns [\Familiarity].
   \end{learningoutcomes}
\end{unit}

%Inteligencia Artificial
\begin{unit}{Inteligencia Artificial}{Inteligencia Artificial}{brookshear2014}{2}{a,b,c,h,1,2,5}
   \begin{topics}
      \item \ISFundamentalIssuesTopicOverview
      \item \ISFundamentalIssuesTopicWhat
      \item \ISAdvancedMachineLearningTopicDefinition
      \item \ISAdvancedMachineLearningTopicSupervised
   \end{topics}
   \begin{learningoutcomes}
      \item \ISFundamentalIssuesLODeterming [\Familiarity].
   \end{learningoutcomes}
\end{unit}

%Computacion Grafica
\begin{unit}{Computación Gráfica}{Computación Gráfica}{brookshear2014}{2}{a,b,c,h,1,2,5}
   \begin{topics}
      \item \GVFundamentalConceptsTopicMedia
      \item \GVFundamentalConceptsTopicDigitization
      \item \CNInteractiveVisualizationTopicGraphing
      \item \CNInteractiveVisualizationTopicImage
   \end{topics}
   \begin{learningoutcomes}
      \item \GVFundamentalConceptsLOIdentifyCommon [\Familiarity].
      \item \GVFundamentalConceptsLOExplainIn [\Familiarity].
   \end{learningoutcomes}
\end{unit}


\begin{coursebibliography}
\bibfile{Computing/CS/CS100}
\end{coursebibliography}
\end{syllabus}
%\end{document}
