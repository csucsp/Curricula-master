\begin{syllabus}

\course{CS1D01. Estructuras Discretas I}{Obligatorio}{CS1D01}
% Source file: ../Curricula.in/lang/Espanol/cycle/2020-I/Syllabi/Computing/CS/CS1D1.tex

\begin{justification}
Las estructuras discretas proporcionan los fundamentos teóricos necesarios para la computación.
Dichos fundamentos no son sólo útiles para desarrollar la computación desde un punto de vista teórico
como sucede en el curso de teoría de la computación,  sino que también son útiles para la práctica de la
computación; en particular se aplica en áreas como verificación, criptografía, métodos formales, etc.
\end{justification}

\begin{goals}
\item Aplicar adecuadamente conceptos de la matemática finita (conjuntos, relaciones, funciones) para representar datos de problemas reales.
\item Modelar situaciones reales descritas en lenguaje natural, usando lógica proposicional y lógica de predicados.
\item Aplicar el método de demostración más adecuado para determinar la veracidad de un enunciado.
\item Construir argumentos matemáticos correctos.
\item Interpretar las soluciones matemáticas para un problema y determinar su fiabilidad, ventajas y desventajas.
\item Expresar el funcionamiento de un circuito electrónico simple usando conceptos del Álgebra de Boole.
\end{goals}

\begin{outcomes}{V2}
    \item \ShowOutcome{1}{3}
    \item \ShowOutcome{6}{3}
\end{outcomes}
\begin{competences}{V2}
    \item \ShowCompetence{C1}{1}
    \item \ShowCompetence{C20}{6}
\end{competences}

\begin{unit}{\DSBasicLogic}{}{Rosen2012,Grimaldi03,howToProve}{14}{a,j,1,6}
   \begin{topics}
        \item \DSBasicLogicTopicPropositional%
        \item \DSBasicLogicTopicLogical%
        \item \DSBasicLogicTopicTruth%
        \item \DSBasicLogicTopicNormal%
        \item \DSBasicLogicTopicValidity%
        \item \DSBasicLogicTopicPropositionalInference%
        \item \DSBasicLogicTopicPredicate%
        \item \DSBasicLogicTopicLimitations%
   \end{topics}
   \begin{learningoutcomes}
	\item \DSBasicLogicLOConvertLogical [\Usage ]
	\item \DSBasicLogicLOApplyFormal [\Usage ]
	\item \DSBasicLogicLOUseThe [\Usage]
	\item \DSBasicLogicLODescribeHowCan [\Familiarity]
	\item \DSBasicLogicLOApplyFormalAnd [\Usage ]
	\item \DSBasicLogicLODescribeTheLimitationsAnd [\Usage]
   \end{learningoutcomes}
 \end{unit}

\begin{unit}{\DSProofTechniques}{}{Rosen2012,Epp10, Scheinerman12}{14}{a,j,1,6}
\begin{topics}
        \item \DSProofTechniquesTopicNotions%
        \item \DSProofTechniquesTopicThe%
        \item \DSProofTechniquesTopicDirect%
        \item \DSProofTechniquesTopicDisproving%
        \item \DSProofTechniquesTopicProof%s
        \item \DSProofTechniquesTopicInduction%
        \item \DSProofTechniquesTopicStructural%
        \item \DSProofTechniquesTopicWeak%
        \item \DSProofTechniquesTopicRecursive%
        \item \DSProofTechniquesTopicWell%
\end{topics}

\begin{learningoutcomes}
    %% itemizar cada learning outcomes [nivel segun el curso]
	\item \DSProofTechniquesLOIdentifyTheUsed [\Assessment]
	\item \DSProofTechniquesLOOutline [\Usage ]
	\item \DSProofTechniquesLOApplyEach [\Usage ]
	\item \DSProofTechniquesLODetermineWhich [\Assessment]
	\item \DSProofTechniquesLOExplainTheIdeas [\Familiarity ]
	\item \DSProofTechniquesLOExplainTheWeak [\Assessment]
	\item \DSProofTechniquesLOStateThe [\Familiarity]
\end{learningoutcomes}
\end{unit}

\begin{unit}{\DSSetsRelationsandFunctions}{}{}{Grimaldi03,Rosen2012}{22}{a,j,1,6}
   \begin{topics}
        \item \DSSetsRelationsandFunctionsTopicSets
         \item \DSSetsRelationsandFunctionsTopicRelations
         \item \DSSetsRelationsandFunctionsTopicFunctions
   \end{topics}
   \begin{learningoutcomes}
	\item \DSSetsRelationsandFunctionsLOExplainWith [\Assessment]
	\item \DSSetsRelationsandFunctionsLOPerformThe [\Assessment]
	\item \DSSetsRelationsandFunctionsLORelate [\Assessment]
   \end{learningoutcomes}
\end{unit}
 

\begin{unit}{Fundamentos de Lógica Digital}{}{Rosen2012,Grimaldi03}{10}{a,j,1,6}
   \begin{topics}
	\item Órdenes Parciales y Conjuntos Parcialmente Ordenados.
 	\item Elementos extremos de un Conjunto Parcialmente Ordenado.
	\item Retículas: Tipos y propiedades.
	\item Álgebras Booleanas
	\item Funciones y expresiones Boolenas
	\item Representación de Funciones Booleanas: Forma Normal Disyuntiva y Conjuntiva
	\item Puertas Lógicas
	\item Minimización de funciones booleanas.
   \end{topics}
   \begin{learningoutcomes}
	\item Explicar la importancia del Álgebra de Boole como unificación de la Teoría de Conjuntos y la Lógica Proposicional [\Familiarity].
	\item Demostrar enunciados usando el concepto de retícula y sus propiedades[\Assessment].
	\item Explicar la relación entre retícula y conjunto parcialmente ordenado [\Familiarity].
	\item Demostrar para una terna formada por un conjunto y dos operaciones internas, si cumple las propiedades de una Álgebra de Boole [\Assessment].
	\item Representar una función booleana en sus formas canónicas[\Usage].
	\item Representar una función booleana como un circuito booleano usando puertas lógicas  [\Usage].
	\item Minimizar una función booleana [\Usage].
    \end{learningoutcomes}
 \end{unit}

\begin{coursebibliography}
\bibfile{Computing/CS/CS1D1}
\end{coursebibliography}

\end{syllabus}

%\end{document}
