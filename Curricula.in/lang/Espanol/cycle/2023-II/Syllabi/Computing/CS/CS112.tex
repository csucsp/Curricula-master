\begin{syllabus}

\course{CS1102. Programación Orientada a Objetos I}{Obligatorio}{CS1102}
% Source file: ../Curricula.in/lang/Espanol/cycle/2020-I/Syllabi/Computing/CS/CS112.tex

\begin{justification}
Este es el segundo curso en la secuencia de los cursos introductorios a la Ciencia de la Computación.
El curso introducirá a los participantes en los diversos temas del área de computación como: algoritmos, estructuras de datos, ingeniería del software, etc.
\end{justification}

\begin{goals}
\item Introducir al alumno a los fundamentos del paradigma de orientación a objetos, permitiendo asimilar los conceptos necesarios para desarrollar sistemas de información.
\end{goals}



\begin{unit}{Visión General de los Lenguajes de Programación}{}{Stroustrup2013,Deitel17}{1}{a,1}
    \begin{topics}
        \item Breve revisión de los paradigmas de programación.
        \item Comparación entre programación funcional y programación imperativa.
        \item Historia de los lenguajes de programación.
    \end{topics}
    \begin{learningoutcomes}
        \item \SPHistoryLODiscussTheForLanguage [\Familiarity]
    \end{learningoutcomes}
\end{unit}

\begin{unit}{\OSVirtualMachines}{}{Stroustrup2013,Deitel17}{2}{a,b,i,1,6}
    \begin{topics}
        \item El concepto de máquina virtual.
        \item \OSVirtualMachinesTopicTypes.
        \item Lenguajes intermedios.
    \end{topics}
    \begin{learningoutcomes}
        \item \OSVirtualMachinesLOExplainTheVirtual [\Familiarity]
        \item \OSVirtualMachinesLODifferentiateEmulation [\Familiarity]
        \item \OSVirtualMachinesLOEvaluateVirtualization [\Assessment]
    \end{learningoutcomes}
\end{unit}

\begin{unit}{\PLBasicTypeSystems}{}{Stroustrup2013,Deitel17}{6}{a,b,i,1,6}
    \begin{topics}
        \item \PLBasicTypeSystemsTopicA        
        \item Declaración de modelos (enlace, visibilidad, alcance y tiempo de vida).
        \item Vista general del chequeo de tipos.
    \end{topics}
    \begin{learningoutcomes}
        \item \PLBasicTypeSystemsLOForBoth [\Familiarity] 
        \item \PLBasicTypeSystemsLOForA [\Familiarity] 
        \item \PLBasicTypeSystemsLODescribeExamples [\Familiarity] 
        \item \PLBasicTypeSystemsLOForMultiple [\Usage] 
        \item \PLBasicTypeSystemsLOGiveAnThat [\Familiarity] 
        \item \PLBasicTypeSystemsLOUseTypes [\Usage] 
        \item \PLBasicTypeSystemsLOExplainHowDefine [\Familiarity] 
        \item \PLBasicTypeSystemsLOWriteDown [\Usage] 
        \item \PLBasicTypeSystemsLOExplainWhyType [\Familiarity] 
        \item \PLBasicTypeSystemsLODefineAndPieces [\Usage] 
        \item \PLBasicTypeSystemsLODiscussTheGenerics [\Familiarity] 
        \item \PLBasicTypeSystemsLOExplainMultiple [\Familiarity] 
    \end{learningoutcomes}
\end{unit}

\begin{unit}{\SDFFundamentalProgrammingConcepts}{}{Stroustrup2013,Deitel17}{10}{a,b,i,1,6}
    \begin{topics}
        \item \PLObjectOrientedProgrammingTopicObject\item \PLBasicTypeSystemsLOForBoth [\Familiarity] 
        \item \SDFFundamentalProgrammingConceptsTopicBasic
        \item \SDFFundamentalProgrammingConceptsTopicVariables
        \item \SDFFundamentalProgrammingConceptsTopicExpressions
        %\item \SDFFundamentalProgrammingConceptsTopicSimple
        \item \SDFFundamentalProgrammingConceptsTopicConditional
        %\item \SDFFundamentalProgrammingConceptsTopicFunctions

    \end{topics}
    \begin{learningoutcomes}
        \item \SDFFundamentalProgrammingConceptsLOAnalyzeAndBehavior [\Assessment]
        \item \SDFFundamentalProgrammingConceptsLOIdentifyAndOf [\Familiarity]
        \item \SDFFundamentalProgrammingConceptsLOWritePrograms [\Usage]
        \item \SDFFundamentalProgrammingConceptsLOModify [\Usage]
        \item \SDFFundamentalProgrammingConceptsLODesignImplement [\Usage]
        %\item \SDFFundamentalProgrammingConceptsLOWriteAUses [\Usage]
        \item \SDFFundamentalProgrammingConceptsLOChooseAppropriateIteration [\Assessment]
        %\item \SDFFundamentalProgrammingConceptsLODescribeTheRecursion [\Familiarity]
        %\item \SDFFundamentalProgrammingConceptsLOIdentifyTheAndCase [\Assessment]

    \end{learningoutcomes}
\end{unit}

\begin{unit}{Funciones}{}{Stroustrup2013,Deitel17}{3}{a,b,i,1,6}
    \begin{topics}
        \item \SDFFundamentalProgrammingConceptsTopicFunctions
        \item Paso de parámetros
        \item Sobrecarga en funciones
        \item Fundamentos de la recursidad 
        \item Conceptos de plantillas en funciones
    \end{topics}
    \begin{learningoutcomes}
        \item  \SDFFundamentalProgrammingConceptsLODesignImplement [\Usage]
        \item  Entiende y aplica el concepto de paso de parámetros a una función, tanto por valor como por referencia.[\Usage]
        \item  Identifica y aplica el concepto de sobrecarga de funciones.[\Usage]
        \item  \SDFFundamentalProgrammingConceptsLODescribeTheRecursion [\Familiarity]
        \item  Diseña, implementa y aplica el concpeto de plantillas asociado a la ncesidad de crear funciones genéricas.[\Usage]

    \end{learningoutcomes}
\end{unit}

\begin{unit}{Arreglos y Punteros}{}{Stroustrup2013,Deitel17}{3}{a,b,i,1,6}
    \begin{topics}
       \item Definición de arreglos
        \item Arreglos multidimensionales
        \item Fundamentos sobre punteros
        \item Administración dinámica de memoria 
        \item Conceptos avanzados de Punteros 
    \end{topics}
    \begin{learningoutcomes}
       \item  Entiende e implementa arreglos unidimensionales. [\Familiarity]
        \item  Diseña y aplica el concepto de arreglos multidimensionales.[\Usage]
        \item  Entiende y aplica el concepto de referencias y punteros.[\Familiarity]
        \item  Entiende, aplica y evalua la relación entre punteros y arreglos.[\Assessment]
        \item  Entiende e implementa la gestión dinámica de la memoria. Diferenciando las regiones de memoria: heap y stack. [\Assessment]
        \item  Diseña, implementa y evalua el concepto de puntero a puntero, puntero a función, entre otros conceptos.[\Assessment]
    \end{learningoutcomes}
\end{unit}

\begin{unit}{\PLObjectOrientedProgramming}{}{Stroustrup2013,Deitel17}{2}{a,b,i,1,6}
    \begin{topics}
        \item \PLObjectOrientedProgrammingTopicObject
        \item \PLObjectOrientedProgrammingTopicObjectOriented
        \item \PLObjectOrientedProgrammingTopicDefinition
        \item \PLObjectOrientedProgrammingTopicSubclasses
        \item \PLObjectOrientedProgrammingTopicSubtyping
        \item \PLObjectOrientedProgrammingTopicUsing
        \item \PLObjectOrientedProgrammingTopicDynamic
    \end{topics}
    \begin{learningoutcomes}
        \item \PLObjectOrientedProgrammingLODesignAndClass [\Usage]
        \item \PLObjectOrientedProgrammingLOUseSubclassing [\Usage]
        \item \PLObjectOrientedProgrammingLOCorrectly [\Usage]
        \item \PLObjectOrientedProgrammingLOCompareAndThe [\Assessment]
        \item \PLObjectOrientedProgrammingLOExplainTheObject [\Familiarity]
        \item \PLObjectOrientedProgrammingLOUseObject [\Usage]
        \item \PLObjectOrientedProgrammingLODefineAndAnd [\Usage]
    \end{learningoutcomes}
\end{unit}

\begin{unit}{Plantillas y STL}{}{Stroustrup2013,Deitel2017}{2}{a,b,i,1,6}
    \begin{topics}%
        \item Definición de plantillas en clases
        \item Conceptos básicos sobre la Standard Template Library (STL)
      
    \end{topics}
    \begin{learningoutcomes}
        \item  Entiende los conceptos de plantillas en clases. [\Familiarity]
        \item  Implementa y crea nuevos tipos de datos genéricos. [\Usage]
        \item  Entiende las estructuras básicas de la STL. [\Familiarity]
        \item  Usa las estructuras de datos básicas como: pila, cola, lista, vector contenidos en la STL. [\Usage]

    \end{learningoutcomes}
\end{unit}


\begin{unit}{Conceptos Avanzados}{}{Stroustrup2013,Deitel2017}{2}{a,b,i,1,6}
    \begin{topics}%
        \item Definición de sobrecarga de operadores
        \item Manipulación de entrada y salida de datos (I/O)
        \item Patrones de diseño
      
    \end{topics}
    \begin{learningoutcomes}
        \item  Entiende los conceptos de sobrecarga de operadores. [\Familiarity]
        \item  Implementa la sobrecarga de operadores permitidos en el lenguaje de programación. [\Usage]
        \item  Entiende los conceptos de manipulación de archivos. [\Familiarity]
        \item  Crea programas de lectura y escrita en archivos. [\Usage]
        \item  Entiende los conceptos de patrones de diseño. [\Familiarity]

    \end{learningoutcomes}
\end{unit}

\begin{coursebibliography}
   \bibfile{Computing/CS/CS112}
\end{coursebibliography}

\end{syllabus}
