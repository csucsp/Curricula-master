\begin{syllabus}

\begin{justification}
Este es el primer curso en la secuencia de los cursos introductorios a la Ciencia de la Computación. 
En este curso se pretende cubrir los conceptos señalados por la Computing Curricula IEEE-CS/ACM 2013.
La programación es uno de los pilares de la Ciencia de la Computación; cualquier profesional del Área, necesitará programar para concretizar sus modelos y propuestas.
Este curso introdución a los participantes en los conceptos fundamentales de este arte. 
Lo tópicos incluyen tipos de datos, estructuras de control, funciones, listas, recursividad y la mecánica de la ejecución, prueba y depuración.
\end{justification}
 
\begin{goals}
\item Introducir los conceptos fundamentales de programación.
\item Desarrollar su capacidad de abstracción utilizar un lenguaje de programación.
\end{goals}


% See: ../Curricula.in/lang/Espanol/cycle/2020-I/Syllabi/Computing/CS/CS111.tex

\begin{unit}{Historia de los Lenguajes de Programación}{Historia de los Lenguajes de Programación}{Brookshear2019}{5}{a}
    \begin{topics}
        \item Perspectiva histórica de los lenguajes de programación.
        \item Conceptos de Programación tradicionales.
    \end{topics}
    
    \begin{learningoutcomes}
        \item \SPHistoryLOIdentifySignificant [\Familiarity]
        \item \SPHistoryLODiscussTheForLanguage [\Familiarity]
    \end{learningoutcomes}
\end{unit}


\begin{unit}{\SDFFundamentalProgrammingConcepts}{}{Guttag13,Zelle10}{9}{a,1}
    \begin{topics}
        \item \SDFFundamentalProgrammingConceptsTopicBasic
        \item \SDFFundamentalProgrammingConceptsTopicVariables
        \item \SDFFundamentalProgrammingConceptsTopicExpressions
	\item Operaciones básicas I/O.
        \item \SDFFundamentalProgrammingConceptsTopicConditional
	\item  Funciones definidas por el usuario.
        \item \SDFFundamentalProgrammingConceptsTopicFunctions
        \item \SDFFundamentalProgrammingConceptsTopicThe
    \end{topics}

    \begin{learningoutcomes}
        \item \SDFFundamentalProgrammingConceptsLOAnalyzeAndBehavior [\Assessment] 
        \item \SDFFundamentalProgrammingConceptsLOIdentifyAndOf [\Familiarity]
        \item \SDFFundamentalProgrammingConceptsLOWritePrograms [\Usage] 
        \item \SDFFundamentalProgrammingConceptsLOModify [\Usage] 
        \item \SDFFundamentalProgrammingConceptsLODesignImplement [\Usage] 
        \item \SDFFundamentalProgrammingConceptsLOChooseAppropriateIteration [\Familiarity]
        \item \SDFFundamentalProgrammingConceptsLODescribeTheRecursion [\Assessment] 
        \item \SDFFundamentalProgrammingConceptsLOIdentifyTheAndCase [\Familiarity]
    \end{learningoutcomes}
\end{unit}

\begin{unit}{\ALFundamentalDataStructuresandAlgorithms}{}{Guttag13,Zelle10}{8}{a,b,1}
    \begin{topics}
       \item \ALFundamentalDataStructuresandAlgorithmsTopicSimple
       \item \ALFundamentalDataStructuresandAlgorithmsTopicWorst
       \item  Algoritmos de ordenamiento con peor caso o caso promedio en O(N lg N)
       \item \ALFundamentalDataStructuresandAlgorithmsTopicSequential
        
    \end{topics}

    \begin{learningoutcomes}
        \item \ALFundamentalDataStructuresandAlgorithmsLOImplement [\Usage] 
        \item \ALFundamentalDataStructuresandAlgorithmsLOImplementSimple [\Assessment] 
        \item \ALFundamentalDataStructuresandAlgorithmsLOBe [\Usage] 
        \item Discutir el tiempo de ejecución y eficiencia de memoria de los principales algoritmos de ordenamiento y búsqueda. 
        \item \ALFundamentalDataStructuresandAlgorithmsLODiscussFactors [\Familiarity]
        \item \ALFundamentalDataStructuresandAlgorithmsLODemonstrate [\Assessment] 
        \item \ALFundamentalDataStructuresandAlgorithmsLOTrace [\Usage] 
    \end{learningoutcomes}
\end{unit}

\begin{unit}{\PLObjectOrientedProgramming}{}{Guttag13,Zelle10}{4}{a,1}
    \begin{topics}
        \item Lenguajes orientados a objetos para la encapsulación: Privacidad y visibilidad de miembros de la clase.
        \item \PLObjectOrientedProgrammingTopicDefinition
        \item Subclases y herencia.
        \item \PLObjectOrientedProgrammingTopicDynamic
    \end{topics}

    \begin{learningoutcomes}
        \item \PLObjectOrientedProgrammingLODesignAndClass [\Usage] 
        \item \PLObjectOrientedProgrammingLOUseSubclassing [\Familiarity] 
        \item \PLObjectOrientedProgrammingLOCompareAndThe [\Familiarity]
        \item \PLObjectOrientedProgrammingLOExplainTheObject [\Familiarity] 
        \item Usar mecanismos de encapsulación orientada a objetos [\Familiarity].   
     \end{learningoutcomes}
\end{unit}

\begin{unit}{Programación de Video Juegos}{Programación de Video Juegos}{sweigart2012making}{}{a,b,1}
    \begin{topics}
        \item Uso de una librería de video juegos.
    \end{topics}
    \begin{learningoutcomes}    
        \item Aplicar los fundamentos y los conceptos de lenguajes de programación para el desarrollo de un video juego simple.  [\Usage] 
    \end{learningoutcomes}
\end{unit}

\begin{coursebibliography}
    \bibfile{Computing/CS/CS111}
\end{coursebibliography}

\end{syllabus}
