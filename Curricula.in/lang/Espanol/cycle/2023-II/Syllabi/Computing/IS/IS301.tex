
\begin{syllabus}

\course{IS3001. Diseño y Arquitectura de Sistemas de Negocios }{Obligatorio}{IS3001}
% Source file: ../Curricula.in/lang/Espanol/cycle/2020-I/Syllabi/Computing/IS/IS301.tex

\begin{justification}
 La forma en que los componentes de software (subrutinas, clases, funciones, etc.) están organizados, y las interacciones entre ellos, se llama arquitectura. En este curso, estudiará las formas en que se representan estas arquitecturas, tanto en UML como en otras herramientas visuales. Presentaremos las arquitecturas más comunes, sus cualidades y compensaciones. Hablaremos sobre cómo se evalúan las arquitecturas, qué hace que una buena arquitectura y una arquitectura puedan mejorarse. También hablaremos sobre cómo la arquitectura toca el proceso del software.
   \end{justification}
   
   \begin{goals}
   \item Capacidad.
   \end{goals}
   
   \begin{outcomes}{V1}
      \item \ShowOutcome{a}{2}
   \end{outcomes}
   
   \begin{unit}{Introducción}{}{Sheldon,Menden}{6}{a}
   \begin{topics}
         \item Introducción I.
      \end{topics}
   
      \begin{learningoutcomes}
         \item Aprender
      \end{learningoutcomes}
   \end{unit}
   
   
   
   
   
   
   \begin{coursebibliography}
   \bibfile{Computing/IS/IS}
   \end{coursebibliography}
   
   \end{syllabus}
   
